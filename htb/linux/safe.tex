\chapter{[to finish] Safe}
\begin{itemize}
    \item {\bf technics}: 
    \item {\bf Components}: 
    \item {\bf tools}: 
\end{itemize}


\section{Résumé}


\section{Details}

\subsection{Recon}
\subsubsection{nmap}
\begin{verbatim}
PORTS=$(sudo nmap -p- -T4 --min-rate=1000 10.10.10.147 | grep '^[0-9]' |
    cut -d '/' -f 1 | tr '\n' ',' | sed s/',$'//)
sudo nmap -p$PORTS -sV -sC -oX nmap.xml 10.10.10.147


PORT     STATE SERVICE VERSION
22/tcp   open  ssh     OpenSSH 7.4p1 Debian 10+deb9u6 (protocol 2.0)
| ssh-hostkey: 
|   2048 6d:7c:81:3d:6a:3d:f9:5f:2e:1f:6a:97:e5:00:ba:de (RSA)
|   256 99:7e:1e:22:76:72:da:3c:c9:61:7d:74:d7:80:33:d2 (ECDSA)
|_  256 6a:6b:c3:8e:4b:28:f7:60:85:b1:62:ff:54:bc:d8:d6 (ED25519)
80/tcp   open  http    Apache httpd 2.4.25 ((Debian))
|_http-title: Apache2 Debian Default Page: It works
|_http-server-header: Apache/2.4.25 (Debian)
1337/tcp open  waste?
| fingerprint-strings: 
|   DNSStatusRequestTCP: 
|     00:58:13 up 2 min, 0 users, load average: 0.00, 0.00, 0.00
|   DNSVersionBindReqTCP: 
|     00:58:08 up 2 min, 0 users, load average: 0.00, 0.00, 0.00
|   GenericLines: 
|     00:57:57 up 2 min, 0 users, load average: 0.01, 0.00, 0.00
|     What do you want me to echo back?
|   GetRequest: 
|     00:58:03 up 2 min, 0 users, load average: 0.00, 0.00, 0.00
|     What do you want me to echo back? GET / HTTP/1.0
|   HTTPOptions: 
|     00:58:03 up 2 min, 0 users, load average: 0.00, 0.00, 0.00
|     What do you want me to echo back? OPTIONS / HTTP/1.0
|   Help: 
|     00:58:18 up 2 min, 0 users, load average: 0.00, 0.00, 0.00
|     What do you want me to echo back? HELP
|   NULL: 
|     00:57:57 up 2 min, 0 users, load average: 0.01, 0.00, 0.00
|   RPCCheck: 
|     00:58:03 up 2 min, 0 users, load average: 0.00, 0.00, 0.00
|   RTSPRequest: 
|     00:58:03 up 2 min, 0 users, load average: 0.00, 0.00, 0.00
|     What do you want me to echo back? OPTIONS / RTSP/1.0
|   SSLSessionReq, TLSSessionReq, TerminalServerCookie: 
|     00:58:19 up 2 min, 0 users, load average: 0.00, 0.00, 0.00
|_    What do you want me to echo back?

\end{verbatim}

\subsubsection{1337}

il fait un echo back jusqu'a une certaine longueur de caractère après il ne
repond plus

\begin{verbatim}

\end{verbatim}

\subsubsection{http}
\begin{verbatim}

<!DOCTYPE html PUBLIC "-//W3C//DTD XHTML 1.0 Transitional//EN" "http://www.w3.org/TR/xhtml1/DTD/xhtml1-transitional.dtd">
<html xmlns="http://www.w3.org/1999/xhtml">
<!-- 'myapp' can be downloaded to analyze from here
     its running on port 1337 -->
  <head>


  curl -O http://10.10.10.147/myapp
\end{verbatim}


\subsubsection{myapp}

\begin{verbatim}
$ file myapp
myapp: ELF 64-bit LSB executable, x86-64, version 1 (SYSV), dynamically linked, interpreter /lib64/ld-linux-x86-64.so.2, for GNU/Linux 3.2.0, BuildID[sha1]=fcbd5450d23673e92c8b716200762ca7d282c73a, not stripped



$ strings myapp
/lib64/ld-linux-x86-64.so.2
libc.so.6
gets
puts
printf
system
__libc_start_main
GLIBC_2.2.5
__gmon_start__
H=H@@
[]A\A]A^A_
/usr/bin/uptime
What do you want me to echo back?
;*3$"
GCC: (Debian 8.3.0-6) 8.3.0
crtstuff.c
deregister_tm_clones
__do_global_dtors_aux
\end{verbatim}

avec ghidra on a
\begin{verbatim}
undefined8 main(void)

{
  char local_78 [112];
  
  system("/usr/bin/uptime");
  printf("\nWhat do you want me to echo back? ");
  gets(local_78);
  puts(local_78);
  return 0;
}
\end{verbatim}
donc overflow

\begin{verbatim}
 checksec --file=myapp
RELRO           STACK CANARY      NX            PIE             RPATH      RUNPATH      Symbols         FORTIFY Fortified       Fortifiable       FILE
Partial RELRO   No canary found   NX enabled    No PIE          No RPATH   No RUNPATH   65 Symbols        No    0               2myapp
\end{verbatim}

donc pas d'appel sur la pile.

\begin{verbatim}

\end{verbatim}


\begin{verbatim}

\end{verbatim}


\begin{verbatim}

\end{verbatim}


\begin{verbatim}

\end{verbatim}


\begin{verbatim}

\end{verbatim}


\begin{verbatim}

\end{verbatim}


\begin{verbatim}

\end{verbatim}


\begin{verbatim}

\end{verbatim}




\subsection{Foothold}
\subsubsection{x}
\begin{verbatim}
\end{verbatim}

\subsubsection{x}

\begin{verbatim}

\end{verbatim}

\begin{verbatim}

\end{verbatim}
\section{Theorie}


