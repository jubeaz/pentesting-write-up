\chapter{Blocky}
\begin{itemize}
    \item {\bf technics}: 
    \item {\bf Components}: 
    \item {\bf tools}: 
\end{itemize}


\section{Résumé}


\section{Details}

\subsection{Recon}
\subsubsection{nmap}
\begin{verbatim}
TARGET_IP=10.10.10.37
$ PORTS=$( sudo nmap --min-rate=1000 -T4 -p- $TARGET_IP | grep '^[0-9]' | 
    cut -d'/' -f 1 | tr '\n' ',' | sed s/',$'//)
$ echo $PORTS
$ sudo nmap -sVC -p$PORTS $TARGET_IP


PORT      STATE  SERVICE   VERSION
21/tcp    open   ftp       ProFTPD 1.3.5a
22/tcp    open   ssh       OpenSSH 7.2p2 Ubuntu 4ubuntu2.2 (Ubuntu Linux; protocol 2.0)
| ssh-hostkey:
|   2048 d6:2b:99:b4:d5:e7:53:ce:2b:fc:b5:d7:9d:79:fb:a2 (RSA)
|   256 5d:7f:38:95:70:c9:be:ac:67:a0:1e:86:e7:97:84:03 (ECDSA)
|_  256 09:d5:c2:04:95:1a:90:ef:87:56:25:97:df:83:70:67 (ED25519)
80/tcp    open   http      Apache httpd 2.4.18
|_http-title: Did not follow redirect to http://blocky.htb
|_http-server-header: Apache/2.4.18 (Ubuntu)
8192/tcp  closed sophos
25565/tcp open   minecraft Minecraft 1.11.2 (Protocol: 127, Message: A Minecraft Server, Users: 0/20)
Service Info: Host: 127.0.1.1; OSs: Unix, Linux; CPE: cpe:/o:linux:linux_kernel

\end{verbatim}

\subsubsection{http}


We are currently developing a wiki system for the server and a core plugin to
track player stats and stuff. Lots of great stuff planned for the future

\verb+wpscan+ indique:
\begin{itemize}
    \item akismet v3.3.2
    \item twentyseventeen 
\end{itemize}

\begin{verbatim}
$ ffuf -u http://blocky.htb/FUZZ -w /usr/share/wordlists/seclists/Discovery/Web-Content/directory-list-2.3-medium.txt
wiki                    [Status: 301, Size: 307, Words: 20, Lines: 10, Duration: 76ms]
wp-content              [Status: 301, Size: 313, Words: 20, Lines: 10, Duration: 75ms]
plugins                 [Status: 301, Size: 310, Words: 20, Lines: 10, Duration: 75ms]
\end{verbatim}

dans plugins on trouve 2 fichiers

on trouve également:
\begin{verbatim}
[i] User(s) Identified:

[+] notch
 | Found By: Author Posts - Author Pattern (Passive Detection)
 | Confirmed By:
 |  Wp Json Api (Aggressive Detection)
 |   - http://blocky.htb/index.php/wp-json/wp/v2/users/?per_page=100&page=1
 |  Author Id Brute Forcing - Author Pattern (Aggressive Detection)
 |  Login Error Messages (Aggressive Detection)

[+] Notch
 | Found By: Rss Generator (Passive Detection)
 | Confirmed By: Login Error Messages (Aggressive Detection)
 \end{verbatim}


\subsubsection{Minecreft}

installation d'un
\href{https://github.com/sjhilt/Nmap-NSEs/blob/master/minecraft-info.nse}{nse
pour minecraft}:

\begin{verbatim}

$ sudo nmap -p25565 
    --script=minecraft-info blocky.htb
PORT      STATE SERVICE
25565/tcp open  minecraft
| minecraft-info:
|   Description:
|     text: A Minecraft Server
|   Max Players: 20
|   Players Online: 0
|   Version: 1.11.2
|_  Protocol: 316
\end{verbatim}

serait vulnerable à log4j ?

\subsection{Binary}
\subsubsection{BlockyCore}
\begin{verbatim}
package com.myfirstplugin;

public class BlockyCore {
  public String sqlHost = "localhost";

  public String sqlUser = "root";

  public String sqlPass = "8YsqfCTnvxAUeduzjNSXe22";

  public void onServerStart() {}

  public void onServerStop() {}

  public void onPlayerJoin() {
    sendMessage("TODO get username", "Welcome to the BlockyCraft!!!!!!!");
  }

  public void sendMessage(String username, String message) {}
}
\end{verbatim}

\subsubsection{griefprevention}


href{https://docs.griefprevention.com/}{GriefPrevention} GriefPrevention is
compatible with Purpur, Spigot, Paper, and any other implemention of the Bukkit
API.

\subsection{ftp}

avec le user et le mdp (\verb+8YsqfCTnvxAUeduzjNSXe22+) on peut se connecter au ftp.

\begin{verbatim}
$ ftp notch@blocky.htb

\end{verbatim}

\begin{verbatim}
$ curlftpfs blocky.htb tmp -o user=notch:8YsqfCTnvxAUeduzjNSXe22

$ cat start.sh
if ! screen -list | grep blockycraft > /dev/null
then
        screen -dmS blockycraft java -Xms500M -Xmx500M -jar ./sponge.jar nogui
fi
\end{verbatim}


on trouve que le port closed est lié à \verb+nuvotifier+ (version 1-2.3.4):

NuVotifier is a plugin that allows your server to be notified (aka votified)
when a vote is made on a Minecraft server top list. NuVotifier is secure, and
makes sure that all vote notifications are delivered by authentic top lists.

\begin{verbatim}
$ cat config/nuvotifier/config.yml
# The IP to listen to. Use 0.0.0.0 if you wish to listen to all interfaces on your server. (All IP addresses)
# This defaults to the IP you have configured your server to listen on, or 0.0.0.0 if you have not configured this.
host: 127.0.0.1

# Port to listen for new votes on
port: 8192

# Whether or not to print debug messages. In a production system, this should be set to false.
# This is useful when initially setting up NuVotifier to ensure votes are being delivered.
debug: false

# Setting this value to false will turn off the votifier port listening for external votes. This is beneficial if the server
# is only listening for votes coming from NuVotifier running on your BungeeCord.
enableExternal: true

# Setting this option to true will disable handling of Protocol v1 packets. While the old protocol is not secure, this
# option is currently not recommended as most voting sites only support the old protocol at present. However, if you are
# using NuVotifier's proxy forwarding mechanism, enabling this option will increase your server's security.
disable-v1-protocol: false

# All tokens, labeled by the serviceName of each server list.
tokens:
  # Default token for all server lists, if another isn't supplied.
  default: atksa8vubshu31vktargk27c6v

# Configuration section for all vote forwarding to NuVotifier
forwarding:
  # Sets whether to set up a remote method for fowarding. Supported methods:
  # - none - Does not set up a forwarding method.
  # - pluginMessaging - Sets up plugin messaging
  method: none
  pluginMessaging:
    channel: NuVotifier
\end{verbatim}

\subsection{Binary}
\subsubsection{sponge}

\url{https://www.spongepowered.org}: A community-driven open source Minecraft: Java Edition modding platform.

\begin{verbatim}
SpongeAPI version Minecraft version Minimum Java Version Notes 
7.x 1.12.2 Java 8 Requires at least update 20. Does not work with Java 9 or later.
\end{verbatim}

les plugins sont installés dans \verb+/mods+ et l'on retrouve blocky nuvotifier 

\begin{verbatim}
Manifest.MF:

Manifest-Version: 1.0
Implementation-Title: SpongeVanilla
Implementation-Version: 1.11.2-6.1.0-BETA-11
Specification-Vendor: SpongePowered
Specification-Title: SpongeAPI
FMLAT: common_at.cfg vanilla_at.cfg
Class-Path: minecraft_server.1.11.2.jar libraries/net/minecraft/launch
 wrapper/1.12/launchwrapper-1.12.jar asm-all-5.2.jar
TweakClass: org.spongepowered.server.launch.VanillaServerTweaker
Implementation-Vendor: SpongePowered
Main-Class: org.spongepowered.server.launch.VersionCheckingMain
Git-Commit: 22dc7823f97933e2ab5bddcbbd2a712d6361ca91
Git-Branch: origin/stable-6
Specification-Version: 6.1.0-SNAPSHOT-a15cc3e
Created-By: 1.8.0_131 (Oracle Corporation)
\end{verbatim}

on retrouve un log4j \verb+org.apache.logging.log4j.core.config.plugins+

\begin{verbatim}
log4j2.xml

<?xml version="1.0" encoding="UTF-8"?>
<Configuration status="warn">
    <Appenders>
        <TerminalConsole name="Console">
            <PatternLayout pattern="[%d{HH:mm:ss} %level] [%logger{1}]: %msg%n"/>
        </TerminalConsole>
        <TerminalConsole name="MinecraftConsole">
            <PatternLayout pattern="[%d{HH:mm:ss} %level]: %msg%n"/>
        </TerminalConsole>

        <!-- Keep a console appender open so log4j2 doesn't close our main out stream if we redirect System.out to the logger -->
        <Console name="SysOut" target="SYSTEM_OUT"/>

        <RollingRandomAccessFile name="File" fileName="logs/latest.log" filePattern="logs/%d{yyyy-MM-dd}-%i.log.gz">
            <PatternLayout pattern="[%d{HH:mm:ss}] [%t/%level] [%logger{1}]: %replace{%msg}{(?i)\u00A7[0-9A-FK-OR]}{}%n"/>
            <Policies>
                <TimeBasedTriggeringPolicy/>
                <OnStartupTriggeringPolicy/>
            </Policies>
        </RollingRandomAccessFile>
    </Appenders>

    <Loggers>
        <!-- Log Minecraft messages without prefix -->
        <Logger name="net.minecraft" level="info" additivity="false">
            <filters>
                <MarkerFilter marker="NETWORK_PACKETS" onMatch="DENY" onMismatch="NEUTRAL"/>
            </filters>
            <AppenderRef ref="MinecraftConsole"/>
            <AppenderRef ref="File"/>
        </Logger>
        <Logger name="com.mojang" level="info" additivity="false">
            <AppenderRef ref="MinecraftConsole"/>
            <AppenderRef ref="File"/>
        </Logger>

        <!-- Hide LaunchWrapper messages unless it's a real problem -->
        <Logger name="LaunchWrapper" level="all" additivity="false">
            <AppenderRef ref="Console" level="warn"/>
            <AppenderRef ref="File"/>
        </Logger>
        <!-- Hide HikariCP messages unless it's a real problem -->
        <Logger name="com.zaxxer.hikari" level="info" additivity="false">
            <AppenderRef ref="Console" level="warn"/>
            <AppenderRef ref="File"/>
        </Logger>

        <!-- Log all other messages with prefix -->
        <Root level="all">
            <AppenderRef ref="Console" level="info"/>
            <AppenderRef ref="File"/>
        </Root>
    </Loggers>
</Configuration>

\end{verbatim}

\subsubsection{nuvotifier}

\begin{verbatim}
name: Votifier
main: com.vexsoftware.votifier.NuVotifierBukkit
version: 2.3.4
description: A plugin that gets notified when votes are made for the server on toplists.
authors: [blakeman8192, Kramer, tuxed]
\end{verbatim}

en regardant le code source il semblerait que ce soit lui qui aille logger des
choses.


\subsection{Foothold}
\subsubsection{ssh}
comme tout au dessus semble très compliqué on va regarer si on peut se
connecter en ssh et trouver un moyen depuis l'interieur.
\begin{verbatim}
$ sudo -l
[sudo] password for notch:
Matching Defaults entries for notch on Blocky:
    env_reset, mail_badpass, secure_path=/usr/local/sbin\:/usr/local/bin\:/usr/sbin\:/usr/bin\:/sbin\:/bin\:/snap/bin

User notch may run the following commands on Blocky:
    (ALL : ALL) ALL
\end{verbatim}

\subsubsection{x}

\begin{verbatim}

\end{verbatim}

\begin{verbatim}

\end{verbatim}
\section{Theorie}


