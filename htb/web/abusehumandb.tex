\section{[TO FINISH] AbuseHumanDB}

Le flag est en BDD. Son engeristrement n'est pas approuvé.

les routes:
\begin{itemize}
    \item \verb+/+
    \item \verb+/entries+: appel l'api \verb+api/entries+
    \item \verb+/api/entires+
        \begin{itemize}
            \item \verb+GET+: liste toutes les entrée si \verb+isLocalhost(req)+sinon seulement celles
                approuved.
            \item \verb+POST+: appel \verb+bot.visitPage(url)+ après validation
                de l'url sans rien inserer en bd.
        \end{itemize}
    \item \verb+/api/entries/search+ (\verb+GET+) liste les entrées en filtrant
        sur le titre et en appliquant le même filtre controle sur approved en
        fonction de \verb+isLocalhost+
    \item \verb+/+
    \item \verb+/+
\end{itemize}


\begin{verbatim}
const isLocalhost = req => ((req.ip == '127.0.0.1' && req.headers.host == '127.0.0.1:1337') ? 0 : 1);
\end{verbatim}

Le bot ne fait juste que visiter une page.

client side \verb+entries.js+:
deux fonctions:
\begin{itemize}
    \item \verb+loadEntries+: fetch sur \verb+/api/entries+ en \verb+GET+
    \item \verb+searchEntry+ fetch sur \verb+/api/entries/search+ en \verb+GET+
\end{itemize}

client side \verb+imain.js+: effectue l'appel en \verb+POST+ sur
\verb+api/enties+


donc il faut que l'appel soit fait en local sauf que le seul qui peut faire
cela c'est la bot.

si on lui dit de venir lire une page que l'on heberge et que cette page
contient un fetch vers soit \verb+/api/search+ ou \verb+api/entries+ cela
passerait le controle. Mais le soucis c'est qu'il ne nous donnera jamais
d'info.


\href{https://cheatsheetseries.owasp.org/cheatsheets/XS_Leaks_Cheat_Sheet.html}{Cross-site
leaks Cheat Sheet}
