\section{TrueSecrets}


donc on a un fichier \verb+raw+ 

\begin{verbatim}
$ file TrueSecrets.raw
TrueSecrets.raw: data
\end{verbatim}

l'hitroire nous dit que c'est une catpure memoire mais quel OS :)

a mon avs du windows 
\begin{verbatim}
$ strings  TrueSecrets.raw |more
...SNIP...
sy@%systemroot%\system32\oobefldr.dll,-1142

\end{verbatim}


volatility

Volatility is an open-source memory forensics framework for incident response
and malware analysis. This is a very powerful tool and we can complete lots of
interactions with memory dump files


\begin{verbatim}
$ sudo vol -f TrueSecrets.raw windows.info
Volatility 3 Framework 2.0.1
Progress:  100.00               PDB scanning finished
Variable        Value

Kernel Base     0x82606000
DTB     0x185000
Symbols file:///usr/lib/python3.10/site-packages/volatility3/symbols/windows/ntkrpamp.pdb/92D32EE7188A4CB3AB23EDA0CB0F9D7B-2.json.xz
Is64Bit False
IsPAE   True
layer_name      0 WindowsIntelPAE
memory_layer    1 FileLayer
KdDebuggerDataBlock     0x82732c78
NTBuildLab      7601.23915.x86fre.win7sp1_ldr.17
CSDVersion      1
KdVersionBlock  0x82732c50
Major/Minor     15.7601
MachineType     332
KeNumberProcessors      1
SystemTime      2022-12-14 21:33:30
NtSystemRoot    C:\Windows
NtProductType   NtProductWinNt
NtMajorVersion  6
NtMinorVersion  1
PE MajorOperatingSystemVersion  6
PE MinorOperatingSystemVersion  1
PE Machine      332
PE TimeDateStamp        Wed Sep 13 14:47:57 2017
\end{verbatim}

On liste les process running:
\begin{verbatim}
$ sudo vol -f TrueSecrets.raw windows.pslist
Volatility 3 Framework 2.0.1
Progress:  100.00               PDB scanning finished
PID     PPID    ImageFileName   Offset(V)       Threads Handles SessionId       Wow64   CreateTime      ExitTime        File output

...SNIP...
2128    1464    TrueCrypt.exe   0x91892030      4       262     1       False   2022-12-14 21:08:31.000000      N/A     Disabled
2760    452     svchost.exe     0x91865790      13      362     0       False   2022-12-14 21:10:23.000000      N/A     Disabled
2332    584     WmiPrvSE.exe    0x83911848      5       112     0       False   2022-12-14 21:12:23.000000      N/A     Disabled
2580    452     taskhost.exe    0x8e1ef208      5       86      1       False   2022-12-14 21:13:01.000000      N/A     Disabled
2176    1464    7zFM.exe        0x8382f198      3       135     1       False   2022-12-14 21:22:44.000000      N/A     Disabled
3212    1464    DumpIt.exe      0x83c1d030      2       38      1       False   2022-12-14 21:33:28.000000      N/A     Disabled
\end{verbatim}

donc le dump memoire a été fait avec dumpit.exe
On regarde les commandlines
\begin{verbatim}
$ sudo vol -f TrueSecrets.raw windows.cmdline
Volatility 3 Framework 2.0.1
Progress:  100.00               PDB scanning finished
PID     Process Args

4       System  Required memory at 0x10 is not valid (process exited?)
...SNIP...
2128    TrueCrypt.exe   "C:\Program Files\TrueCrypt\TrueCrypt.exe"
2760    svchost.exe     C:\Windows\System32\svchost.exe -k secsvcs
2332    WmiPrvSE.exe    C:\Windows\system32\wbem\wmiprvse.exe
2580    taskhost.exe    Required memory at 0x1e1004 is inaccessible (swapped)
2176    7zFM.exe        "C:\Program Files\7-Zip\7zFM.exe" "C:\Users\IEUser\Documents\backup_development.zip"
...SNIP...
\end{verbatim}


\begin{verbatim}
$ sudo vol -f TrueSecrets.raw windows.filescan |grep Documents
0x724c038  100.0\Users\IEUser\Documents 128
0xbbf6158       \Users\IEUser\Documents\backup_development.zip  128
0xc50c550       \Users\IEUser\Documents\development.tc  128
0xc520b68       \Users\IEUser\Documents\desktop.ini     128
0xc54a0d8       \Users\Public\Documents 128
0xc56e038       \Users\Public\Documents 128
\end{verbatim}

\begin{verbatim}
$ sudo vol -f TrueSecrets.raw windows.dumpfiles -h
Volatility 3 Framework 2.0.1
usage: volatility windows.dumpfiles.DumpFiles [-h] [--pid PID] [--virtaddr VIRTADDR] [--physaddr PHYSADDR]

options:
  -h, --help           show this help message and exit
  --pid PID            Process ID to include (all other processes are excluded)
  --virtaddr VIRTADDR  Dump a single _FILE_OBJECT at this virtual address
  --physaddr PHYSADDR  Dump a single _FILE_OBJECT at this physical address

$ sudo vol -f TrueSecrets.raw windows.dumpfiles
...SNIP...

$ file file.2176.0x839339d0.backup_development.zip.dat
file.2176.0x839339d0.backup_development.zip.dat: Zip archive data, at least v1.0 to extract, compression method=store
\end{verbatim}

tc is an extension of TrueCrypt

on peut dumper le process trueCrypt

\begin{verbatim}
$ python2 vol.py -f TrueSecrets.raw --profile=Win7SP0x86 memdump -p 2128 -D tmp 
\end{verbatim}


\begin{verbatim}
$ export VOLATILITY_PROFILE=Win7SP0x86   
\end{verbatim}

\url{https://www.youtube.com/watch?v=A2d2OFGSnKU}


\begin{verbatim}
$ git clone https://github.com/volatilityfoundation/community.git
\end{verbatim}


\begin{verbatim}
 python2 vol.py -f TrueSecrets.raw --plugins=./community truecryptpassphrase
Found at 0x89ebf064 length 28: X2Hk2XbEJqWYsh8VdbSYg6WpG9g7

$ python2 vol.py -f TrueSecrets.raw --plugins=./community truecryptsummary
Volatility Foundation Volatility Framework 2.6.1
*** Failed to import volatility.plugins.malware.apihooks (NameError: name 'distorm3' is not defined)
*** Failed to import volatility.plugins.malware.threads (NameError: name 'distorm3' is not defined)
*** Failed to import volatility.plugins.mac.apihooks_kernel (ImportError: No module named distorm3)
*** Failed to import volatility.plugins.mac.check_syscall_shadow (ImportError: No module named distorm3)
*** Failed to import volatility.plugins.ssdt (NameError: name 'distorm3' is not defined)
*** Failed to import volatility.plugins.mac.apihooks (ImportError: No module named distorm3)
Password             X2Hk2XbEJqWYsh8VdbSYg6WpG9g7 at offset 0x89ebf064
Process              TrueCrypt.exe at 0x91892030 pid 2128
Service              truecrypt state SERVICE_RUNNING
Kernel Module        truecrypt.sys at 0x89e8b000 - 0x89ec2000
Symbolic Link        D: -> \Device\TrueCryptVolumeD mounted 2022-12-14 21:33:00 UTC+0000
Symbolic Link        Volume{d22d7a9d-7b72-11ed-b81d-0800273bf313} -> \Device\TrueCryptVolumeD mounted 2022-12-14 21:10:21 UTC+0000
Symbolic Link        D: -> \Device\TrueCryptVolumeD mounted 2022-12-14 21:33:00 UTC+0000
Driver               \Driver\truecrypt at 0xbe6b780 range 0x89e8b000 - 0x89ec1b80
Device               TrueCryptVolumeD at 0x8391b9b0 type FILE_DEVICE_DISK
Container            Path: \??\C:\Users\IEUser\Documents\development.tc
Device               TrueCrypt at 0x83e6b600 type FILE_DEVICE_UNKNOWN
\end{verbatim}


\begin{verbatim}
$ sudo cryptsetup open --type tcrypt development.tc PrivateDatas 
Enter passphrase for development.tc: 
$ sudo mount -o uid=1000 /dev/mapper/PrivateDatas /mnt
\end{verbatim}


donc maintenant on a un programme

\begin{verbatim}
$ strings AgentServer.cs                              
using System;
using System.IO;
using System.Net;
using System.Net.Sockets;
using System.Text;
using System.Security.Cryptography;
class AgentServer {
    static void Main(String[] args)
    {
        var localPort = 40001;
        IPAddress localAddress = IPAddress.Any;
        TcpListener listener = new TcpListener(localAddress, localPort);
        listener.Start();
        Console.WriteLine("Waiting for remote connection from remote agents (infected machines)...");
    
        TcpClient client = listener.AcceptTcpClient();
        Console.WriteLine("Received remote connection");
        NetworkStream cStream = client.GetStream();
    
        string sessionID = Guid.NewGuid().ToString();
    
        while (true)
        {
            string cmd = Console.ReadLine();
            byte[] cmdBytes = Encoding.UTF8.GetBytes(cmd);
            cStream.Write(cmdBytes, 0, cmdBytes.Length);
            
            byte[] buffer = new byte[client.ReceiveBufferSize];
            int bytesRead = cStream.Read(buffer, 0, client.ReceiveBufferSize);
            string cmdOut = Encoding.ASCII.GetString(buffer, 0, bytesRead);
            
            string sessionFile = sessionID + ".log.enc";
            File.AppendAllText(@"sessions\" + sessionFile, 
                Encrypt(
                    "Cmd: " + cmd + Environment.NewLine + cmdOut
                ) + Environment.NewLine
            );
        }
    }
    
    private static string Encrypt(string pt)
    {
        string key = "AKaPdSgV";
        string iv = "QeThWmYq";
        byte[] keyBytes = Encoding.UTF8.GetBytes(key);
        byte[] ivBytes = Encoding.UTF8.GetBytes(iv);
        byte[] inputBytes = System.Text.Encoding.UTF8.GetBytes(pt);
        
        using (DESCryptoServiceProvider dsp = new DESCryptoServiceProvider())
        {
            var mstr = new MemoryStream();
            var crystr = new CryptoStream(mstr, dsp.CreateEncryptor(keyBytes, ivBytes), CryptoStreamMode.Write);
            crystr.Write(inputBytes, 0, inputBytes.Length);
            crystr.FlushFinalBlock();
            return Convert.ToBase64String(mstr.ToArray());
        }
    }
       
\end{verbatim}

on passe par cyberchef mais ne marche pas trop bien

donc en toute logique il faut pour chaque ligne
\begin{itemize}
    \item decoder le base64
    \item decrypter le DES \verb+(key = "AKaPdSgV", iv = "QeThWmYq";)+
\end{itemize}

\begin{verbatim}
from Crypto.Cipher import DES
from base64 import b64decode

key = b"AKaPdSgV"
iv = b"QeThWmYq"

desd = DES.new(key, DES.MODE_CBC, iv)

messages = [
    'wENDQtz...SNIP...7a4Plq8h68=',
    ...SNIP...
]

for mes in messages:
    mes_dec = b64decode(mes)
    plaintext = desd.decrypt(mes_dec)
    print(plaintext)
\end{verbatim}
