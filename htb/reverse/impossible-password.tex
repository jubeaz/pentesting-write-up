\section{Impossible password}

en gros main a un premier imput qui compare à \verb+SuperSeKretKey+

puis un deuxieme qui compare à une chaine aléatoire.

si la comparaison est bonne il decode 20 chars sur la stack de main en faisant
un xor 9

comme pas online challange on peut bypass les controles des valeurs saisies
facilement en éditant les registres dans gdb ou même si on veut on edit
directement le binaire pour inverser le checks

\begin{verbatim}
pwndbg> break *0x4006a0
Breakpoint 1 at 0x4006a0
pwndbg> r
Starting program: /home/<REDACTED>/documents/pentesting-games/htb/challenges/reverse/rev_impossiblepassword/impossible_password.bin
[Thread debugging using libthread_db enabled]
Using host libthread_db library "/usr/lib/libthread_db.so.1".

Breakpoint 1, 0x00000000004006a0 in ?? ()

pwndbg> break *0x40085d
Breakpoint 2 at 0x40085d    ; main
pwndbg> c
Continuing.

Breakpoint 2, 0x000000000040085d in ?? ()

pwndbg> b *0x00400968 ; before decode stack
Breakpoint 3 at 0x400968
pwndbg> c
Continuing.
* SuperSeKretKey
[SuperSeKretKey]
** aaa

Breakpoint 3, 0x0000000000400968 in ?? ()

pwndbg> help eflags
Undefined command: "eflags".  Try "help".
pwndbg> $eflags
Undefined command: "$eflags".  Try "help".
pwndbg> i r eflags
eflags         0x282               [ SF IF ]
pwndbg> set ($eflags)|=0x42

pwndbg> i r eflags
eflags         0x2c2               [ ZF SF IF ]
pwndbg> c
Continuing.
HTB{40b949f92b86b18}
[Inferior 1 (process 864168) exited with code 012]
\end{verbatim}

