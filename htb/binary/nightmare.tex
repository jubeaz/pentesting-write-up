\chapter{Nightmare}
\begin{itemize}
    \item {\bf technics}: 
    \item {\bf tools}: 
\end{itemize}


\section{Résumé}

\section{Analysis}
\subsection{File info}
\begin{verbatim}
$ file pwnshop
\end{verbatim}

\subsection{Security}
\begin{verbatim}
$ pwntools-pwn checksec --file=./nightmare
    Arch:     amd64-64-little
    RELRO:    No RELRO
    Stack:    Canary found
    NX:       NX enabled
    PIE:      PIE enabled
\end{verbatim}

\begin{verbatim}
$ echo 0 | sudo tee  /proc/sys/kernel/randomize_va_spac
\end{verbatim}

donc va falloir leak \verb+piebase+

\subsection{Execution}


\section{static analysis}

en regardant vite fait avec ghidra.

la fonction \verb+scream()+ est vulnérable au format string. et devrait
permettre de leak le canary. Pas d'overflow mais gros buffer.

la fonction \verb+escape()+ est la seule contant un return et va donc permettre
de faire un ret2libc par contre il risque de falloir faire un stack pivoting

\section{Exploit}

\subsection{Strategy}


\subsection{Exploit build}


\begin{verbatim}
$ pwntools-pwn template --host 10.10.10.10 --port 10 ./ > pwn-bin.py
\end{verbatim}


